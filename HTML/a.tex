\documentclass{article}

\newcounter{example}[section]
\newenvironment{example}[1][]{\refstepcounter{example}\par\medskip
   \noindent\textbf{Example~\theexample. #1} \rmfamily}{\medskip}

\begin{document}

	\tableofcontents
	\newpage
	\listoffigures
	\newpage
	\listoftables
\section{Three examples}
\begin{example}
Create a label in this first example \verb|\label{ex:1}|\label{ex:1}. This is the first example. The \texttt{example} counter will be reset at the start of each new each document \verb|\section|.
\end{example}

\begin{example}
And here's another numbered example. Create a second \verb|\label{ex:2}|\label{ex:2} to later reference this one. In Example \ref{ex:1} we read... something.
\end{example}

\begin{example}
And here's another numbered example: use \verb|\theexample| to typeset the number currently assigned to the \texttt{example} counter: it is  \theexample.
\end{example}

\section{Another section}
We've just started a new section meaning that the  \texttt{example} counter has been set to \theexample.
We'll reference examples from the previous section (Examples \ref{ex:1} and \ref{ex:2}).  This is a dummy section with no purpose whatsoever but to contain text. The \texttt{section} counter for this section can be typeset using \verb|\thesection|: it is  currently assigned the value of \thesection.

\begin{example}
This is the first example in this section: the \texttt{example} counter has been stepped and now set to \theexample.
\end{example}
\end{document}
