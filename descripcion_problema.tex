\setcounter{page}{1}

\begin{center}
	\section{Planteamiento del problema}
\end{center}


\subsection{Descripción del problema}

	En la actualidad se vive un crecimiento exponencial a nivel industrial dadas las altas demandas de alimentos e insumos de toda clase; Esto es posible gracias al motor eléctrico, este es un artefacto que transforma la energía eléctrica en energía mecánica (movimiento), de manera que puede impulsar el funcionamiento de una máquina y son utilizados ampliamente en: Bombas para Agua, Bombas Industriales, Mezcladoras, Molinos, Correas Transportadoras, Zarandas, Cortadoras, Ventiladores, Grúas, y en todo proceso que involucre movimiento.\\
	Si a esta diversidad de usos se le suma la gran perdida (horas, insumos,dinero, etc) que ocasiona una parada de emergencia en una planta, Se obtiene la gran importancia de que los motores se encuentren completamente operativos y funcionales. Para procurar su buen estado se realizan mantenimientos.\\



	Existen diferentes tipos de mantenimiento, entre ellos están: 
	\begin{itemize}
		\item correctivo: se espera que ocurra una falla para
	reparar o cambiar un equipo. Esto puede degradar la vida útil del equipo y debido a que la falla puede ocurrir en cualquier momento, usualmente se produce un paro en la linea de producción, por lo tanto este tipo de mantenimiento suele y debe ser evitado.

		\item preventivo: para evitar una falla mayor se detiene la maquinaria para hacer un mantenimiento preparado con anticipación, se inspecciona la maquinaria y se remplazan las piezas propensas a dañarse. Este tipo de mantenimiento en algunas circunstancias es más que suficiente pero en el caso de los rodamientos puede ser contraproducente.

		\item predictivo: se predice cuando una falla esta a punto de ocurrir; A través de mediciones y estudio se predice cuando una falla esta a punto de ocurrir y de esta forma se realiza una mejor planificación. Cabe resaltar que este no solo permite predecir fallas sino que puede dar
		información acerca del origen de las mismas.
	\end{itemize}

		En el caso de los motores eléctricos, como dice Dr.S. J. Lacey en The Role of Vibration Monitoring in Predictive Maintenance, el mantenimiento preventivo tiene muchas desventajas dado que existen tanto problemas de índole mecánico como administrativo y monetario, como lo pueden ser, los altos costos de reemplazo, dado que las partes se reemplazan muy pronto, el riesgo de perdida completa dado un error humano, instalación de una pieza defectuosa, además de la posibilidad de generar daño o una incorrecta instalación de la misma, Y por ultimo, el hecho de que las piezas reemplazadas pueden tener muchos años de vida útil~\cite{Lacey}.\\
		por otro lado, el mantenimiento predictivo ofrece mas control sobre estas variables, aunque no evita las posibilidades de error humano, si permite reaccionar a este; Adicionalmente a esto, se debe considerar las altas perdidas y retrasos, ademas de dificultades administrativas, que generan las paradas periódicas de la planta.




	