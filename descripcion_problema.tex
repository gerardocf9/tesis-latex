\setcounter{page}{1}

\begin{center}
	\section{Planteamiento del problema}
\end{center}


\subsection{Descripción del problema}

	En la actualidad se vive un crecimiento exponencial a nivel industrial dadas las altas demandas de alimentos e insumos de toda clase; Esto es posible gracias al motor eléctrico, este es un artefacto que transforma la energía eléctrica en energía mecánica (movimiento), de manera que puede impulsar el funcionamiento de una máquina y son utilizados ampliamente en: Bombas para Agua, Bombas Industriales, Mezcladoras, Molinos, Correas Transportadoras, Zarandas, Cortadoras, Ventiladores, Grúas, y en todo proceso que involucre movimiento.\\
	Si a esta diversidad de usos se le suma la gran perdida (horas, insumos,dinero, etc) que ocasiona una parada de emergencia en una planta, Se obtiene la gran importancia de que los motores se encuentren completamente operativos y funcionales. Para procurar su buen estado se realizan mantenimientos.\\



	Existen diferentes tipos de mantenimiento, entre ellos están: 
	\begin{itemize}
		\item correctivo: se espera que ocurra una falla para
	reparar o cambiar un equipo. Esto puede degradar la vida útil del equipo y debido a que la falla puede ocurrir en cualquier momento, usualmente se produce un paro en la linea de producción, por lo tanto este tipo de mantenimiento suele y debe ser evitado.

		\item preventivo: para evitar una falla mayor se detiene la maquinaria para hacer un mantenimiento preparado con anticipación, se inspecciona la maquinaria y se remplazan las piezas propensas a dañarse. Este tipo de mantenimiento en algunas circunstancias es más que suficiente pero en el caso de los rodamientos puede ser contraproducente.

		\item predictivo: se predice cuando una falla esta a punto de ocurrir; A través de mediciones y estudio se predice cuando una falla esta a punto de ocurrir y de esta forma se realiza una mejor planificación. Cabe resaltar que este no solo permite predecir fallas sino que puede dar
		información acerca del origen de las mismas.
	\end{itemize}

	En el caso de los motores eléctricos, como dice Dr.S. J. Lacey en The Role of Vibration Monitoring in Predictive Maintenance ~\cite{Lacey}, el mantenimiento preventivo tiene muchas desventajas dado que existen tanto problemas de índole mecánico como administrativo y monetario, como lo pueden ser, los altos costos de reemplazo, dado que las partes se reemplazan muy pronto, el riesgo de perdida completa dado un error humano, instalación de una pieza defectuosa, además de la posibilidad de generar daño o una incorrecta instalación de la misma, Y por ultimo, el hecho de que las piezas reemplazadas pueden tener muchos años de vida útil.\\
	
	Por otro lado, el mantenimiento predictivo ofrece mas control sobre estas variables, aunque no evita las posibilidades de error humano, si permite reaccionar a este; Adicionalmente a esto, se debe considerar las altas perdidas y retrasos, ademas de dificultades administrativas, que generan las paradas periódicas de la planta.\\

	Por estas razones surge la necesidad de reconocer las fallas, dado que al estar los factores causantes de estas bajo continuo monitoreo , principio del mantenimiento predictivo, se pueden atenuar estos, según los estudios realizados por J. Kammermann, I. Bolvashenkov, S. Schwimmbeck, y H.-G. Herzog en Reliability of Induction Machines: Statistics, Tendencies, and Perspectives ~\cite{Kammermann}, la media de la probabilidad de fallas en maquinas de inducción permanece al nivel de los años 70 ($10^{-6}/hour$) y esta altamente relacionada a la falla de los rodamientos.\\

	En el mismo estudio se estipula que un 59\% de las fallas son causadas por los rodamientos, esto es debido a que son piezas sometidas a mucho estrés mecánico, permiten soporte y asimismo necesitan tener poca fricción. Por esta razon, su principal falla es el desgaste, de igual forma se presentan fallas estructurales, entre otras. Sin embargo, todas estas generan vibraciones, cabe resaltar que todas las fallas mecánicas generan vibración sin importar su relación con las rodamientos y lo hacen a distinta frecuencia ($f$). \\


	Por lo tanto, es de suma importancia el estudio de esta variable, para esto se suele implementar un acelerometro y con un estudio de la frecuencia se puede obtener la causa y la magnitud de la falla y de esta forma programar su reparación.\\
	Debido a que este estudio debe ser realizado en cada rodamiento y acople o extensión del motor y dadas las rodamientos de mediciones que se deben realizar por unidad (motor y acoples) así mismo por la gran cantidad de unidades existentes a nivel industrial, es virtualmente imposible la realización de estudio con un acelerometro convencional( a pesar de que la medición no sea un proceso muy largo y los cálculos y evaluaciones sean realizados posteriormente). Sumado a esto, las industrias suelen poseer medidas y controles sanitario, aumentados por la pandemia actual, que no permiten el constante monitoreo de la planta significando esto la imposibilidad de implementar este tipo de mantenimiento de forma manual, por lo cual se debe recurrir a un sistema de automatización capaz de medir las vibraciones en todos estos equipos y que a su vez permita el estudio de estos datos de forma remota. 

	