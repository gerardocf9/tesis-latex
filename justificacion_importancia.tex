
\subsubsection{Justificación}

A lo largo del tiempo, debido al constante uso de los motores eléctricos en los procesos industriales, sus componentes sufren deterioro y desgaste. Conforme este evoluciona el riesgo de sufrir una falla aumenta poniendo en peligro todo el proceso, linea de ensamblaje o sistema que necesite de su funcionamiento y desencadenando de esta forma una gran cantidad de perdidas para la empresa.\\
Para prevenir esto se deben tener bajo constante monitoreo los activos para la realización de un mantenimiento puntual que elimine dichos peligros. De esta forma el mantenimiento predictivo es la clave para mejorar la vida útil, funcionamiento y planificación de todo proceso especialmente a niveles industriales, donde la cantidad de motores eléctricos es bastante elevada, y por lo cual la automatización del proceso es crucial.\\

Sin embargo, dado los altos costos que implican realizar una automatización y en especial a escalas industriales se suele hacer una simulación o una emulación para estudiar y obtener el modelo mas preciso, gracias a la facilidad para manipular con mayor eficacia los diseños y conocer su comportamiento real sin necesidad de construirlo, antes de realizar la implementación y todo el proceso que esta conlleva.\\

Habiendo expuesto la importancia del mantenimiento como también la de realizar simulaciones, se justifica el hecho de realizar la implementación de un código el cual permita emular el comportamiento y las salidas de un acelerómetro, como también otorgue las herramientas necesarias para poder realizar un mantenimiento predictivo, y de esta forma se pueda estudiar a profundidad su estado actual como también su evolución histórica. Todo esto sumado a las facilidades de portabilidad que ofrece un sistema web, facilitando de esta forma la revisión constante sin las dificultades de los protocolos de acceso y sanidad.