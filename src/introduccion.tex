\thispagestyle{empty}
\addcontentsline{toc}{section}{INTRODUCCIÓN}
\section*{INTRODUCCIÓN}

No es una exageración decir que la mayoría de los procesos industriales
requieren de por lo menos un motor eléctrico, incluso algunos están compuestos, en su
mayoría, de mecanismos controlados exclusivamente por ellos. Debido a esto,
una gran parte de los recursos de mantenimiento son destinados a garantizar
su correcto funcionamiento.

Las fallas más comunes en los motores eléctricos se producen en sus rodamientos
y, a diferencia de otro tipo de fallas, estas no se benefician
del mantenimiento preventivo por la inexactitud al predecir cuándo
fallará un rodamiento, dada la gran cantidad de factores que afectan su vida útil;
asimismo, son muy comunes los errores humanos durante su reemplazo e instalación,
por lo cual termina siendo contraproducente reemplazar un rodamiento,
generando gastos operativos y posiblemente daños al motor,
sin conocer su estado

Por estas razones, a nivel industrial, es altamente demandado el mantenimiento
predictivo; este permite una mayor precisión al momento de determinar una avería
ya que permite conocer, mediante mediciones y estudios, el estado del motor.
De igual forma que el mantenimiento preventivo, el mantenimiento predictivo puede
ser realizado de forma manual, mediante el chequeo y medición de las variables de
interés inherentes al motor; sin embargo, en una empresa suele haber una gran cantidad
de motores eléctricos y es, prácticamente, imposible monitorizarlos todos sin un sistema
automatizado. Para solucionar este inconveniente se suele utilizar
uno o varios dispositivos, dedicados por motor, los cuales se encarguen de tomar
la información para un estudio posterior.

Este trabajo plantea la implementación de una herramienta computacional para la
visualización y análisis de los datos de vibración en motores eléctricos y la
prueba de su funcionamiento mediante información obtenida de una
simulación digital.

El trabajo está estructurado de la siguiente forma primero se tiene el Capítulo
1, también denominado Planteamiento del problema el cual tiene como objetivo
describir a detalle el problema que se intenta resolver en el trabajo, los
objetivos de la investigación, su justificación e importancia y las
limitaciones y alcances del mismo.

Luego el Capítulo 2 o Revisión Bibliográfica la cual comienza con un
análisis de varios trabajos previos que sustentan la elaboración de este
trabajo y después se tiene un marco teórico en el cual se explican muchos
conceptos claves necesarios para entender la elaboración del mismo.

El Capítulo 3 Marco Metodológico expone las diferentes etapas de la metodología
utilizada para la elaboración de este trabajo, se explican la naturaleza de la
investigación, el tipo de investigación, las diferentes fases de la
investigación  y culmina con una explicación de las diferentes metodologías
utilizadas.
\thispagestyle{myheadings}
\markright{INTRODUCCIÓN}
Finalmente, en el Capítulo 4 Análisis y discusión de los resultados se explican los
resultados obtenidos y el proceso por el cual se obtuvieron, así
como tablas e imágenes que exponen el funcionamiento de la aplicación y todos
sus componentes.


