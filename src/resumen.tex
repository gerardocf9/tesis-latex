\thispagestyle{empty}
\addcontentsline{toc}{section}{RESUMEN}

    \begin{table}[t]
        \centering
        \begin{tabular}{ p{3cm} p{8.5cm} p{3cm} }
            \begin{flushleft}\includegraphics[width=2.4cm]{logo_poli.png}\end{flushleft} &

                \begin{center}
                    República Bolivariana de Venezuela\\
                    Universidad Nacional Experimental Politécnica “Antonio José de Sucre”\\
                    Vice Rectorado Barquisimeto \\
                    Departamento de Ingeniería Electrónica\\

                \end{center}

                & \begin{flushright}\includegraphics[width=1.7cm]{logo_electronica.jpg} \end{flushright}
        \end{tabular}

        \vspace*{-3mm}

\parbox[c]{15cm}{
    \begin{center}
        \textbf{Herramienta computacional para el análisis de la vibración en
        motores eléctricos alimentada mediante datos de una simulación
        digital\\}


        \vspace*{3mm}
        \textbf{Autores:} Gerardo Campos, José Cortez\\

        \textbf{Tutor:} Dra. Luisa Escalona,
        \textbf{Cotutor:} Dr. Carlos Zambrano\\
    \end{center}
}
    \end{table}


La presente investigación ha tenido como propósito el desarrollo de una
herramienta computacional para el análisis de la vibración en motores
eléctricos, haciendo uso de  datos obtenidos mediante una simulación
digital para verificar su funcionamiento y una metodología de desarrollo
Web enfocada a microservicios. Para lograrlo se desarrolló un modelo
estadístico, capaz de entregar las mediciones necesarias para la
implementación, a partir de una base de datos existente; así mismo se
crearon 3 microservicio para cubrir las necesidades de obtención de
información y comunicación con el modelo, almacenar la información
(database as a service DBaaS) y mostrar la información por medio de una
página Web; se desarrollaron estructuras de datos y se diseñó una base de
datos no relacional para el almacenamiento de la información, se automatizó
la interconexión entre el modelo estadístico, el servidor de adquisición de
información y el DBaaS mediante un cliente y un Script y se despliega todo
el servicio en una página Web con 3 vistas, General, específica,
exhaustiva, en los cuales se puede obtener una vista global del estado de
los motores en la base de datos, un recuento de la evolución histórica del
motor (por medio de gráficas y una tabla)  y una gráfica de un análisis en
frecuencia correspondiente al espectro de vibración que posee el motor;
además la página consta de un  manual de usuario para facilitar su
utilización. De esta forma de tiene un sistema completamente modular que
puede escalar con el tiempo y puede ser alimentado por cualquier sistema de
adquisición de datos que respete los formatos y estructuras definidas.

Palabras claves: servidor, microservicio, modelo estadístico, vibración,
motores eléctricos.

