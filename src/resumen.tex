\thispagestyle{empty}
\addcontentsline{toc}{section}{RESUMEN}

    \begin{table}[t]
        \centering
        \begin{tabular}{ p{3cm} p{8.5cm} p{3cm} }
            \begin{flushleft}\includegraphics[width=2.4cm]{logo_poli.png}\end{flushleft} &

                \begin{center}
                    República Bolivariana de Venezuela\\
                    Universidad Nacional Experimental Politécnica “Antonio José de Sucre”\\
                    Vice Rectorado Barquisimeto \\
                    Departamento de Ingeniería Electrónica\\

                \end{center}

                & \begin{flushright}\includegraphics[width=1.7cm]{logo_electronica.jpg} \end{flushright}
        \end{tabular}


        \vspace*{0.6cm}

\parbox[c]{15cm}{
    \begin{center}
        \textbf{\large Herramienta computacional para el análisis de la vibración en
        motores eléctricos alimentada mediante datos de una simulación
        digital\\}


        \vspace*{1cm}
        \textbf{Autores:}\\
        Gerardo Alfonzo Campos Fonseca\\
        José Andrés Cortez Terán\\

        \textbf{Tutor:} Dra. Luisa Mercedes Escalona,
        \textbf{Cotutor:} Dr. Carlos Zambrano\\
    \end{center}
}
    \end{table}

        \section*{RESUMEN}

La presente investigación ha tenido como propósito el desarrollo de una
herramienta computacional para el análisis de la vibración en motores
eléctricos, haciendo uso de  datos obtenidos mediante una simulación
digital y una metodología de desarrollo
Web, enfocada a microservicios. Para lograrlo se desarrolló un modelo
estadístico, capaz de entregar las mediciones necesarias, a partir de una base
de datos real. Se
crearon 3 microservicios para la obtención, almacenamiento
(database as a service, DBaaS) y muestreo  de la información;
se desarrollaron estructuras de datos y una base de
datos no relacional para el almacenamiento de la información, se automatizó
la interconexión entre el modelo estadístico, el servidor de adquisición de
información y el DBaaS, mediante un cliente, y se despliega todo
el servicio en una página Web con 3 vistas, general, específica y
exhaustiva, que permiten observar una vista global del estado de
los motores, un recuento de la evolución histórica del
motor   y una gráfica  del espectro de
vibración que este posee;
además, la página consta de un  manual de usuario.
Se tiene un sistema  modular
 escalable en el tiempo que puede ser alimentado por cualquier sistema de
adquisición de datos que respete los formatos y estructuras definidas.

Palabras claves: servidor, microservicio, modelo estadístico, vibración,
motores eléctricos.

