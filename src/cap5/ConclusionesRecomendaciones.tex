\thispagestyle{empty}

\section{CONCLUSIONES Y RECOMENDACIONES}
\subsection{Conclusiones}
Tras la exposición de los resultados plasmados en el Capítulo IV, se pueden
plantear las siguientes conclusiones:

\begin{itemize}
    \item La investigación realizada ha sido de tipo aplicada y su naturaleza
        la clasifica como proyecto especial. Ésta ha consistido en la creación
        de una herramienta computacional para el análisis de la vibración en motores
        eléctricos mediante la implementación de dos servidores, uno de estilo
        microservicio, y un cliente Web, la cual puede ser alimentada mediante
        simulaciones, como es el caso del modelo estadísticos y el sistema de
        sensorica, o mediante mediciones a tiempo real de cualquier sistema
        que respete las estructuras de datos empleadas.
        %
    \item Se deciden implementar múltiples servicios, cada uno escrito con un
        lenguaje de programación que satisface los requerimientos de velocidad,
        paralelismo y robustez necesarios para satisfacer los requerimientos de
        uso, además de una fácil adecuación y modificación a las necesidades,
        de múltiples clientes.
        %
    \item Existe una gran cantidad de lenguajes de programación en la actualidad
        diseñados para satisfacer una gran cantidad de tareas en múltiples campos,
        siendo Go y Python herramientas de gran poder y fácil implementación
        al momento de realizar servidores y API, además de la creación de Scripts
        para la automatización de procesos; de igual forma Python es increíblemente
        polifacético y cuenta con una gran cantidad de librerías y Frameworks
        especialmente en el área de la ciencia de datos y estadística; sin embargo
        en términos de programación Web a nivel cliente, es casi la única
        posibilidad el utilizar HTML, CSS, JavaScript, presentando el ultimo la
        mayor variedad de Frameworks para agilizar el trabajo, siendo React uno
        de los mas grandes en la industria.
        %
    \item Se escoge implementar dos modelos estadísticos, uno para entregar el
        equivalente a mediciones diarias y otro para mediciones continuas a tiempo
        real necesarias para hacer un análisis en frecuencia; de esta forma se
        utilizaron distribuciones de probabilidad para representar 3 variables de interés,
        velocidad vertical, horizontal y aceleración, expresadas de forma independiente
        como magnitud y ángulo de la velocidad y aceleración con las dos primeras
        siendo la representación en coordenadas polares; las distribuciones que
        satisfacen de mejor manera estas variables fueron la distribución Burr
        de tipo III para la magnitud de velocidad y aceleración, y la distribución
        normal para el ángulo. De esta forma, se obtienen las mediciones diarias
        y con un modelo de señal no estadísticos ajustado a la amplitud dada por
        el modelo de la aceleración y una gaussiana para simular ruido, se obtienen
        las mediciones continuas.
        %
    \item Para facilitar la accesibilidad al modelo se implementó una API Web con
        el Frameworks FastAPI y dos endpoints que entregan la información especificada
        en el Url vía parámetros de configuración, nivel de daño (un numero del 0 al 10)
        y el identificador único del motor que tiene asociado.
        %
    \item Una vez obtenidas las salidas del modelo estadísticos, se procede con
        la implementación del análisis en frecuencia, esto se logra mediante una
        transformada rápida de Fourier (FFT) lo que permite llevar la señal
        discretizada (mediciones a tiempo real) al dominio de la frecuencia
        y posteriormente graficarla; todo esto se logra mediante las librerías
        \textbf{numpy} y \textbf{matplotlib} de Python.
        %
    \item  Se utilizo una base de datos no relacional, \textbf{MongoDB}, en forma
        de microservicio con la empresa \textbf{MongoDB Atlas} la cual permite
        la creación de un \textbf{cluster} gratuito para finalidades de practicas
        y estudios, capaz de almacenar una cierta cantidad de información e
        implementar diversas bases de datos y colecciones para las mismas, siempre
        que no se exceda este limite; esta base de datos llamada \textbf{tesis}
        contiene 2 colecciones, una se encarga de almacenar la información de las
        mediciones diarias y la otra de almacenar los identificadores únicos que
        representan a los motor de los que se posee información.
        %
    \item La automatización de procesos es fundamental para agilizar las tareas y
        la forma mas sencilla y óptima de hacerlo es mediante un Scripts; para
        llenar la base de datos se utilizo uno escrito en Go el cual acepta
        configuración al momento de la ejecución vía parámetros, siendo obligatorios
        la especificación de los identificadores únicos correspondientes al motor
        y los primeros 2 sensores.
        %
    \item Se crearon 2 microservicios como servidores, el primero llamado
        \textbf{servidor de sensorica}
        se encarga de la comunicación con los sensores, simulados mediante el
        \textbf{cliente de sensorica} y el \textbf{modelo estadísticos}, y la
        entrega de la información tomada a tiempo real necesaria para el análisis
        en frecuencia. Por otro lado, el otro microservicio, \textbf{servidor Web},
        se encarga de satisfacer las peticiones Web al enviar el HTML-CSS-JavaScript
        necesario para mostrar la pagina, además de los archivos estáticos e
        información requeridos por los mismos para mostrar el estado de los motores
        dependiendo del nivel de análisis solicitado.
        %
    \item Para facilitar la utilización del modelo estadísticos y la conexión con
        el microservicio de \textbf{sensorica} se implemento un \textbf{cliente de
        sensorica} escrito en Go el cual permite especificar las características
        inherentes al motor (Identificador del motor y de los sensores, características, etc)
        y al daño existente en el mismo (en una escala del 1 al 10), además permite
        especificar la dirección IP, o puerto local, al cual se conectara.
        %
    \item Se crearon 3 vistas para el cliente Web, estas son:
        \begin{enumerate}
            \item \textbf{General},
                tiene el menor grado de estudio ya que permite conocer el estado de todos
                los motores registrados en la BBDD, esto se consigue mediante paginación
                en una ventana, esta muestra 12 motores por vez y permite mover el índice
                para mostrar los siguientes 12, regresar e ir automáticamente al inicio,
                además existe una ventana de búsqueda que permite especificar el identificador
                único del motor del cual se quiere obtener la información.
                %
            \item \textbf{específica}, tiene un grado de estudio intermedio dado que
                permite conocer la evolución histórica del motor mediante gráficas
                y una tabla exportable a Excel; adicionalmente permite solicitar
                la vista exhaustiva.
                %
            \item \textbf{exhaustiva}, tiene el mayor grado de estudio ya que
                solicita mediciones a tiempo real de la aceleración del motor
                y las descompone en frecuencia, permitiendo observar una gráfica
                de las frecuencias y magnitudes de las mismas en las que vibra
                el motor; adicionalmente muestra toda la información de la
                vista específica.
        \end{enumerate}
        %
    \item Dado el tiempo de desarrollo y la complejidad del sistema no se
        pueden escribir \textbf{test} automatizados, sin embargo es absolutamente
        necesario el realizar pruebas a un software para garantizar su funcionamiento,
        por esta razón se probó manualmente el sistema en los puntos claves
        e interconexiones, es decir, se comprobó:
        \begin{enumerate}
            \item La unicidad de la información.
            \item Fallas de conexión con el microservicio de base de datos.
            \item Solicitud de información a tiempo real a un motor sin conexión
                establecida.
            \item Peticiones invalidas en el servidor Web
        \end{enumerate}
        %
    \item Una vez establecido el sistema, se procedió a comprobar que cumpliera
        los requerimientos mínimos solicitados por el cliente, es decir, se
        verificaron la clasificación en los distintos niveles de daño de acuerdo
        a las especificaciones del cliente, se corroboró la coherencia y legibilidad
        de las gráficas históricas y de la gráfica del análisis en frecuencia.
        %
\end{itemize}



\subsection{Recomendaciones}
\begin{itemize}
    \item Debido al costo que implica no se realizaron placas de adquisición de
        datos y tampoco se obtuvo data en tiempo real de motores eléctricos
        para la elaboración de este trabajo, pero el mismo puede ser
        complementado implementando dichos dispositivos.

    \item Para su uso empresarial la aplicación requiere de un sistema de
        autenticación y de seguridad para que solo los usuarios con los
        privilegios correctos puedan acceder a la información, esto
        probablemente requiera el uso de otra base de datos preferiblemente una
        base de datos relacional.

    \item Un sistema de notificaciones configurable por el usuario, para
        observar si algún motor recibe una medida fuera de los valores normales
        puede ser una gran ayuda al operador de la aplicación en el diagnóstico
        de fallas de forma temprana.

    \item Debido a la complejidad de las conexiones HTTP2 una alternativa es el
        uso de Websockets los cuales permiten una conexión bidireccional sin la
        complejidad del nuevo protocolo pero posee su propia serie de
        desventajas las cuales tienen que ser evaluadas con los requerimientos
        de la aplicación. Otra alternativa aún no disponible pero en un futuro
        cercano posiblemente lo este, es HTTP3 la cual está siendo discutida
        por el comité del estándar HTTP para corregir las fallas de la última
        versión.

    \item Las pruebas fueron realizadas de forma manual pero una decisión que
        puede minimizar el tiempo de depuración y evitar las fallas en
        producción es el uso de pruebas automatizadas pero eso implica un coste
        en el tiempo de desarrollo dedicado al diseño de las pruebas por lo
        tanto la elección de usarlas o no dependerá de la experiencia previa
        del programador con este tipo de metodología de desarrollo.

    \item El ecosistema de librerías y Frameworks frontend es muy cambiante por
        lo tanto la elección de las herramientas debe ser realizada de acuerdo
        a las tecnologías más populares al momento del desarrollo.

    \item  Si bien no se llegó a utilizar Julia sigue siendo un lenguaje con
        mucho potencial tratando de incorporar ventajas de lenguajes
        interpretados como Python y lenguajes compilados como Fortran. Su
        principal desventaja es lo joven que es pero a futuro puede llegar ser
        uno de los lenguajes más dominantes en el campo científico.

\end{itemize}
