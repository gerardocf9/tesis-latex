\thispagestyle{empty}

\section{CONCLUSIONES Y RECOMENDACIONES}
Tras la exposición de los resultados plasmados en el Capítulo IV, se pueden
plantear las siguientes conclusiones:

\begin{itemize}
    \item La investigación realizada ha sido de tipo aplicada y su naturaleza
        la clasifica como proyecto especial. Ésta ha consisitido en la creación
        de una herramienta computacional para el análisis de la vibracion en motores
        eléctricos mediante la implementación de dos servidores, uno de estilo
        microservicio, y un cliente Web, la cual puede ser alimentada mediante
        simulaciones, como es el caso del modelo estadistico y el sistema de
        sensorica, o mediante mediciones a tiempo real de cualquier sistema
        que respete las estructuras de datos empleadas.
        %
    \item Se deciden implementar multiples servicios, cada uno escrito con un
        lenguaje de programacion que satisface los requerimientos de velocidad,
        paralelismo y robustez necesarios para satisfacer los requerimientos de
        uso, ademas de una facil adecuacion y modificacion a las necesidades,
        de multiples clientes.
        %
    \item Existe una gran cantidad de lenguajes de programacion en la actualidad
        diseñados para satisfacer una gran cantidad de tareas en multiples campos,
        siendo Golang y Python herramientas de gran poder y facil implementacion
        al momento de realizar servidores y API, ademas de la creacion de scripts
        para la automatizacion de procesos; de igual forma Python es increiblemente
        polifacetico y cuenta con una gran cantidad de librerias y Frameworks
        especialmente en el area de la ciencia de datos y estadistica; sin embargo
        en terminos de programacion Web a nivel cliente, es casi la unica
        posibilidad el utilizar HTML, CSS, JavaScript, presentando el ultimo la
        mayor variedad de Frameworks para agilizar el trabajo, siendo React uno
        de los mas grandes en la industria.
        %
    \item Se escoge implementar dos modelos estadisticos, uno para entregar el
        equivalente a mediciones diarias y otro para mediciones continuas a tiempo
        real necesarias para hacer un analisis en frecuencia; de esta forma se
        utilizaron distribuciones de probabilidad para representar 3 variables de interes,
        velocidad vertical, horizontal y aceleracion, expresadas de forma independiente
        como magnitud y angulo de la velocidad y aceleracion con las dos primeras
        siendo la representacion en coordenadas polares; las distribuciones que
        satisfacen de mejor manera estas variables fueron la distribucion Burr
        de tipo III para la magnitud de velocidad y aceleración, y la distribucion
        normal para el angulo. De esta forma, se obtienen las mediciones diarias
        y con un modelo de señal no estadistico ajustado a la amplitud dada por
        el modelo de la aceleración y una gaussiana para simular ruido, se obtienen
        las mediciones continuas.
        %
    \item Para facilitar la accesibilidad al modelo se implementó una API Web con
        el framework FastAPI y dos endpoints que entregan la información especificada
        en el Url via parametros de configuracion, nivel de daño (un numero del 0 al 10)
        y el identificador unico del motor que tiene asociado.
        %
    \item La automatización de procesos es fundamental para agilizar las tareas y
        la forma mas sencilla y optima de hacerlo es mediante un script; para
        llenar la base de datos se utilizo uno escrito en Golang el cual acepta
        configuracion al momento de la ejecucion via parametros, siendo obligatorios
        la especificación de los identificadores unicos correspondientes al motor
        y los primeros 2 sensores.
        %

\end{itemize}
