\thispagestyle{empty}

\addcontentsline{toc}{section}{CONCLUSIONES Y RECOMENDACIONES}
\section*{CONCLUSIONES Y RECOMENDACIONES}
\subsection{Conclusiones}
Tras la exposición de los resultados plasmados en el Capítulo IV, se pueden
plantear las siguientes conclusiones:

\begin{itemize}
    \item El desarrollo  de una herramienta computacional para el análisis de
        la vibración en motores eléctricos se hizo posible mediante la
        implementación de dos servidores, uno de estilo
        microservicio y un cliente Web, puede ser alimentada mediante la
        simulación de un  modelo estadístico o
        mediante mediciones a tiempo real de cualquier sistema
        que respete las estructuras de datos empleado.
        %
    \item Se implementaron múltiples servicios, cada uno escrito con un
        lenguaje de programación que satisface los requerimientos de velocidad,
        paralelismo y robustez necesarios para  los requerimientos de
        uso, además de una fácil adecuación y modificación a las necesidades,
        de múltiples clientes.
        %
    \item Existe una gran cantidad de lenguajes de programación en la actualidad
        diseñados para satisfacer una gran cantidad de tareas en múltiples campos,
        siendo Go y Python herramientas de gran poder y fácil implementación
        al realizar servidores y API, además de la creación de Script
        para la automatización de procesos; de igual forma Python es increíblemente
        polifacético y cuenta con una gran cantidad de librerías y Frameworks
        especialmente en el área de la ciencia de datos y estadística.
    %
    \item En términos de programación Web a nivel cliente,
        HTML, CSS, JavaScript, son casi una obligación, sin embargo,
        JavaScript cuenta con una  variedad de Frameworks para agilizar el
        trabajo, siendo React uno
        de los mas utilizados en la industria.
        %
    \item Se implementaron dos modelos estadísticos, uno para entregar el
        equivalente a mediciones diarias y otro para mediciones continuas a tiempo
        real necesarias para hacer un análisis en frecuencia,
        utilizando distribuciones de probabilidad para representar 3 variables de interés,
        velocidad vertical, horizontal y aceleración, expresadas de forma independiente
        como magnitud y ángulo de la velocidad y aceleración las dos primeras
        siendo representadas en coordenadas polares.

    \item Las distribuciones que satisfacen de mejor manera las variables
        modeladas fueron la distribución Burr
        de tipo III, para la magnitud de velocidad y aceleración, y la distribución
        normal para el ángulo, de esta forma se obtienen las mediciones diarias;
        con un modelo de señal no estadístico ajustado a la amplitud, dada por
        el modelo de la aceleración, mas una gaussiana, para simular ruido, se obtienen
        las mediciones continuas.
        %
    \item Se implementó una API Web, para facilitar la accesibilidad al modelo, con
        el Frameworks FastAPI y dos endpoints que entregan la información especificada
        en el Url y en los parámetros de configuración, nivel de daño (un número del 0 al 10)
        y el identificador único del motor que tiene asociado.
        %
    \item El análisis en frecuencia se realiza mediante una
        transformada rápida de Fourier (FFT) lo que permite llevar la señal
        discretizada (mediciones a tiempo real) al dominio de la frecuencia
        y posteriormente graficarla; todo esto se logra mediante las librerías
        \textbf{numpy} y \textbf{matplotlib} de Python.
        %
    \item  Se utilizó una base de datos no relacional, \textbf{MongoDB}, en forma
        de microservicio con la empresa \textbf{MongoDB Atlas};
        esta base de datos llamada \textbf{tesis}
        contiene 2 colecciones, una se encarga de almacenar la información de las
        mediciones diarias y la otra de almacenar los identificadores únicos que
        representan a los motores de los que se posee información.
        %
    \item La automatización de procesos es fundamental para agilizar las tareas y
        la forma mas sencilla y eficiente de hacerlo fue mediante un Script; para
        llenar la base de datos se utilizó uno, escrito en Go, el cual acepta
        configuración al momento de la ejecución, vía parámetros, siendo obligatorios
        la especificación de los identificadores únicos correspondientes al motor
        y los primeros 2 sensores.
        %
    \item Se crearon 2 microservicios como servidores, el primero llamado
        \textbf{servidor de sensorica}
        se encarga de la comunicación con los sensores, el segundo es el \textbf{servidor Web},
        y se ocupa de satisfacer las peticiones Web al enviar el HTML-CSS-JavaScript
        necesario para mostrar la página, los archivos estáticos e
        información requeridos por los mismos para mostrar el estado de los motores
        dependiendo del nivel de análisis solicitado.
        %
    \item Facilitar la utilización del modelo estadístico y la conexión con
        el microservicio de \textbf{sensorica} se implementó con un \textbf{cliente de
        sensorica} escrito en Go el cual permitió especificar las características
        inherentes al motor (Identificador del motor y de los sensores, características, etc)
        y al daño existente en el mismo (en una escala del 1 al 10), además,
        especificar la dirección IP, o puerto local, al cual se conectará.
        %
    \item Se crearon 3 vistas para el cliente Web, estas son:
        \begin{enumerate}
            \item \textbf{General}, con el cual se conoce el estado de todos
                los motores registrados en la base de datos, mediante paginación
                en una ventana se muestran 12 motores por vez, y controlar
                cuáles son mostrados,
                y especificar el identificador
                único del motor del cual se quiere obtener mas información.
                %
            \item \textbf{Específica},
                muestra la evolución histórica del motor mediante gráficas
                y una tabla exportable a Excel; adicionalmente permite solicitar
                la vista exhaustiva.
                %
            \item \textbf{Exhaustiva},
                solicita mediciones a tiempo real de la aceleración del motor
                y las descompone en frecuencia, permitiendo observar una gráfica
                de las frecuencias y magnitudes  en las que vibra
                el motor; adicionalmente muestra toda la información de la
                vista específica.
        \end{enumerate}
        %
    \item Se probó manualmente el sistema en los puntos claves
        e interconexiones, con lo cual  se comprobó:
        \begin{enumerate}
            \item La unicidad de la información.
            \item Fallas de conexión con el microservicio de base de datos.
            \item Solicitud de información a tiempo real a un motor sin conexión
                establecida.
            \item Peticiones invalidas en el servidor Web
        \end{enumerate}
        %
    \item Se comprobó el cumplimiento de
        los requerimientos mínimos que satisfacen las condiciones planteadas,
        mediante la
        demostración de la clasificación en los distintos niveles de daño,
         se corroboró la coherencia y legibilidad
        de las gráficas históricas y de la gráfica del análisis en frecuencia.
        %
\end{itemize}



\subsection{Recomendaciones}
\begin{itemize}
    \item Utilizar un sistema de adquisición de datos, sensores o placas diseñadas
        para la obtención de las mediciones diarias, por ejemplo los datos a tiempo
        real para alimentar el sistema.

    \item  Implementar un sistema de autenticación y seguridad para que solo
        los usuarios con los
        privilegios correctos puedan acceder a la información, esto
        probablemente requiera el uso de otra base de datos preferiblemente
        relacional, y de esta forma poder hacer uso empresarial
        de la aplicación.

    \item Un sistema de notificaciones configurable por el usuario,
        puede ser una gran ayuda al operador de la aplicación en el diagnóstico
        de fallas de forma temprana, para observar si algún motor recibe una
        medida fuera de los valores normales.

    \item Debido a la complejidad de las conexiones HTTP2 una alternativa es el
        uso de Websockets los cuales permiten una conexión bidireccional sin la
        complejidad del nuevo protocolo evaluando  los requerimientos
        de la aplicación. Otra alternativa, en un futuro
        cercano, posiblemente lo sea el HTTP3 la cual está siendo discutida
        por el comité del estándar HTTP para corregir fallas de la última
        versión.

    \item Creación y uso de  pruebas automatizadas para minimizar el tiempo de
        depuración y evitar fallas en producción, cabe destacar que esto
        implica un coste
        en el tiempo de desarrollo dedicado al diseño de las pruebas.
        La elección de usarlas dependerá de la experiencia previa
        del programador con este tipo de metodología de desarrollo.


\end{itemize}
