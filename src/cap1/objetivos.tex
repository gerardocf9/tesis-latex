\subsection{OBJETIVOS DE LA INVESTIGACIÓN}

\subsubsection{Objetivo general}
	Desarrollar una herramienta computacional para el análisis de la vibración en motores eléctricos, mediante la simulación digital de un acelerómetro, con la finalidad de que un operador determine averías y sus causas.


\subsubsection{Objetivos específicos}

	\begin{enumerate}
		\item Justificar la escogencia de las herramientas y lenguajes a utilizar en las diferentes etapas que requiere la simulación. (José Cortez y Gerardo Campos)

		\item Generar un modelo estadístico de la vibración en motores eléctricos con distinto grado de daño utilizando una base de datos de la salida de un acelerómetro digital. (José Cortez)

		\item Elaborar una base de datos con información obtenida del modelo estadístico para alimentar los niveles de análisis de la herramienta. (José Cortez y Gerardo Campos)

		\item Realizar análisis de fallas en frecuencia, a partir de la salida del modelo del acelerómetro. (Gerardo Campos)

		\item Mostrar la información solicitada de acuerdo al nivel de análisis seleccionado. según sea: Vista Principal, Vista Específica o Vista Exhaustiva. (José Cortez y Gerardo Campos)

		\item Elaboración de una página Web para facilitar la utilización del sistema. (José Cortez y Gerardo Campos)

		\item Comprobar los resultados de la herramienta de análisis. (José Cortez y Gerardo Campos)
	\end{enumerate}
