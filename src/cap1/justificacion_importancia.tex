\subsection{JUSTIFICACIÓN E IMPORTANCIA}

Para prevenir las fallas mecánicas que ocurren en los motores, por el deterioro
y desgaste de los rodamientos, estos se deben tener bajo constante monitoreo
para poder efectuar un mantenimiento puntual que elimine dichos peligros. De
esta forma el mantenimiento predictivo es la clave para mejorar la vida útil,
funcionamiento y planificación de todo proceso especialmente a niveles
industriales, donde la cantidad de motores eléctricos es bastante elevada
por lo cual la automatización del proceso es crucial.

Sin embargo, dado los altos costos que implican realizar una automatización y
en especial a escalas industriales se suele hacer una simulación o una
emulación para estudiar y obtener el modelo más preciso, gracias a la facilidad
de manipular con mayor eficacia los diseños y conocer su comportamiento real
sin necesidad de construirlo, antes de realizar la implementación y todo el
proceso que esta conlleva.

Habiendo expuesto la importancia del mantenimiento como también la de realizar
simulaciones, se justifica el hecho de realizar el desarrollo del trabajo aquí
propuesto el cual permita emular el comportamiento y las salidas de un
acelerómetro, como también otorgar las herramientas necesarias para
realizar un mantenimiento predictivo, y de esta forma estudiar a
profundidad su estado actual como también su evolución histórica. Todo esto
sumado a las facilidades de portabilidad que ofrece un sistema Web, facilitando
la revisión constante sin las dificultades de los protocolos de acceso y
sanidad usuales en las plantas industriales.
