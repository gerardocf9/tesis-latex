
\subsection{LIMITACIONES Y ALCANCES}
    En función de los objetivos planteados con anterioridad, se puede definir
    tanto las limitaciones como el alcance del proyecto.

\subsubsection{Limitaciones}
\begin{itemize}
    \item Recursos económicos impiden la adquisición de dispositivos para
        pruebas en motores reales.

    \item Disponibilidad de muestras, aunque se cuenta con una base de datos lo
        suficiente grande para cubrir el comportamiento de la vibración en
        motores, incluso de distinta potencia, esta es discreta y con
        intervalos de tiempo considerables entre cada muestra.


    \item Las establecidas por el software de terceros utilizado en el diseño
        y desarrollo del proyecto.
\end{itemize}

\subsubsection{Alcance}
    El principal alcance de este trabajo es el desarrollo de una herramienta
    computacional para el análisis de la vibración en motores eléctricos; en
    función de esto:

	\begin{itemize}
        \item La herramienta computacional permite una vista general del estado
            de todos los motores en un espacio previamente delimitado y
            seleccionado (planta o piso) que se encuentren en la base de datos.

        \item La herramienta computacional permite una vista detallada del
            estado de un motor seleccionado, previamente por el usuario, que se
            encuentre en la base de datos.

        \item El sistema cuenta con una base de datos a partir de la cual se
            desarrollarán los resultados ofrecidos.

        \item no se contempla la construcción de hardware de ningún tipo.


	\end{itemize}
