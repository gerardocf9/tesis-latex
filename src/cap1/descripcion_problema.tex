\thispagestyle{empty}
\pagenumbering{arabic}
\setcounter{page}{5}

\section{PLANTEAMIENTO DEL PROBLEMA}

\subsection{DESCRIPCIÓN DEL PROBLEMA}

En la actualidad se vive un crecimiento exponencial a nivel industrial en todos
los rublos, desde manufactura hasta desarrollo de ciencias aplicadas,  especialmente
existe una muy alta demanda de alimentos, la evolucion y la posibilidad de
satisfacer tal demanda es debido, entre
otras cosas, al motor eléctrico, este es un artefacto que transforma la
energía eléctrica en energía mecánica (movimiento), de manera que puede
impulsar el funcionamiento de una máquina y son utilizados ampliamente en:
Bombas para Agua, Bombas Industriales, Mezcladoras, Molinos, Correas
Transportadoras, Zarandas, Cortadoras, Ventiladores, Grúas, y en todo proceso
que involucre movimiento.

Adicional a la versatilidad de uso de los motores, existen factores que deben ser
considerados al evaluar su rendimiento, como las grandes pérdidas
(horas, insumos, dinero, etc.) que ocasiona una parada
de emergencia en una planta, por lo cual se considera de vital importancia que
los motores se encuentren completamente operativos y funcionales. Para
procurar su buen estado y funcionamiento, se deben realizar mantenimientos.

Existen diferentes tipos de mantenimiento, entre ellos están:
\begin{itemize}

\item{Correctivo: se espera que ocurra una falla para reparar o cambiar un
    equipo. Esto puede degradar la vida útil del equipo y debido a que la falla
        puede ocurrir en cualquier momento, lo que usualmente produce un paro en la
        línea de producción; por lo tanto, este tipo de mantenimiento suele y
        debe ser evitado.}

\item{Preventivo: para evitar una falla mayor se detiene la maquinaria para
    hacer un mantenimiento preparado con anticipación, se inspecciona la
        maquinaria y se reemplazan las piezas propensas a dañarse. Este tipo de
        mantenimiento en algunas circunstancias es más que suficiente pero en
        el caso de los rodamientos puede ser contraproducente.}

\item{Predictivo: se pronostica cuando una falla está a punto de ocurrir; a
    través de mediciones, y del estudio de las mismas, se prevé cuando un
        desperfecto está a punto de ocurrir y de esta forma se realiza una
        mejor planificación. Cabe resaltar que este tipo de mantenimiento puede
        dar información acerca del origen y la gravedad de las averías.}

\end{itemize}

En el caso de los motores eléctricos, como dice ~\textcite{Lacey}, el
mantenimiento preventivo tiene muchas desventajas dado que existen problemas de
índole mecánico así como administrativos y monetarios, un ejemplo de esto
pueden ser los altos costos de reemplazo dado que las partes se reemplazan muy
pronto, el riesgo de pérdida completa o parcial dado un error humano, como la posibilidad
de generar daño o una incorrecta instalación de la misma, la instalación de una
pieza defectuosa, y por último, el hecho de que las piezas reemplazadas
pueden tener muchos años de vida útil.

De esta forma se ve que una observación constante ofrece más control sobre las
variables mencionadas y, aunque no evita las posibilidades de error humano, si
permite reaccionar a este; adicionalmente, se deben considerar las altas
pérdidas y retrasos, sumadas a las dificultades administrativas, que generan
las paradas periódicas de la planta. Todas estas consideraciones son de suma
importancia ya que para poder ser ejecutadas se necesitan herramientas capaces de
dar a conocer el estado actual de una máquina y permitir a los operadores,
encargados o ingenieros tomar acciones cuando sean necesarias y de acuerdo a
las políticas de la empresa.

Por las razones expuestas surge la necesidad de reconocer las fallas y de tener
bajo continuo monitoreo los factores que las causan para, de esta forma,
atenuarlos. Según los estudios realizados por ~\textcite{Kammermann} la media
de la probabilidad de fallas en máquinas de inducción permanece al nivel de los
años 70 ($10^{-6}/hour$) y está altamente relacionada a la falla de los
rodamientos, y que un 59\% de las fallas son causadas por los rodamientos, esto
es debido a que son piezas sometidas a mucho estrés mecánico, permiten soporte
y asimismo necesitan tener poca fricción. Por esta razón, su principal falla es
el desgaste, de igual forma se presentan otras fallas estructurales que generan
vibraciones. Cabe resaltar que todas las fallas mecánicas generan vibración sin
importar su relación con los rodamientos y lo hacen a distinta frecuencia
($f$).

Por lo expuesto, es de suma importancia el estudio de la vibración, para la que
se suele implementar un acelerómetro y con un estudio de la frecuencia se puede
obtener la causa y la magnitud de la falla y de esta forma programar su
reparación. Sin embargo, debido a que esta evaluación debe ser realizada en
cada rodamiento y acople o extensión del motor  y dadas las mediciones que se
deben realizar por unidad (motor y acoples) y por la gran cantidad de unidades
existentes a nivel industrial, es virtualmente imposible el estudio con un
acelerómetro convencional (a pesar de que la medición no sea un proceso muy
largo y los cálculos y evaluaciones sean realizados posteriormente). Sumado a
esto, las industrias suelen poseer medidas y controles sanitarios, aumentados
por la pandemia actual, que no permiten el constante monitoreo de la planta
significando esto la imposibilidad de implementar este tipo de acciones de
forma manual, por lo cual se debe recurrir a un sistema de automatización capaz
de medir las vibraciones en todos estos equipos que, a su vez, permita el
estudio de estos datos de forma remota.
