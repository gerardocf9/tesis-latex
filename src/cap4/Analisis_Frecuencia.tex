
\subsection{IMPLEMENTACIÓN DEL ANÁLISIS EN FRECUENCIA}

Una de las herramientas mas importantes para tomar una decision en cuanto
al daño del motor es el analisis en frecuencia de la vibracion, esto es porque
permite obtener un mayor conocimiento del estado de los rodamientos y de cual
puede ser la falla, en el caso de que exista alguna, que estos presentan. Para
porder observar esta informacion, se procedio a realizar una transformacion de
dominio, mediante una FFT, para llevar las mediciones al dominio de la frecuencia
y mediante una grafica permitir el analisis.

Cabe destacar que este analisis y la generacion de la grafica son parte de la
api del analisis exhaustiva, en el servidor Web y el resultado es visualizado
mediante el cliente Web, en el endpoint correspondiente a la vista exhaustiva.

Para comenzar este proceso se debe de solicitar la informacion al servidor de
sensorica, esto se hace mediante una \textbf{peticion Web GET} al endpoint
correspondiente (DireccionIP/exhaustive\textbf{?idMotor=identificador}), posteriormente
se examina el \textbf{header} de la respuesta para determinar si fue exitosa
la solicitud, si esta no lo fue, no se puede generar este estudio por lo se envia una
imagen de error asociada al cliente; en caso de que si se pueda proceder, se
obtiene la informacion, proveniente en formato JSON, y mediante la libreria
numpy se obtiene un array con las mediciones, se procede a utilizar la transformada
\textbf{fft} de la misma libreria, se convierte a una transformada unilateral,
por estandard en la industria, y se grafica mediande la libreria \textbf{matplotlib}
y se guarda en disco, para despues ser enviada como un archivo estatico por otro
puerto del servidor Web.
