
\subsection{IMPLEMENTACIÓN DEL ANÁLISIS EN FRECUENCIA}

Una de las herramientas mas importantes para tomar una decisión en cuanto
al daño del motor es el análisis en frecuencia de la vibración, esto es porque
permite obtener un mayor conocimiento del estado de los rodamientos y de cuál
puede ser la falla, en el caso de que exista alguna, que estos presentan. Para
poder observar esta información, se procedió a realizar una transformación de
dominio, mediante una \textbf{Transformada Rápida de Fourier} (FFT), para
llevar las mediciones al dominio de la frecuencia
y mediante una gráfica permitir el análisis.

Cabe destacar que este análisis y la generación de la gráfica son parte de la
API del análisis exhaustivo, en el servidor Web y el resultado es visualizado
mediante el cliente Web, en el endpoint correspondiente a la vista exhaustiva.

Para comenzar este proceso se debe  solicitar la información al servidor de
sensorica, esto se hace mediante una \textbf{petición Web GET} al endpoint
correspondiente (DireccionIP/exhaustive\textbf{?idMotor=identificador}) posteriormente,
se examina el \textbf{header} de la respuesta para determinar si fue exitosa
la solicitud, si esta no lo fue, no se puede generar este estudio por lo que se envía una
imagen de error asociada al cliente; en caso de que si se pueda proceder, se
obtiene la información, proveniente en formato JSON, y mediante la librería
numpy se obtiene un array con las mediciones, se procede a utilizar la transformada
\textbf{FFT} de la misma librería, se convierte a una transformada unilateral,
por estándar en la industria,  se grafica mediante la librería \textbf{matplotlib}
y se guarda en disco, para después ser enviada como un archivo estático por otro
puerto del servidor Web.
