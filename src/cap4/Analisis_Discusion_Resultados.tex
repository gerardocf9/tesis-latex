\thispagestyle{empty}

\section{ANÁLISIS Y DISCUSIÓN DE RESULTADOS}

\subsection{ELECCIÓN DE LAS HERRAMIENTAS Y LENGUAJES}
    La elección de las herr..


    % MODELO ESTADISTICO DE LA VIBRACION
    % SERVIDORES, CONEXION CON SENSORES, BASES DE DATOS, ANALISIS EN FRECUENCIA, WEB
    % UI, CLIENTE WEB
\subsection{IMPLEMENTACIÓN DEL MODELO ESTADISTICO}


\subsection{IMPLEMENTACIÓN DE LA BASE DE DATOS}

    Como se ha mencionado anteriormente, se ha escogido una base de datos no
    relacional, especificamente MongoDB, dentro de esta se han desarrollado
    una base de datos llamada ``tesis"\   y dos colecciones llamadas ``MotorData"\  y
    ``MotoresInDB"\ ; la primera coleccion se encarga de almacenar toda la informacion
    enviada por los sensores (en este caso el cliente simulado con los datos
    proporcionados por el modelo estadistico) y la segunda coleccion contiene una
    lista con los identificadores unicos de cada motor que tenga al menos un documento
    en la coleccion ``MotorData"\ , es decir, hay informacion registrada de su actividad.

    Cada coleccion tiene una estructura fija definida para facilitar el manejo y
    la consistencia de los documentos, esta es similar un json y sus campos
    estan explicados en las tablas \ref{tab:MotorDatabson} para ``MotorData"\  y
    \ref{tab:MotorInDBbson} para ``MotorInDB"\ .

    \begin{table}[h!]
        \begin{center}
            Tabla de la estructura seguida para la colección ``MotorData"\ .\\

            \vspace{0.3cm}
            \begin{tabular}{|c|c|p{9cm}|}
                \hline
                Elemento        & tipo de dato & Descripcion \\\hline\hline
                %
                $\_$id      & []bytes  & Elemento utilizado por MongoDB para
                identificar y facilitar la busqueda de los documentos\\\hline
                %
                IdMotor         & uint64   & Identificador unico del motor.\\\hline
                %
                Caracteristicas & string   & Descripcion e informacion del motor.\\\hline
                %
                IdSensor        & []uint64 & lista de los identificadores-sensores
                que tiene conectado este motor.\\\hline
                %
                Data            & []DataSensor & lista en forma de subcoleccion
                que contiene los resultados del sensor.\\\hline
                %
                Time            & time  & Estampa de tiempo, fecha y hora de la muestra.\\\hline
                \hline
                \multicolumn{3}{|c|}{Sub coleccion  ``DataSensor"\ }\\\hline\hline
                %
                IdSensorData & uint64 & Identificador unico del sensor que tomo la muestra.\\\hline
                AcelerationX & float64 & Muestra de aceleracion en el eje X.\\\hline
                AcelerationY & float64 & Muestra de aceleracion en el eje Y.\\\hline
                AcelerationZ & float64 & Muestra de aceleracion en el eje Z.\\
                \hline
            \end{tabular}
        \end{center}
        \caption[Estructura de MotorData]{Estructura de la coleccion MotorData}
        \label{tab:MotorDatabson}
    \end{table}

\vspace{1cm}


    \begin{table}[h!]
        \begin{center}
            Tabla de la estructura seguida para la coleccion ``MotorInDB"\ \\
            \vspace{0.3cm}
            \begin{tabular}{|c|c|p{11cm}|}
                \hline
                Elemento & tipo     & Descripcion \\\hline\hline
                %
                \_id      & []bytes  & Elemento utilizado por MongoDB para
                identificar y facilitar la busqueda de los documentos\\\hline
                %
                IdMotor  & []uint64 & Lista que contiene todos los identificadores
                unicos de los motores. Facilita busqueda e implementacion de los
                clientes Webs\\\hline
            \end{tabular}
        \end{center}
        \caption[Estructura de MotorInDB]{Estructura de la coleccion MotorInDB}
        \label{tab:MotorInDBbson}
    \end{table}

    Cabe realtar que ``uint64"\  hace referencia a un numero natural de 64 bits, y
    ``float64"\  a un numero racional representado como punto flotante de 64 bits,
    Asimismo se usan estos tipos ya que en el caso de los datos es una unidad
    comun, para maximizar la precision y en el caso de los identificadores de
    sensores, permite el uso de 64 bits (16 bytes) los cuales son codificados de
    la forma expuesta en la tabla \ref{tab:CodIdSensor}, para
    poder transmitir mas informacion y facilitar la escalabilidad del sistema en
    un futuro.

    \begin{table}[h!]
        \begin{center}
            Tabla de la estructura seguida para ``IdSensor"\  \\

            \vspace{0.3cm}
            \begin{tabular}{|c|c|c|c|}
                \hline
                Tipo de sensor & Ubicación & Reservado  & Serial \\\hline
                $ B_{15} $ & $ B_{14} $  & $ B_{13}B_{12} $ & $ B_{11}\cdots B_{0} $\\
                \hline
            \end{tabular}

            \vspace{0.3cm}
            \begin{tabular}{|c|p{13cm}|}
                \hline
                Campo       & Descripción
                \\\hline\hline
                tipo        & tipo de sensor usado, Acelerometro, Temperatura, etc.
                Con $0000b$ siendo Acelerometro y 15 posibilidades adicionales.
                \\\hline
                Ubicación   & Posicion con respecto al motor y acoples. Con:
                $0000b$ Lado con carga, $ 0001b$ Lado libre, $ 0010b\cdots1111b $
                disponibles para acoples y chumaceras.
                \\\hline
                Reservado   & No se utilizan, son 0x00 siempre y se reservan para
                posibles expansiones y/o necesidades.
                \\\hline
                Serial      & Numero de fabricacion del sensor, desde 0 hasta $2^48$.
                \\\hline
            \end{tabular}
        \end{center}
        \caption[Estructura IdSensor]{Estructura seguida en el identificador de
        los sensores para permitir y facilitar la escalabilidad del sistema}
        \label{tab:CodIdSensor}
    \end{table}


 \newpage
\subsection{IMPLEMENTACION DE LOS SERVIDORES}

\subsection{IMPLEMENTACION DEL CLIENTE WEB}

\subsection{COMPROBACIÓN DE LOS RESULTADOS}

