
\subsection{IMPLEMENTACIÓN DE LOS CLIENTES}

Un cliente puede ser definido como un equipo o software que se conecta a un
servidor para obtener un beneficio, sea por poder de computo, para obtener
una informacion dada o comunicarse con un programa que se ejecuta en el lado
del servidor. Siguiendo esta definicion, Se crearon 2 clientes, uno para
facilitar la insersion de datos en la simulacion del sensor y el cliente web
el cual permite manejar el analisis y la monitorizacion de los sensores.

\subsubsection{Cliente de Sensorica}

Este fue elaborado en Go para facilitar la interconexion con el servidor y
de igual forma aprovechar el paralelismo y la multiplataforma que el
lenguaje ofrece. Se puede subdividir en 3 acciones fundamentales:

\begin{itemize}
    \item GUI: Es la interfaz grafica de usuario, utiliza el framework fyne por
        facilidades
        de diseño y es una ventana que permite ingresar datos equivalentes
        a las caracteristicas del motor y conectar con el servidor (ademas de
        especificar en que direccion esta el motor), y posee una ventana de
        log en la cual se comunica al usuario acciones como la conexion exitosa y
        el envio de informacion al servidor. Permite especificar:
        \begin{enumerate}
            \item Direccio Ip: Lugar a conectar.
            \item Id Motor: Identificador unico del motor.
            \item Potencia: Informacion adicional, opcional.
            \item Nivel de daño: numero entre 1 y 10 que determina los posibles
                valores del modelo.
            \item Informacion: Informacion adicional, pensado para comentarios,
                opcional.
            \item Sensores: lista de 1 a 5 posibles sensores, que representan
                los identificadores unicos que tienen los sensores asociados
                al motor.
        \end{enumerate}

        Cabe resaltar que a ser una GUI es el proceso principal, las demas
        tareas son realizadas de forma paralela.

    \item Conexion al servidor: Es un protocolo que ocurre cada vez que se presiona
        el boton de conectar, se encarga de intercambiar informacion con el servidor
        para poder conectarse (bajo el protocolo https) y envia los datos de
        que motor se va a conectar (simular) y que sensores tiene asociados, espera
        una aprobacion de conexion (no exista un motor con el mismo id enviando
        informacion), obtiene las caracteristicas especificadas en la GUI y
        comienza un proceso de envio-recibo de informacion.
        Este es un proceso
        que se ejecuta paralelamente a la GUI y se inicia cada vez que se
        presiona el boton de conectar, Si habia un proceso previo y se vuelve a
        presionar conectar finaliza el anterior y comienza uno nuevo.

    \item Comunicacion con el servidor: Es un proceso de envio bidireccional
        de informacion, esta rutina se inicia cuando la conexion al servidor
        es completada exitosamente y se subdivide en 3 procesos que a su
        vez corren paralelamente (se toma el proceso de conexion al servidor
        y se crean 2 hijos para un total de 4 procesos paralelizados si se
        incluye la GUI). Estos procesos son:

        \begin{enumerate}
            \item timer: Es un proceso que se encarga de cronometrar cada cuanto
                se va a mandar una mensaje de informacion con los datos
                correspondientes a una medicion normal al servidor.

            \item listen: Es un proceso que se encarga de verificar si hay una
                solicitud de informacion, ya sea del subproceso timer (como un
                mensaje normal) o del servidor para solicitar informacion de la
                vista exhaustiva (la cantidad de informacion enviada es sustancialmente
                diferente) o para la terminacion del proceso e informa a handler
                que se debe hacer.

            \item handler: Se encarga de realizar la tarea pedida por listen,
                al enviar la informacion solicitada al servidor, enviando 2 mensajes,
                uno con el tipo de informacion que se envia y otro con la informacion.
                Esto fue establecido como una especie de protocolo para dar mayor
                seguridad y a la vez facilitar el intercambio de informacion con
                el servidor.
        \end{enumerate}

        Cabe resaltar que estos procesos son asincronos y no sufren prelaciones
        entre ellos.
\end{itemize}

\subsubsection{Cliente WEB}

Este cliente es el mas conocido, ya que es el que permite que se muestre la
informacion en el navegador web. Esta constituido por las vistas general,
especifica y exhaustiva, cada una de estas representa una pagina web separada y
todas fueron construidas utilizando el framework de javascript \textbf{React}
y con html y css para dar estructura basica y estilo. Se opto por utilizar un
estilo multi paginas con renderizado de lado servidor (especificamente del
servidor web) por la necesidad de los calculos avanzados y graficas que
se deben realizar, ademas de los llamados importantes e interconexiones con la
api del servidor de Go, y la BBDD. Esto difiere con el paradigma tradicional de
React (monopagina de renderizado en servidor) pero permite optimizar recursos y
facilita expansiones a futuro.

Su estructura viene dada por las 3 paginas o sub aplicaciones que permiten:

\begin{itemize}
    \item General: conocer el estado general de un grupo de motores, indicando
        en codigo de colores (verde, amarillo, rojo) el nivel de daño que posee
        un motor. Este nivel es determinado por las muestras mas actuales de
        la informacion del motor y unos valores parametros proporcionados en
        conjunto con los datos de los cuales se elaboro el modelo estadistico.

        Cabe resaltar que estos valores son una extrapolacion empirica de la
        vibracion en una planta especifica, es configurable y puede variar dadas
        las caracteristicas propias de cada instalacion. Esto es debido a que
        las bases utilizadas en la instalacion, los soportes, entre otros factores
        \textbf{causados por ignorar las normas de instalacion} afectan las medidas.

    \item Especifica:  permite el estudio del estado de un motor especifico,
        este es enviado como parametro en el url al servidor. Permite observar
        un histograma y una tabla exportable a excel de sus caracteristicas y
        evolucion en el tiempo, siempre y cuando se tenga medicion de ese periodo
        en la base de datos, asimismo permite solicitar la vista exhaustiva del
        mismo motor si es requerido un nivel mayor de analisis.

    \item Exhaustiva: esta vista incluye todo lo anterior de la vista especifica,
        con la salvacion de que permite regresar al menu general en vez de hacer
        una vista exhaustiva, ademas de que realiza un analisis en frecuencia
        del estado a tiempo real del motor (para que se pueda realizar este
        analisis se tiene que tener acceso a tiempo real con el motor, es decir,
        el servidor de sensorica debe tener una conexion con un sensor que se
        encargue de monitorear el respectivo motor permitiendole asi solicitar
        la informacion necesaria para hacer el analisis.

        Cabe resaltar que esta accion es tratada como una vista aparte ya que
        tiene un peso computacional relativamente alto asociado, y por ende,
        ademas de consumir recursos, tiene una duracion de carga de algunos
        segundos que deteriora la experiencia de usuario y por esto se busca que
        sea obtenida solamente por una solicitud explicita.
\end{itemize}




\subsection{COMPROBACIÓN DE LOS RESULTADOS}
