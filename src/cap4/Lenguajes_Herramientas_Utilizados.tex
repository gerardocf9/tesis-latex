\thispagestyle{empty}

\section{ANÁLISIS Y DISCUSIÓN DE RESULTADOS}

\subsection{PROCEDIMIENTO DE RECOLECCIÓN DE INFORMACIÓN}
Esta parte del proyecto es bastante extensa así que debe de ser dividida en dos
categorías, investigación de campo e investigación documental.

La investigación de campo, según \textcite{MetodologiaInvestigacion}
es aquella que consiste en la recolección de datos directamente de los sujetos
investigados, o de la realidad donde ocurren los hechos,
sin manipular o controlar variable alguna, es decir, el investigador obtiene la
información pero no altera las  condiciones existentes. De allí su carácter de
investigación no experimental.

De esta forma, análisis de campo juega un rol importantísimo para obtener la
información y los requerimientos mínimos para el diseño de la interfaz de usuario
y del modelo estadísticos.

Asimismo, la investigación documental, según \textcite{MetodologiaInvestigacion}
es un proceso
basado en la búsqueda, recuperación, análisis, critica e interpretación de datos
secundarios, es decir, los obtenidos y registrados por otros investigadores en
fuentes documentales impresas, audiovisuales o electrónicas,
siendo esta fundamental para la creación de clasificaciones,comparaciones, tablas
y la implementación propiamente dicha del trabajo. Para esto se revisaron
libros, trabajos de investigación, documentos de fuente electrónicas, cursos,
entre otros.

\subsection{ELECCIÓN DE LAS HERRAMIENTAS Y LENGUAJES}
Se debe resaltar que los valores otorgados a las características en los lenguajes
fueron decisión tomadas a juicios y criterios  personales entre los miembros
del trabajo y orientada a conocimientos obtenidos en los momentos de investigación,
y pre-existentes.

    Dada la gran cantidad de actividades y los requerimientos que tenían,
    se hizo un estudio de los lenguajes y herramientas mas utilizados en
    la actualidad para tareas similares y junto a esto, se establecieron
    ciertos criterios y requerimientos mínimos para cada tarea a realizar, de
    esta forma, se tomaron las siguientes decisiones:

    \subsubsection{Servidor}
    La decisión del servidor se puede  dividir de acuerdo a los
    microservicios a realizar, es decir, el servidor dedicado a sensorica y
    el servidor Web. En ambos casos se manejaron los mismos lenguajes, asimismo, debido a
    las características variantes entre estos y la posibilidad, dada la arquitectura
    distribuida empleada de utilizar múltiples herramientas, se optó por utilizar
    el que mejor cumplía los requerimientos establecidos; como se observa en la
    tabla \ref{tab:LenguajesServidor}, se comparan los lenguajes Golang, Python
    con Framework Django y NodeJs, esto es debido a que los servidores siguen un
    sistema Web y es mas fácil el desarrollo en estos lenguajes especializados que
    en sistemas como Apache o Nginx mas robustos pero generales.

    \begin{table}[ht]
        \caption[Comparativa de posibles lenguajes nivel servidor]{comparación entre
        los posibles lenguajes utilizables para servidores}
        \label{tab:LenguajesServidor}
        \begin{center}
            Distintas opciones manejadas al momento de desarrollar los servidores.\\

            \vspace{0.3cm}
            \begin{tabular}{|c|c|c|c|}
                \hline
                Característica              & Golang & Django & NodeJs\\\hline
                Velocidad                   & 10    & 7     &   8   \\\hline
                Facilidad de implementación & 8     & 9     &  7\\\hline
                Robustez                    & 6     & 9     & 7 \\\hline
                Escalabilidad               & 10    & 10    & 9 \\\hline
                Paralelismo                 & 10    & -     & 2 \\\hline
                Facilidades a conexión Http2& 9     &7      & 7 \\
                \hline
            \end{tabular}
        \end{center}
    \end{table}

    Para el microservicio de sensorica, las características mas importantes fueron
    el paralelismo y la facilidad de implementar una conexión Http2, por la
    necesidad de una conexión full dúplex; por estos motivos Golang fue un claro
    ganador dado que es un lenguaje diseñado para microservicios y paralelismo y
    el uso de librerías facilitan increíblemente el establecer una conexión Http2.

    Por otro lado, en términos del servidor Web, las necesidades giraban en torno
    a la facilidad de implementación, robustez y escalabilidad que otorgaban los
    lenguajes siendo en este caso Python + Django el ganador indiscutible.


    Cabe resaltar que paralelismo hace referencia al mejor uso posible de los
    recursos de hardware del servidor, por la gran cantidad de conexiones
    continuas y tareas adicionales que deben  ser realizadas y estar
    establecidas, no a la cantidad de peticiones que recibe el servidor.

    \subsubsection{Cliente}

    En términos de clientes también se debe hacer diferencia entre los 2 sistemas
    creados, cliente de sensorica y cliente Web. El cliente de sensorica se
    implementó en Golang por las mismas razones dadas para el servidor de sensorica.

    Por otro lado, en la actualidad los clientes Web están compuestos casi en
    su totalidad de una combinación de HTML-CSS-JavaScript, existen casos en los
    que no se usa JavaScript o clientes escritos en otros lenguajes e interpretados
    en la web; de esta forma, la decisión está en qué Framework de JavaScript se
    va a utilizar, existe una gran variedad que cubre desde JavaScript ``vainilla",
    sin Framework y JQuerry, que otorgan interactividad pero se complica rápidamente
    cuando la aplicación crece, hasta los 3 gigantes en la actualidad, ReactJs,
    Angular y VueJs, que facilitan y agilizan increíblemente el proceso de desarrollo.
    Como se explica en la tabla \ref{tab:LenguajesCliente}, La razón mas importante
    en la toma de la decisión fueron conocimientos previos con ReactJs.


    \begin{table}[ht]
        \caption[Comparativa de posibles lenguajes nivel cliente Web]{comparación entre
        los posibles lenguajes utilizables para clientes Webs}
        \label{tab:LenguajesCliente}
        \begin{center}
            Distintas opciones manejadas al momento de desarrollar los clientes Web.\\

            \vspace{0.3cm}
            \begin{tabular}{|c|c|c|c|}
                \hline
                Característica              & JavaScript & React & Otros JsFrameworks\\\hline
                Facilidad de implementación & 1         & 7     &  7\\\hline
                Conocimientos previos       & 6         & 5     &  - \\\hline
            \end{tabular}
        \end{center}
    \end{table}



    \subsubsection{Modelo estadístico}

    \subsubsection{Análisis en frecuencia}


