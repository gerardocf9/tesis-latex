\thispagestyle{empty}

\section{ANÁLISIS Y DISCUSIÓN DE RESULTADOS}

\subsection{PROCEDIMIENTO DE RECOLECCIÓN DE INFORMACIÓN}

\subsection{ELECCIÓN DE LAS HERRAMIENTAS Y LENGUAJES}
    Dada la gran cantidad de actividades y los requerimientos que tenian las mismas,
    se hizo un estudio de los lenguajes y herramientas mas utilizados en
    la actualidad para tareas similares y junto a esto, se establecieron
    ciertos criterios y requerimientos minimos para cada tarea a realizar, de
    esta forma, se tomaron las siguientes decisiones:

    \subsubsection{Servidor}
    La decision del servidor se puede a su vez dividir de acuerdo a los
    microservicios realizados, es decir, el servidor dedicado a sensorica y
    el servidor Web. En ambos casos se manejaron los mismos lenguajes, asimismo, debido a
    las caracteristicas variantes entre estos y la posibilidad, dada la arquitectura
    distribuida utilizada, de utilizar multiples herramientas se opto por utilizar
    el que mejor cumplia los requerimientos establecidos; como se observa en la
    tabla \ref{tab:LenguajesServidor}, se comparan los lenguajes Golang, Python
    con framework Django y NodeJs, esto es debido a que los servidores siguen un
    sistema Web y es mas facil el desarrollo en estos lenguajes especializados que
    en sistemas como Apache o Nginx mas robustos pero generales.

    \begin{table}[ht]
        \begin{center}
            Distintas opciones manejadas al momento de desarrollar los servidores.\\

            \vspace{0.3cm}
            \begin{tabular}{|c|c|c|c|}
                \hline
                Caracteristica              & Golang & Django & NodeJs\\\hline
                Velocidad                   & 10    & 7     &   8   \\\hline
                Facilidad de implementacion & 8     & 9     &  7\\\hline
                Robustez                    & 6     & 9     & 7 \\\hline
                Escalabilidad               & 10    & 10    & 9 \\\hline
                Paralelismo                 & 10    & -     & 2 \\\hline
                Facilidades a conexion http2& 9     &7      & 7 \\
                \hline
            \end{tabular}
        \end{center}
        \caption[Comparativa de posibles lenguajes nivel servidor]{Comparacion entre
        los posibles lenguajes utilizables para servidores}
        \label{tab:LenguajesServidor}
    \end{table}

    Para el microservicio de sensorica, las caracteristicas mas importantes furon
    el paralelismo y la facilidad de implementar una conexion http2, por la
    necesidad de una conexion full duplex, y por estos motivos Golang fue un claro
    ganador, dado que es un lenguaje diseñado para microservicios y paralelismo y
    el uso de librerias facilitan increiblemente el establecer una conexion http2.

    Por otro lado, en terminos del servidor Web, las necesidades giraban entorno
    a la facilidad de implementacion, robustez y escalabilidad que otorgaban los
    lenguajes siendo en este caso Python + Django el ganador indiscutible.


    Cabe resaltar que paralelismo hace referencia al mejor uso posible de los
    recursos de hardware del servidor por la gran cantidad de conexiones
    continuas y tareas adicionales que deben de ser realizadas y estar
    establecidas, no a la cantidad de peticiones que recibe el servidor.

    \subsubsection{Cliente}

    En terminos de clientes tambien se debe hacer diferencia entre los 2 sistemas
    creados, cliente de sensorica y cliente Web. El cliente de sensorica se
    implemento en Golang por las mismas razones dadas para el servidor de sensorica.

    Por otro lado, en la actualidad los clientes Webs estan compuestos casi en
    su totalidad de una combinacion de HTML-CSS-JavaScript, existen casos en los
    que no se usa JavaScript o clientes escritos en otros lenguajes e interpretados
    en la web; de esta forma, la decision esta en que framework de javascript se
    va a utilizar, existen una gran variedad que cubre desde JavaScript ``vanilla",
    sin framework y JQuerry, que otorgan interactividad pero se complica rapidamente
    cuando la aplicacion crece, hasta los 3 gigantes en la actualidad, ReactJs,
    Angular y VueJs, que facilitan y agilizan increiblemente el proceso de desarrollo.
    Como se explica en la tabla \ref{tab:LenguajesCliente}, La razon mas importante
    en la toma de la decision fueron conocimientos previos con ReactJs.


    \begin{table}[ht]
        \begin{center}
            Distintas opciones manejadas al momento de desarrollar los clientes Webs.\\

            \vspace{0.3cm}
            \begin{tabular}{|c|c|c|c|}
                \hline
                Caracteristica              & JavaScript & React & Otros JsFrameworks\\\hline
                Facilidad de implementacion & 1         & 7     &  7\\\hline
                Conocimientos previos       & 6         & 5     &  - \\\hline
                \hline
            \end{tabular}
        \end{center}
        \caption[Comparativa de posibles lenguajes nivel cliente Web]{Comparacion entre
        los posibles lenguajes utilizables para clientes Webs}
        \label{tab:LenguajesCliente}
    \end{table}



    \subsubsection{Modelo estadistico}

    \subsubsection{Analisis en frecuencia}
