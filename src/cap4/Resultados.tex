
\subsection{COMPROBACIÓN DE LOS RESULTADOS}

Como se ha mencionado anteriormente, el trabajo consta de una gran cantidad de
puntos que han sido trabajados de forma modular con la finalidad
de maximizar la escalabilidad y de facilitar al máximo posible el realizar pruebas
a los módulos y a los acoples realizados. La comprobación debe de
tomar en consideración 2 factores, los resultados técnicos, es decir
funcionalidades de la aplicación las cuales, como se mencionó anteriormente, no se
automatizaron por factores tiempo y cantidad de pruebas y, los resultados finales,
es decir la capacidad de la herramienta de facilitar el trabajo, mediante
las páginas generales, específicas y exhaustivas.

\subsubsection{Comprobación de resultados a nivel técnico}

Las fallas internas de los servidores, o de procesos de comunicación
con la base de datos, conllevan a la finalización de la ejecución del proceso
(aplicación) correspondiente; esto se hace para poder reiniciar el sistema y
agregar el error o falla catastrófica a un archivo (log) que sirve para seguir el
comportamiento del servidor.

En el caso del servidor de sensorica, estos errores son manejados de forma manual
y se utiliza la característica de Go \textbf{``panic"} para detener la ejecución, mas
específicamente un paquete de la librería estándar \textbf{``log"} con la función
\textbf{``log.Faltalf"}.
Por otro lado, el Framework de Django se encarga de esto, si el error es manejable;
algún problema en una petición, o algo no catastrófico, arroja una excepción y un
error 500, fallo interno en el servidor, y finaliza la petición; en el caso de no
poder ser recuperado finaliza el proceso.

La finalización del servidor es una práctica relativamente común ya que permite
el reinicio del proceso, esto se hace mediante una configuración interna al
momento de hacer \textbf{deploy} en el servidor, como se observa en la figura
\ref{img:ProcesosLinux}
es la configuración recomendada por DigitalOcean para un reinicio
continuo, en caso de falla, a intervalos de 5 segundos.


\begin{figure}[H]
    \centering
    \caption{Configuración de proceso en Linux para reinicio automático de procesos.}
    \includegraphics[width=\linewidth]{comprobacion_resultados/tecnicos/reinicio_servidor.png}
    Fuente: \textcite{configuracionprocesos}.
    \label{img:ProcesosLinux}
\end{figure}

Por esta razón, las pruebas mas considerables son interconexiones y configuraciones
de seguridad y problemas de datos. es decir:

\begin{itemize}
    \item Prueba de no unicidad de la información, al haber múltiples
        clientes de sensorica enviando información sobre el mismo motor.

        Para esta posibilidad, se optó por una configuración que rechaza el
        segundo intento de conexión, es decir, se rechaza cualquier intento de
        conexión de motor con una conexión establecida previamente (que todavía
        exista), y su implementación se observa en la figura \ref{img:NoUnicidad}.
        %
        \begin{figure}[H]
            \centering
            \caption{Prueba de error, no unicidad de información}
            \includegraphics[width=\linewidth]{comprobacion_resultados/tecnicos/NoUnicidadSensorica.png}

            logs: Izquierda cliente conectado, Centro cliente Rechazado, Derecha.
            Servidor.
            \label{img:NoUnicidad}
        \end{figure}
        %
    \item Prueba de falla en la conexión de la base de datos.

        Cada servidor toma esta falla de una forma específica, en el caso del
        servidor de sensorica, esta es considerada una falla catastrófica, por
        lo que se finaliza la ejecución, como se observa en la figura
        \ref{img:NoBBDDSensorica}.

        Por otro lado, el servidor Web arroja una excepción, la cual es manejada
        internamente por Django y resuelta de forma que la petición es respondida
        con un error 500, falla interna del servidor, este error en la respuesta
        dada es manejado por el cliente y se observa en la figura
        \ref{img:NoBBDDWeb} .
        %
        \begin{figure}[H]
            \centering
            \caption{Prueba de error, no BBDD Sensorica}
            \includegraphics[width=\linewidth]{comprobacion_resultados/tecnicos/conexionInvalidaNoBBDDSensorica.png}

            El sistema no inicia por no poder conectarse a la BBDD.
            \label{img:NoBBDDSensorica}
        \end{figure}

        \begin{figure}[H]
            \centering
            \caption{Prueba de error, no BBDD Web}
            \includegraphics[width=\linewidth]{comprobacion_resultados/tecnicos/conexionInvalidaNoBBDDWeb.png}

            El sistema continua ejecución y resuelve el error con un error 500.
            \label{img:NoBBDDWeb}
        \end{figure}
        %
    \item Prueba de solicitud de vista exhaustiva a un motor sin conexión
        establecida.

        Este caso corresponde al servidor de sensorica; al no haber una conexión
        con el sensor establecida no se pueden tomar las mediciones para una vista
        exhaustiva, estas deben ser en tiempo real, por lo tanto se responde
        con una error en la respuesta como se observa en la figura \ref{img:ErrorExhaustiva}.
        %

        %
    \item Prueba de petición no completada en el cliente Web,
        información invalida (motor no existente) o error del servidor Web.

        En este caso se muestra al cliente un error por pantalla, como ventana
        emergente, y se redirige a la vista general como se observa en la
        figura \ref{img:SolicitudNoCompletada}.
\end{itemize}

        \begin{figure}[H]
            \centering
            \caption{Prueba de error, petición no completada en el cliente Web}
            \includegraphics[width=0.8\linewidth]{comprobacion_resultados/tecnicos/conexionInvalidaNoMotorSensorica.png}

            Solicitud  desde el navegador, para facilitar vista, al servidor
            sin motores conectados.
            \label{img:ErrorExhaustiva}
        \end{figure}

        \begin{figure}[H]
            \centering
            \caption{Prueba de error, petición exhaustiva inválida}
            \includegraphics[width=0.9\linewidth]{comprobacion_resultados/tecnicos/redireccionClienteWeb.png}

            Solicitud no completada en el cliente Web. \label{img:SolicitudNoCompletada}
        \end{figure}

\subsubsection{Comprobación de resultados finales}
Una vez terminado todo el proyecto se debieron rectificar que los requerimientos
solicitados, en términos de parámetros, sean cumplidos al igual que las gráficas
resultantes, para las históricas como para el análisis en frecuencia, tengan
coherencia, dado esto se procedió de la siguiente forma:

\begin{itemize}
    \item Corroboración del análisis en nivel de daño básico (vista general)
        que se ajuste a los parámetros previamente establecidos.

        Para esta prueba se ingresaron 5 motores a la BBDD cuya información fue
        estrictamente configurada para cumplir los 3 estados de daño, y los
        2 valores frontera. Consiguiendo, de esta forma, cubrir todas las
        posibilidades, este resultado puede verse en las figuras
        \ref{img:NivelesMotores}
        y
        \ref{img:FronterasEnLosMotores},     donde el color
        implica el estado y la tabla los valores que tiene el motor
        para los cálculos.


\item Confirmación de las gráficas históricas.

        Esta verificación es conseguida mediante la comparación con los valores
        históricos, igual que para la prueba anterior, se ingresó un motor
        con valores de medición fáciles de seguir para corroborar la legibilidad
        y precisión de la gráfica, como se puede observar en la figura \ref{img:GraficasHistoricas}.

    \item Demostración de la gráfica de la transformada de Fourier.

        En este caso se procedió a comparar la gráfica obtenida mediante el
        algoritmo de Python con una obtenida mediante el software Octave (versión
        código abierta de Matlab), los resultados pueden verse en la figura
        \ref{img:FFTSensores}.
\end{itemize}







