\subsection{IMPLEMENTACIÓN DEL MODELO ESTADÍSTICO}

El diseño del modelo comenzó con la limpieza de los datos y la selección de las
variables de interés de la base de datos proporcionada. Las variables de
interés fueron la velocidad horizontal, la velocidad vertical, la aceleración,
la potencia del equipo, el tipo de equipo y el código del mismo. Una vez
obtenida la data se le realizó un análisis exploratorio, el cual
consistió principalmente en la elaboración de gráficas y sacar estadísticos de
las variables de interés de lo cual se obtuvieron las observaciones que se incluyen
a continuación.

Las variables de velocidad horizontal, velocidad vertical y aceleración son
siempre positivas, todas muestran en su distribución una asimetría con
tendencia hacia la izquierda, lo cual las hace similar a distribuciones como la
exponencial, la weibull o la log-normal. Cuando se calcula la correlación entre
las variables velocidad horizontal y vertical obtenemos una
correlación de Pearson del $56.71\%$, el cuadrado de este valor o $R^2$ (también
conocido como coeficiente de determinación) es del $32.16\%$ y se puede
interpretar como que el valor de la velocidad horizontal predice $32\%$ del valor
de la velocidad vertical, estos valores son muy elevados para considerar las
variables como independientes. En el caso de la aceleración las correlaciones
no fueron significativas.

Debido a la correlación considerable entre las dos velocidades no era posible
modelar las variables como independientes pero tampoco era tan elevada para
realizar una regresión lineal, por lo tanto se recurrió a buscar un cambio de
variables en el cual desapareciera la dependencia entre las dos variables
aleatorias. Como la velocidad es un vector y se tienen sus componentes,
un cambio de variables lógico es el de coordenadas polares ya que como es
una transformación biyectiva no se pierde información aplicándola y, a
su vez, es fácil de interpretar. La correlación entre las nuevas variables
magnitud y ángulo fue del $-25.18\%$ y un valor de $R^2$ del $6.34\%$ lo cual para
efectos prácticos se puede considerar como dos variables aleatorias
independientes. Similarmente con las velocidades con componentes cartesianos la
correlación con la aceleración es no significante.

Una vez obtenidas las tres variables de interés, para modelar se buscó la
distribución que mejor se aproxima a los datos, la escogencia de la
distribución de probabilidad por lo general no es un proceso cuantitativo sino
que el investigador conociendo en profundidad el problema escoge entre un grupo
de distribuciones conocidas, cuál se adapta mejor al problema planteado, pero
debido a que hoy en día existen herramientas computacionales para la
estadística este proceso se puede hacer un poco más cuantitativo. Para la
selección de los modelos de cada variable se iteró entre 84 distribuciones
pertenecientes a Scipy y se aplicó a cada una la prueba de Kolmogorov–Smirnov,
se filtraron aquellas distribuciones con valores P mayores a $5\%$ y, finalmente, se
ordenaron de mayor a menor, debido a que la hipótesis nula de la
prueba de Kolmogorov-Smirnov es que si las distribuciones poseen la misma
distribución un valor elevado de P indica que la distribución, se aproxima al
valor teórico.

El modelo estadístico seleccionado fue el siguiente, se tienen 3
variables independientes: magnitud de la velocidad, ángulo de la velocidad y
aceleración. Para la magnitud de la velocidad se seleccionó una distribución
Burr de tipo III debido a que es una distribución netamente positiva y la
prueba de Kolmogorov dio un valor P del $78,01\%$ siendo este el segundo valor más
alto y el primero que cumple la condición de ser positiva siempre, en el caso
del ángulo múltiples distribuciones satisfacen pero como la distribución normal
tenía un valor P del $93.79\%$ y es una distribución conocida, se seleccionó esta;
finalmente, en el caso de la aceleración la distribución de Burr tipo III poseía
un valor P de $99.28\%$ si bien había otras distribuciones con valores ligeramente
mayor se optó por escoger esta porque ya la magnitud de la velocidad seguía esta
distribución y simplifica la implementación.

En el caso de la vista exhaustiva se utilizó una mezcla entre un modelo de la
señal no estadístico y el modelo estadístico previamente seleccionado de tal
forma que se pueda obtener una gran cantidad de datos para poder realizar el
análisis de frecuencia pero a la vez conservar la coherencia con el resto de
mediciones; para esto se ajustó la amplitud de la señal de acuerdo al valor
esperado según el modelo de la aceleración y para expresar el ruido de la
medición a los parámetros del modelo se les suma una señal de ruido de tipo
gaussiana.

Una vez obtenido el modelo, se implementó una API Web usando FastAPI con dos
endpoints, uno utilizado para la vista general y específica y otro para la
vista exhaustiva. Ambos endpoints solicitan el nivel de daño el cual es un
número entre 0 y 10 que indica el grado de daño del motor y solicitan el id del
motor, el cual es un número entero  que identifica el motor. Esta API se encuentra
en un Host independiente en Digital Ocean y la accede el microservicio de Go
para obtener los valores del modelo.
