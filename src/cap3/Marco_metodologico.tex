\thispagestyle{empty}

\section{MARCO METODOLÓGICO}

Como fue explicado en el primer capítulo, se pretende hacer una herramienta
computacional para el análisis de la vibración en motores eléctricos alimentada
mediante datos de una simulación digital. Partiendo de esto, se comenzará
con la definición de los siguientes aspectos.

\subsection{NATURALEZA DE LA INVESTIGACIÓN}

El presente trabajo es clasificado como Proyecto Especial, puesto que según
\textcite{Hernandez} lleva~a:

\begin{center}
    \parbox[ht]{13.5 cm}{Trabajos que lleven a creaciones tangibles,
    susceptibles de ser utilizadas como soluciones a problemas demostrados, o
    que respondan a necesidades e intereses de tipo cultural. Se incluyen en
    esta categoría los trabajos de elaboración de libros de texto y de
    materiales de apoyo educativo, el desarrollo de software, prototipos y de
    productos tecnológicos en general, así como también los de creación
    literaria y artística.}
\end{center}


\subsection{TIPO DE INVESTIGACIÓN}

De acuerdo con la clasificación el tipo de investigación de este trabajo se
cataloga como investigación aplicada, puesto que "persigue fines inmediatos y
concretos a través de la búsqueda de un nuevo conocimiento técnico con aplicación
inmediata a un problema determinado"\ \textcite{Velez}.

\subsection{FASES DE LA INVESTIGACIÓN}

En función de los objetivos específicos, se pueden definir las fases que conforman
la investigación:

\subsubsection{Definición de los requerimientos de usuario}
Esta fase tiene relación directa con la concepción del proyecto ya que fue el
planteamiento de los problemas y posibles soluciones, separación de vistas,
mecanismos de análisis, entre otros. Sin una concepción  de las estructuras
de datos y los lenguajes necesarios para elaborarla.
Como en todo proyecto, esta fase se logro mediante la interacción con un ingeniero
mecánico que propuso la idea y facilitó los recursos e información necesaria
con la finalidad de puntualizar los requerimientos
mínimos necesarios para obtener una aplicación viable.

De esta manera, la primera fase ha constado de las siguientes actividades:

\begin{itemize}
    \item Selección de la cantidad de vistas óptimas para facilitar el estudio
        y el monitoreo de la planta, siendo esto decidido a 3, una vista
        \textbf{general} del estado de los motores en la planta, una
        \textbf{especifica} en donde se observa el comportamiento de un motor,
        una \textbf{exhaustiva} que añada información a la especifica, gráfica en
        el dominio de la frecuencia, agregada para mejorar la experiencia de
        usuario, por la demora necesaria para generar este estudio.
    %
    \item Definición de la estructura mínima necesaria para poder evaluar un
        motor, en términos de evolución histórica de la vibración, con base en
        una tabla histórica o gráfica (histórica).
    %
    \item Definición del tipo de transformada de Fourier a graficar para el
        análisis exhaustivo (por gustos y costumbres en la industria).
    %
    \item Entrega de una base de datos histórica con mediciones reales y esporádicas
        del estado de los motores en una fábrica.
    %
    \item Especificación de los parámetros a considerar para determinar un nivel
        básico del estado de un motor, en términos de velocidad y aceleración.
        Se debe resaltar que estos valores difieren de los establecidos a nivel
        teórico y recomendado en los estándares por factores inherentes a las
        empresas (malas instalaciones,
        utilización de bases artesanales, políticas de daños mínimos) los cuales
        amplían los rangos teóricos.
    %
    \item Selección de las unidades en las que se tomarán las medidas, siguiendo
        estándares, y siendo $\frac{mm}{s}(rms)$ para la velocidad y $g$ para la
        aceleración.
\end{itemize}

Es importante recalcar el hecho de que los parámetros utilizados, tanto
para el establecimiento de niveles de daño en el modelo estadístico como
en la vista general para distinguir el estado de los motores, son el resultado
empírico asociados a años de estudios del comportamiento de los motores en la
planta utilizada. Esto se tomó en consideración al momento del diseño y se puede
modificar, vía modificación de constantes en el código, los rangos de valores
en velocidad y vibración adecuados a los estándares de cada planta o ingeniero
que utilice el sistema.

\subsubsection{Revisión documental}
Relacionada parcialmente con los primeros dos objetivos de la investigación,
se basó en los estudios necesarios para poder tomar la decisión del tipo de modelo
y lenguajes a utilizar, el planteamiento de requerimientos para los
lenguajes, dada la gran variedad de posibilidades disponibles para las distintas
tareas a realizar; de esta forma, se pueden enumerar las siguientes actividades
realizadas:

\begin{itemize}
    \item Recolección de información de diversos lenguajes de programación.
    \item Estudio de estadística y los distintos modelos y distribuciones existentes.
    \item Estructurar los criterios y requerimientos mínimos para poder seleccionar
        las herramientas a utilizar.
\end{itemize}

\subsubsection{Análisis de la información}
Fase previa a la implementación pero relacionada al primer objetivo
de la investigación. Se procedió a analizar la información en las bases de datos,
comparar con los requerimientos mínimos, previamente definidos, y estructurar
mas detalladamente
la nueva base de datos (creada posteriormente en MongoDB), los parámetros
necesarios para el modelo estadístico y la selección del tipo, orientado a una
distribución de probabilidad. Asimismo, se tomaron decisiones conforme a los
lenguajes utilizados y la estructura de microservicios que se utiliza en el
desarrollo. De esta forma, se distinguen los siguientes puntos en esta fase:

\begin{itemize}
    \item Organización de la información recolectada.
    \item Toma de decisión de los lenguajes y herramientas a utilizar, basada
        en los criterios previamente establecidos.
    \item Estructuración de los modelos para las bases de datos.
    \item Justificación de las decisiones tomadas.
    \item Estructuración y separación de las tareas a realizar por los distintos
        microservicios (sensorica, Web, BBDD).
\end{itemize}

\subsubsection{Diseño del modelo estadístico}
Corresponde al segundo objetivo del trabajo pero guarda relación con los
objetivos 3 y 4. Se utilizaron técnicas de exploración de data para la
selección de las variables de interés así como la formulación de hipótesis y se
usaron pruebas estadísticas para seleccionar las distribuciones de probabilidad
que mejor modelan las variables de interés. Se realizaron las siguientes
tareas:

\begin{itemize}
    \item Limpieza de los datos.
    \item Análisis exploratorio de la data.
    \item Cálculo de estadísticos de interés.
    \item Cálculo de pruebas de bondad de ajuste para cada variable de interés.
    \item Selección de las distribuciones que mejor modelan cada variable.
\end{itemize}

\subsubsection{Diseño de las vistas para mostrar la información}
Este apartado guarda relación con los objetivos 5 y 6. Se trabajo en la
estructuración y diseño gráfico de las vistas, además de las funcionalidades
implementadas para poder ser creada la página Web. Las tareas realizadas
fueron las siguientes:

\begin{itemize}
    \item Toma de decisión en utilizar una aproximación multipágina (multipage app).
    \item Diseño básico, barra de navegación, encabezado y pie de página.
    \item Diseño de la vista General y su funcionalidad.
    \item Diseño de la vista Especifica y su funcionalidad.
    \item Diseño de la vista Exhaustiva y su funcionalidad.
\end{itemize}

\subsubsection{Implementación}
Dada la gran cantidad de tareas a desarrollar, la implementación se subdivide
en un gran grupo de tareas, con la finalidad de cumplir los objetivos 2, 3, 4, 5, 6
estas divisiones, y subdivisiones, pueden ser expresada de la siguiente forma:

\begin{itemize}
        \item Generación del modelo estadístico:
    \begin{enumerate}
        \item Inspección manual de la data.
        \item Limpieza de las variables de interés.
        \item Elaboración de histogramas para observar la distribución de los datos.
        \item Seleccionadas las variables velocidad horizontal, velocidad
            vertical y aceleración como variables para el modelo.
        \item Cálculo de correlación entre variables.
        \item Cambio de variables para reducir la correlación entre velocidad
            horizontal y velocidad vertical.
        \item Búsqueda de distribuciones que mejor modelan la data, utilizando
            múltiples pruebas de Kolmogorov-Smirnov.
        \item Selección de distribuciones de Burr tipo III para modelar la
            magnitud de la velocidad y aceleración y distribución normal para
            modelar el ángulo.
        \item Cálculo de parámetros de las diferentes distribuciones. Creación
            de API en FastAPI para servir los datos.
        \item Conexión de la API con el microservicio de Go.
    \end{enumerate}

    \item Elaboración de la base de datos:
        \begin{enumerate}
            \item Registro y creación del cluster en MongoDB Atlas.
            \item Creación de la base de datos ``tesis"\  y las colecciones
                ``MotorData"\  y ``MotoresInDB".
            \item Llenado mediante el servidor de sensorica.
        \end{enumerate}

    \item Microservicios de sensorica:
        \begin{enumerate}
            \item Creación de las estructuras de datos necesarias para manejar
                la comunicación con MongoDB.
            \item Obtención de un certificado y TLS/SSL.
            \item Creación del servidor Https.
            \item Creación de un cliente de sensorica, para facilitar la emulación
                de los sensores.
            \item Implementación de las conexiones bidireccionales para el tráfico
                de información.
            \item Creación del punto de conexión ``/sensormessage" para permitir
                la conexión de múltiples clientes y recibir la información.
            \item Creación del punto de conexión ``/exhaustive?idMotor\=identificador"
                para poder enviar la información de la vista exhaustiva.
            \item Intercomunicaciones internas en el servidor para poder comunicar,
                solicitar y verificar información entre los dos puntos de conexión.
            \item Verificaciones de seguridad y conexiones.
        \end{enumerate}

    \item Análisis en frecuencia:
        \begin{enumerate}
                \item Obtención de las mediciones del servidor de Sensorica.
                \item Aplicación del algoritmo FFT para obtener la nueva
                    información.
                \item Graficar lo obtenido en el punto anterior.
        \end{enumerate}

    \item Servidor Web:
        \begin{enumerate}
            \item Configuración para el envío de información estática
                (HTML, CSS, Js, imágenes).
            \item Creación del punto de conexión ``/" para la vista general.
            \item Creación del punto de conexión ``/especifica?IdMotor\= identificador"
                para la vista especifica.
            \item Creación del punto de conexión ``/exhaustiva?IdMotor\=identificador"
                para la vista exhaustiva.
            \item Creación de gráficas históricas.
            \item Solicitud de información y análisis en frecuencia, creación de
                la gráfica de frecuencia.
            \item Establecimientos de los puntos de conexión para las API que
                suministran las informaciones necesarias para las vistas (uno
                para cada vista).
        \end{enumerate}

    \item Muestreo de información, mediante las vistas:
        \begin{enumerate}
            \item Solicitudes de información a las API (fetch).
            \item Configuración de los valores y los niveles de daño.
            \item Establecimiento del motor svg con colores dependiendo del nivel
                de daño.
            \item Mecanismo de paginación y búsqueda para hacer mas fácil y agradable
                la interfaz de usuario.
            \item Creación de la vista general.
            \item Implementación de la tabla y la funcionalidad para exportar a
                Excel.
            \item Solicitud y establecimiento de las imágenes-gráficas.
            \item Implementación de la vista especifica.
            \item Implementación de la vista exhaustiva.
            \item Diseño y estilizado mediante CSS.
        \end{enumerate}

    \item Interconexiones:
        \begin{enumerate}
            \item Interconexiones entre cliente y servidor de sensorica, mediante
                Http2 para poder establecer una comunicación bidireccional (full dúplex).
            \item Interconexión entre servidor de sensorica y API de modelo
                estadístico para información normal, grupo de 10
                mediciones de velocidad y aceleración por cada sensor.
            \item Interconexión entre servidor de sensorica y API del modelo
                estadístico para información exhaustiva, grupo de 1K de mediciones
                de aceleraciones, por sensor registrado al motor.
            \item Interconexiones entre el servidor de sensorica y la base de datos
                (MongoDB Atlas).
            \item Interconexiones entre el servidor Web y la base de datos (MongoDB Atlas).
            \item Interconexión entre servidor Web y servidor de sensorica, para
                solicitar la vista información asociada a la vista exhaustiva.
            \item Interconexión entre cliente Web y servidor Web para obtener
                información de archivos estáticos.
            \item Interconexión entre cliente Web y API del servidor Web para
                obtener la información referente a las vistas general, especifica
                y exhaustiva. Una interconexión y una API para cada vista.
        \end{enumerate}
\end{itemize}


\subsubsection{Comprobar los resultados}
Esta fase hace referencia al último objetivo de la investigación: Comprobar los
resultados de la herramienta de análisis. Dada la extensión de la herramienta,
estas pruebas se pueden desglosar en dos niveles, resultados finales (evaluación
del análisis de la herramienta) y pruebas técnicas (también llamado testing).
Esto da origen a las siguientes tareas:

\begin{itemize}
    \item Testing: La elaboración modular del sistema permite  corroborar su
        funcionamiento de múltiples formas, dado que unas pruebas automatizadas
        escapan de esta investigación (por factores de tiempo y cantidad de pruebas
        y complejidad de las mismas a nivel de código) se toman pruebas manuales
        de sistemas e interconexión entre los mismos:
        \begin{enumerate}
            \item Prueba de no unicidad de la información, al haber múltiples
                clientes de sensorica enviando información sobre el mismo motor.
            \item Prueba de falla en la conexión de la base de datos.
            \item Prueba de solicitud de vista exhaustiva a un motor sin conexión
                establecida.
            \item Prueba de petición no completada en el cliente Web,
            información inválida (motor no existente) o error del servidor.
        \end{enumerate}

    \item Evaluación de resultados:
        \begin{enumerate}
            \item Corroboración del análisis en nivel de daño básico (vista general)
                que se ajuste a los parámetros previamente especificados.
            \item Corroboración de las gráficas históricas.
            \item Corroboración de la gráfica de la transformada de Fourier.
        \end{enumerate}
\end{itemize}


