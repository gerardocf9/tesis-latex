\thispagestyle{empty}

\section{MARCO METODOLÓGICO}

Como fue explicado en el primer capítulo, se pretende hacer una herramienta
computacional para el análisis de la vibración en motores eléctricos alimentada
mediante datos de una simulación digital. Partiendo de esto, se comenzará
con la definición de los siguientes aspectos.

\subsection{NATURALEZA DE LA INVESTIGACIÓN}

El presente trabajo es clasificado como Proyecto Especial, puesto que según
\textcite{Hernandez} lleva~a:

\begin{center}
    \parbox[ht]{13.5 cm}{Trabajos que lleven a creaciones tangibles,
    susceptibles de ser utilizadas como soluciones a problemas demostrados, o
    que respondan a necesidades e intereses de tipo cultural. Se incluyen en
    esta categoría los trabajos de elaboración de libros de texto y de
    materiales de apoyo educativo, el desarrollo de software, prototipos y de
    productos tecnológicos en general, así como también los de creación
    literaria y artística.}
\end{center}


\subsection{TIPO DE INVESTIGACIÓN}

De acuerdo con la clasificación el tipo de investigación de este trabajo se
cataloga como investigación aplicada, puesto que "persigue fines inmediatos y
concretos a través de la búsqueda de un nuevo conocimiento técnico con aplicación
inmediata a un problema determinado"\ \textcite{Velez}.


\subsection{PROCEDIMIENTO DE RECOLECCIÓN DE INFORMACIÓN}

\subsection{FASES DE LA INVESTIGACIÓN}

    %Procedimiento de recoleccion de datos e informacion
    %Fases de la investigacion
