
\subsection{LIMITACIONES Y ALCANCES}
	En función de los objetivos planteados con anterioridad, se puede definir tanto las limitaciones como el alcance del proyecto.

\subsubsection{Limitaciones}
\begin{itemize}
	\item Recursos económicos impiden la adquisición de dispositivos para pruebas en motores reales.

	\item Disponibilidad de muestras, aunque se cuenta con una BBDD lo suficiente grande para cubrir el comportamiento de la vibración en motores, incluso de distinta potencia, esta es discreta y con intervalos de tiempo considerables entre cada muestra.

	\item Hardware utilizado, dado que para el desarrollo del servidor y la prueba del mismo se cuenta es con computadoras portátiles con bastante antigüedad, limitando de esta forma los recursos disponibles y por tanto la velocidad de desarrollo.

	\item La gran variedad de conocimientos. El conocimiento requerido es bastante amplio y ademas cubre áreas fuera del campo de la electrónica, como lo son probabilidad y mecánica de las cuales se cuenta con muy poco conocimiento previo, motivo por el cual se ralentiza el avance del proyecto.

	\item Las establecidas por el software de tercero, utilizado para incrementar la velocidad en el diseño del proyecto (parte gráfica).
\end{itemize}

\subsubsection{Alcance}
	El principal alcance de este trabajo es el desarrollo de una herramienta computacional para el análisis de la vibración en motores eléctricos; en función de esto:

	\begin{itemize}
		\item La herramienta computacional permite una vista general del estado de todos los motores en un espacio previamente delimitado y seleccionado (planta o piso) que se encuentren en la BBDD.

		\item La herramienta computacional muestra el estado especifico de un motor, seleccionado previamente por el operando, con todas sus características de forma mas detallada. 

		\item La La herramienta computacional genera un análisis en frecuencia de la vibración en un motor especifico para facilitar el estudio y la toma de decisiones del operando.

		\item Dados los puntos anteriores, el sistema consta de una Base de datos la cual permita su correcto funcionamiento y el almacenamiento de los valores históricos de cada motor.

		\item Se implementa un modelo estadístico para generar la información necesaria para el llenado de la BBDD  de la herramienta computacional y de esta forma evaluar su funcionamiento.

		\item Todo el sistema puede ser accedido de forma web para facilitar el acceso.

		\item no se contempla la construcción de hardware de ningún tipo.


	\end{itemize}