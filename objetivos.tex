\subsection{Objetivos de la investigación}

\subsubsection{Objetivo general}
	\begin{enumerate}
		\item  Desarrollar la simulación digital para el monitoreo de motores eléctricos.
	\end{enumerate}
	
\subsubsection{Objetivos específicos}

	\begin{enumerate}
		\item Justificar la escogencia de las herramientas y lenguajes a utilizar, basándose en las necesidades y eficiencias requeridas.

		\item Determinar la distribución de la salida de acelerómetro digital, mediante un análisis estadístico en motores con distinto grado de daño.

		\item Establecer un sistema de base de datos para alimentar la salida de la simulación digital.
		\begin{huge}
			$\cdot$ 
		\end{huge}

		\item Realizar análisis de fallas, dada la salida de la distribución del acelerómetro, mediante un análisis en frecuencia.

		\item Mostrar la información solicitada de acuerdo al nivel de análisis seleccionado.

		\item Comprobar los resultados de la simulación digital.
	\end{enumerate}

	\vspace*{3cm}
	\begin{huge}
		los objetivos con $\cdot$ al final no nos convencen	
	\end{huge}
