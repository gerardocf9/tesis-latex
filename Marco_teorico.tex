\subsection{MARCO TEÓRICO}

Esta parte del capítulo expone el contenido teórico necesario para la
realización de este proyecto. Esto incluye informacion sobre los motores
electricos, El analisis de la vibracion, las herramientas computacionales, los
sistemas web y el modelado estadistico.

%https://es.wikipedia.org/wiki/Motor_el%C3%A9ctrico

\subsubsection{ Motores eléctricos}

Un motor electrico es un dispositivo que transforma energia electrica en
energia mecanica mediante la accion de campos magneticos, y estan compuestos
principalmente por un estator (parte fija) y un rotor (parte movil).
Existen dos familias principales de motores electricos las cuales, a su vez,
se subdividen en \textbf{motores de corriente alterna} como lo son los  motores
de induccion, sincrones, entre otros y los \textbf{motores de corriente continua}
como lo son los motores de escobillas,  sin escobillas, de iman permanente,
entre otros.

El motor mas usado en el sector industrial, es el motor de induccion, debido
principalmente a su bajo costo y al poco mantenimiento que requieren para estar
completamente operativo, esto en particular es debido a su sencillo diseño
,en comparacion al de otros motores de igual potencia, ya que requiere  menor
numero de componentes. Cabe resaltar que el segundo tipo de motor mas usado es
el motor de corriente
continua, debido a que su control, en terminos de torque y potencia, es mucho
mas sencillo para ciertos niveles de potencia.

Tambien existen los motores sincronos, cuya construccion es similar a la de un
motor de induccion pero a su vez requiere de una fuente de alimentacion externa
para el rotor, lo cual aumenta su costo, ademas de esto su velocidad es
constante, por lo tanto este tipo de motor es utilizado en aplicaciones
especificas. Como se menciono anteriormente, existen otros tipos de motores,
pero esto son usualmente de menor potencia por
lo tanto su utlidad industrial es mucho mas limitada.

\subsubsection*{Rodamientos}

Para que un motor pueda llevar a cabo la transformacion de potencia debe rotar.
Esta accion es llevada a cabo por el \textbf{rotor}, el cual esta formado por un
eje que soporta un juego de bobinas envueltas sobre un núcleo magnético. Este
se encuentra suspendido por \textbf{rodamientos}, de forma que el movimiento y
las fuerzas producidas en la iteraccion entre las bobinas y el nucleo con los
campos magneticos, producidos por el estator, puedan ser utilizados. Ademas,
los rodamientos, cuando se encuentran en buenas condiciones, permiten minimizar
el roce entre el eje y soporte, maximizando la transferencia de potencia.

Estos tambien son conocidos como \textbf{rolineras}, y son un tipo de cojinete,
un elemento mecanico diseñado para reducir la friccion entre un eje y los
elementos conectados al mismo. Existen muchos tipos de rolineras, sin enbargo
su estructura se fundamenta en dos anillos concentricos, algun elemento rotativo
como pueden ser bolas o rodillos, una jaula que se encarga de mantener los
elementos de rodadura separados ademas de guiados y un lubricante o un sistema
de lubricacion.

Dado que son un elemento mecanico el cual es continuamente usado, sufre mucho
desgaste y es el elemento mas propenso a dañarse, según los estudios realizados
por ~\textcite{Kammermann} el 59\% de las fallas son causadas por los rodamientos
y asimismo su principal falla es el desgaste, de igual forma se presentan
otras fallas estructurales. Y por estas razones es fundamental tener bajo continuo
monitoreo este elemento, esto se suele hacer a traves de un analisis de vibracion.

\subsubsection{Análisis de Vibración}

La vibración, es el  movimiento periódico de un cuerpo o medio
elástico, alrededor de un punto de equilibrio. En general podemos decir que una
vibración es un caso especifico de una oscilación, cuando esta es de origen
mecánico.

Se conoce como análisis de vibración, al conjunto de técnicas que permiten
obtener información de un equipo, a partir de sus vibraciones. Es una de las
técnicas más usadas en el mantenimiento predictivo de equipos mecánicos, debido
a que es un proceso poco invasivo y de bajo costo. El análisis de vibración
permite diagnosticar fallas de forma temprana, así como detectar señales
prematuras de desgaste.

Según Girdhar (2004) el análisis de vibración nos permite detectar las
siguientes fallas:

\begin{itemize}
\item{Defectos en los engranajes}
\item{Defectos en los rodamientos}
\item{Desalineamientos}
\item{Desbalances}
\item{Ejes torcidos}
\item{Excentricidad}
\item{Fallas eléctricas}
\item{Fuerzas hidráulicas o aerodinámicas}
\item{Mala sujeción en las piezas}
\item{Problemas en las correas de transmisión}
\item{Problemas de lubricación}
\item{Resonancia}
\item{Rozamientos en el rotor}
\end{itemize}

\subsubsection*{Adquisición de datos}

Para realizar cualquier tipo de análisis, primero se deben adquirir datos, y
por regla general mientras mas datos se tengan y mas precisas sean las
mediciones, mejores serán los estudios que se pueden realizar.

La toma de datos comienza con el sensor, que es el dispositivo que transforma
la unidad física de interés, en una señal que pueda ser procesada con mayor
facilidad. En el caso de las vibraciones, los sensores mas usados son los
acelerometros.  Esto se debe principalmente a que tienen mayor ancho de banda
que los sensores de posición y los de velocidad, lo que les permite detectar
vibraciones de mayor frecuencia. Adicionalmente, las vibraciones en los
rodamientos por naturaleza son de alta frecuencia, y debido a que estos son
componentes críticos en los motores eléctricos; hace casi exclusivo el uso de
acelerometros en las mediciones de vibración en motores eléctrico. Sin
embargo, en casos mas especializados, como podría ser el análisis de vibración
de maquinaria de larga envergadura son también usados los otros tipos de
sensores.

\subsubsection{Acelerometros}

Se conoce como acelerometro a los dispositivos capaces de medir aceleración. Su
funcionamiento proviene de la segunda ley de Newton, que dice que fuerza
ejercida sobre un cuerpo es proporcional a su masa y a su aceleración. Los
acelerometros por lo general cuentan con una masa de prueba (también conocida
como masa sísmica), alguna especie de resorte y un marco de soporte, que a su
vez puede funcionar como amortiguador. Debido a esto, los acelerometros se
pueden modelar como un sistema lineal de segundo orden, por lo que su respuesta
en frecuencia posee un pico de resonancia.

\subsubsection*{Tipos de acelerometros}

\begin{itemize}
    \item  Acelerometros Capacitivos:

Los acelerometros capacitivos son uno de los modelos mas sencillos de medir
aceleración. Cuando una masa sísmica experimenta una fuerza, se desplaza a una
aceleración proporcional a la fuerza aplicada. Si a su vez a la masa se le
agrega unos resortes unidos a una carcasa, los resortes ejercerán una fuerza
proporcional al desplazamiento de la masa. Estos efectos juntos al
amortiguamiento producen un sistema lineal de segundo orden, el cual podemos
ver como un sistema que al aplicarse una fuerza de entrada produce un
desplazamiento de salida. Finalmente, si conectamos un sensor de
desplazamiento, podemos medir la aceleración. El sensor de desplazamiento mas
popular para este tipo de medidores es el capacitivo.

En su mayoría los acelerómetros capacitivos vienen como dispositivos
microelectromecánicos, mems por sus siglas en ingles, que son dispositivos
fabricados con técnicas similares a la de la fabricación de circuitos
integrados, pero a su vez poseen componentes mecánicos microscópicos. Son los
acelerómetros mas populares en aplicaciones no industriales, pero en la
industria son preferidos los acelerómetros piezorresistivos y los piezoeléctricos.
Aunque en aplicaciones de bajo consumo, bajo coste o cuando las frecuencias con
las que se trabajan no son tan altas, pueden ser usados en aplicaciones
industriales, ya que poseen múltiples ventajas como que no requieren de una
unidad de adecuación de la señal y pueden comunicarse directamente con un
microcontrolador; son bastante económicos, ya que pueden ser producidos en
masa con alta facilidad y son bastante compactos.

\item  Acelerometros piezoresistivos:

Son después de los acelerometros pizoelectricos, los mas usados a nivel
industrial. Están constituidos por una o varias galgas extensiométricas, una
masa de prueba y unos resortes de soporte. La galga sujeta a la masa de
sísmica, y al esta recibir una fuerza produce un desplazamiento proporcional a
la fuerza aplicada, lo que deforma a su vez la galga extensiométrica y
finalmente esto se traduce como un cambio de resistencia del sensor. La ventaja
de los acelerometros piezoresistivos es que pueden medir valores DC lo que los
hace útil en el estudio de impactos, sin embargo, son también usados en el
analisis de vibración en el rango de mediana frecuencia.

\item Acelerometros piezoelectricos:

Son el acelerometro mas usado en aplicaciones industriales y poseen alto rango
dinámico, bajos niveles de ruido, alta linealidad, alto ancho de banda y al no
poseer partes móviles poseen poco desgaste. Su construcción es bastante
sencilla, se tiene disco de un cristal pizoelectrico unido por dos terminales
circulares, de forma similar a la de un condensador, y justo encima se tiene
una masa de prueba. Al experimentar una fuerza el cristal se deforma lo que
produce una diferencia de carga y un voltaje proporcional a la fuerza aplicada.
Como la aceleración de la masa de prueba es también proporcional a la fuerza
aplicada, la aceleración del acelerometro sera entonces directamente
proporcional al voltaje y la carga producida en el cristal.

\end{itemize}

\subsubsection{Procesamiento de señales}

Una vez recolectado los datos el siguiente paso será análisis de los mismos.
Existen una gran variedad de técnicas, y la elección de una dependerá en mayor
medida de lo que se desea diagnosticar. Los análisis se clasificaran en dos
categorías principales: análisis de frecuencia y análisis en el dominio del
tiempo.

El análisis en frecuencia es el mas popular de los dos, esto se debe a
que la mayoría de las fallas poseen frecuencias características, y como el
análisis de frecuencia permite separar las frecuencias, esto facilita su
detección. Sin embargo, cuando las frecuencias son bajas o la frecuencia que se
quiere detectar se encuentra cerca de otro frecuencia, es muy difícil
identificar la señal en el dominio de la frecuencia. En estos casos es mejor
usar un análisis en el dominio del tiempo, ya que son mucho mas sencillas de
detectar las fallas.

\subsubsection{Herramienta Computacional}

\subsubsection{Sistema Web}

\subsubsection*{Página web}

Una pagina web es un tipo de documento que usa en el protocolo HTTP, lo que uno
conoce usualmente como la web. A groso modo una pagina web tiene contenido y
permite redigirir o obtener nuevo contenido a partir de una direccion web. El
protocolo comenzo siendo estatico y dirigido usualmente a la transmicion de
texto pero poco a poco se fue convirtiendo en mas que eso. Es facil decir que
HTTP es el protocolo mas usado en todo el mundo, tanto que es usado como
sinonimo de internet cuando es solo una parte de el.

Debido a la invencion de tecnologias como javascrit y AJAX hoy en dia es
posible tener aplicaciones web, que son programas junto a una interfaz grafica,
que permite comunicarse con servidores que realizan la mayor parte del trabajo.
Debido a la popularidad de la web, asi mismo como su universalidad de la
misma. Es posible el desarrollo de aplicaciones complejas que funcione desde la
comodidad de dispotivos moviles. La web permite por tanto facilidad a la hora
de transmitir informacion asi como poder acceder a cualquier contenido desde
cualquier dispositivo a cualquier momento.

La web suele ser el metodo de acceso de mucha tecnologia y si bien el
desarrollo web usa el mismo estandar de tecnologias. En el lado del servidor
las tecnologias son mucho mas amplias. Cualquier aplicacion que pueda correr
en un ordenador puede ser conectada a una interfaz web. Teniendo una solo
limitante que es la latencia y si bien los protocolos estan avanzado para hacer
esto un problema del pasado seguira siendo cierto que la comunicacion con el
servidor no es instantanea y nunca lo sera.

Pero aun asi, una interfaz web para un usario promedio es una solucion comoda y
accesible y permite mayor libertad al usuario.

\subsubsection*{Servidores}

Un servidor web es un ordenador de proposito especifico que permite la
transmicion de datos a uno o multiples clientes web.

\subsubsection{Modelo estadístico}

Un modelo estadistico es un modelo matematico que posee variables aleatorias,
es decir posee una o mas variables, de las cuales no se tiene completa certeza
de su valor o proviene de algun evento aleatorio.

Un modelo estadistico permite inferir ciertas caracteristicas de un evento,
como que tan probable es tal evento y como se distribuye los valores de la
variable.

Es un primer paso en generar un modelo mas precioso o nos permite obtener
informacion cuando no se tiene suficiente informacion, o la naturaleza del
sistema es extremadamente compleja y dicha tarea es simplemente imposible.
