\subsection{MARCO TEÓRICO}

Esta parte del capítulo expone el contenido teórico necesario para la
realización de este proyecto. Esto incluye informacion sobre los motores
electricos, El analisis de la vibracion, las herramientas computacionales, los
sistemas web y el modelado estadistico.

%https://es.wikipedia.org/wiki/Motor_el%C3%A9ctrico


\subsubsection{ Motores eléctricos}

Un motor electrico es un dispositivo que transforma energia electrica en
energia mecanica mediante la accion de campos magneticos, y estan compuestos
principalmente por un estator (parte fija) y un rotor (parte movil).
Existen dos familias principales de motores electricos las cuales, a su vez,
se subdividen en \textbf{motores de corriente alterna} como lo son los  motores
de induccion, sincrones, entre otros y los \textbf{motores de corriente continua}
como lo son los motores de escobillas,  sin escobillas, de iman permanente,
entre otros.

El motor mas usado en el sector industrial, es el motor de induccion, debido
principalmente a su bajo costo y al poco mantenimiento que requieren para estar
completamente operativo, esto en particular es debido a su sencillo diseño
,en comparacion al de otros motores de igual potencia, ya que requiere  menor
numero de componentes. Cabe resaltar que el segundo tipo de motor mas usado es
el motor de corriente
continua, debido a que su control, en terminos de torque y potencia, es mucho
mas sencillo para ciertos niveles de potencia.

Tambien existen los motores sincronos, cuya construccion es similar a la de un
motor de induccion pero a su vez requiere de una fuente de alimentacion externa
para el rotor, lo cual aumenta su costo, ademas de esto su velocidad es
constante, por lo tanto este tipo de motor es utilizado en aplicaciones
especificas. Como se menciono anteriormente, existen otros tipos de motores,
pero esto son usualmente de menor potencia por
lo tanto su utlidad industrial es mucho mas limitada.


\subsubsection*{Rodamientos}

Para que un motor pueda llevar a cabo la transformacion de potencia debe rotar.
Esta accion es llevada a cabo por el \textbf{rotor}, el cual esta formado por un
eje que soporta un juego de bobinas envueltas sobre un núcleo magnético. Este
se encuentra suspendido por \textbf{rodamientos}, de forma que el movimiento y
las fuerzas producidas en la iteraccion entre las bobinas y el nucleo con los
campos magneticos, producidos por el estator, puedan ser utilizados. Ademas,
los rodamientos, cuando se encuentran en buenas condiciones, permiten minimizar
el roce entre el eje y soporte, maximizando la transferencia de potencia.

Estos tambien son conocidos como \textbf{rolineras}, y son un tipo de cojinete,
un elemento mecanico diseñado para reducir la friccion entre un eje y los
elementos conectados al mismo. Existen muchos tipos de rolineras, sin enbargo
su estructura se fundamenta en dos anillos concentricos, algun elemento rotativo
como pueden ser bolas o rodillos, una jaula que se encarga de mantener los
elementos de rodadura separados ademas de guiados y un lubricante o un sistema
de lubricacion.

Dado que son un elemento mecanico el cual es continuamente usado, sufre mucho
desgaste y es el elemento mas propenso a dañarse, según los estudios realizados
por ~\textcite{Kammermann} el 59\% de las fallas son causadas por los rodamientos
y asimismo su principal falla es el desgaste, de igual forma se presentan
otras fallas estructurales. Y por estas razones es fundamental tener bajo continuo
monitoreo este elemento, esto se suele hacer a traves de un analisis de vibracion.


\subsubsection{Análisis de Vibración}

La vibración, es el  movimiento periódico de un cuerpo o medio
elástico, alrededor de un punto de equilibrio. En general podemos decir que una
vibración es un caso especifico de una oscilación, cuando esta es de origen
mecánico.

Asimismo, se conoce como análisis de vibración, al conjunto de técnicas que permiten
obtener información de un equipo, a partir de sus vibraciones. Es una de las
técnicas más usadas en el mantenimiento predictivo de equipos mecánicos, debido
a que es un proceso poco invasivo y de bajo costo. El análisis de vibración
permite diagnosticar fallas de forma temprana, así como detectar señales
prematuras de desgaste.

Según Girdhar \Cite{Girdhar}  el análisis de vibración nos permite detectar las
siguientes fallas:

\begin{itemize}
\item Defectos en los engranajes
\item Defectos en los rodamientos
\item Desalineamientos
\item Desbalances
\item Ejes torcidos
\item Excentricidad
\item Fallas eléctricas
\item Fuerzas hidráulicas o aerodinámicas
\item Mala sujeción en las piezas
\item Problemas en las correas de transmisión
\item Problemas de lubricación
\item Resonancia
\item Rozamientos en el rotor
\end{itemize}


\subsubsection*{Adquisición de datos}

Para realizar cualquier tipo de análisis, primero se deben adquirir datos, y
por regla general mientras mas datos se tengan y mas precisas sean las
mediciones, mejores serán los estudios que se pueden realizar.

La adquisicion de datos es el proceso de realizar mediciones de fenomenos fisicos
y registrarlos, en algun formato especifico, para analizarlos posteriormente.
Cabe resaltar que en algunos casos es necesario hacer un acondicionamiento de
la señal, esto
implica modificarla controladamente al reducir o aumentar su amplitud ademas de
sumarle un nivel offset, voltaje continuo, para que cumpla requisitos minimos
y pueda ser procesada por los siguientes circuitos.

Al momento de la adquisicion de datos, se toman medidas de una señal analogica
la cual se convertira al pasar por una serie de etapas y dispositivos en una
señal digital y se guardara en un formato deseado y en una unidad de
almacenamiento masivo (Rom, flash, etc).
El primer paso en la toma de  datos comienza con el sensor, que es un
dispositivo el cual transforma la unidad física de interés, en una señal que
pueda ser procesada con mayor facilidad, luego esta se adecuara mediante un
circuito especializado a las caracteristicas y requerimientos del sistema,
luego sera procesada por un convertidor Analógico-Digital, el cual se encarga de
muestrear, reterner y procesar la señal, y de esta forma obtener una version
digital de la misma, la cual se procesara o guardara mediante un software desde
una computadora.

En el caso de las vibraciones, los sensores mas usados son los
acelerometros.  Esto se debe principalmente a que tienen mayor ancho de banda
que los sensores de posición y los de velocidad, lo que les permite detectar
vibraciones de mayor frecuencia. Adicionalmente, dado que las vibraciones en los
rodamientos por naturaleza son de alta frecuencia, y como se menciono anteriormente,
estos son componentes críticos en los motores eléctricos.
Estas caracteristicas hacen al acelerometro el sensor mas utilizado para las
mediciones de vibración en motores eléctrico. Sin embargo,
en casos mas especializados, como podría ser el análisis de vibración
de maquinaria de larga envergadura son también usados los otros tipos de
sensores ya que sus caracteristicas pueden facilitar el estudio.


\subsubsection{Acelerometros}

Un acelerometro es un dispositivo capaz de medir aceleración, no es
necesariamente la misma que la aceleracion de coordenadas (cambio de la velocidad de
un elemento en el espacio), sino que es la acerleracion asociada con el fenómeno
de peso experimentado por una masa de prueba que se
encuentra en el marco de referencia del dispositivo.  Funciona mediante
la utilizacion de la segunda ley de Newton, \textbf{la fuerza resultante
ejercida sobre un cuerpo es proporcional a su masa por su aceleración}. Los
acelerometros por lo general cuentan con una masa de prueba (también conocida
como masa sísmica), alguna especie de resorte y un marco de soporte, que a su
vez puede funcionar como amortiguador. Debido a esto, los acelerometros se
pueden modelar matematicamente como un sistema lineal de segundo orden,
por lo que su respuesta en frecuencia posee un pico de resonancia.


\subsubsection*{Tipos de acelerometros}

\begin{itemize}
    \item  Acelerometros Capacitivos:

        Los acelerometros capacitivos son uno de los modelos mas sencillos
        capaces de medir
        aceleración, ademas de ser facilmente utilizables y reproducibles en masa.
        Funcionan mediante una masa sísmica, dado que esta, al  experimentar una fuerza
        se desplaza con una aceleración proporcional a la fuerza aplicada.
        Si a la masa se le agregan unos resortes unidos a una carcasa, los resortes
        ejercerán una fuerza proporcional al desplazamiento de la masa generando un
        desplazamiento facilmente medible.

        Este fenomeno se produce ya que estos efectos en conjunto al
        amortiguamiento producen un sistema lineal de
        segundo orden, este sistema matematico genera una salida en desplazamiento
        al aplicarse como entrada una fuerza
         Finalmente, si al conectarse un sensor de desplazamiento, se observa
        la aceleración a la que esta sometido el instrumento.
        El sensor de desplazamiento mas
        popular para este tipo de medidores es el capacitivo.

        En su mayoría los acelerómetros capacitivos vienen como dispositivos
        microelectromecánicos, \textbf{mems} por sus siglas en ingles, que son
        dispositivos
        fabricados con técnicas similares a la de la fabricación de circuitos
        integrados, pero a su vez poseen componentes mecánicos microscópicos. Son los
        acelerómetros mas populares en aplicaciones no industriales, sin embargo,
        en aplicaciones de bajo consumo, bajo coste o cuando las frecuencias con
        las que se trabajan no son tan altas, pueden ser usados a niveles
        industriales, ya que poseen múltiples ventajas como:

        \begin{itemize}
            \item No requieren adecuación de la señal.
            \item Pueden comunicarse directamente con un microcontrolador.
            \item Su precio es economico.
            \item Son compactos.
        \end{itemize}


    \item  Acelerometros piezoresistivos:

        Son después de los acelerometros pizoelectricos, los mas usados a nivel
        industrial. Su funcionamiento es similar al de los acelerometros
        capacitivos, ante una aceleracion de entrada se produce un desplazamiento
        de salida, ma en este caso, estos están constituidos por una o varias
        galgas extensiométricas, una masa de prueba y unos resortes de soporte.
        La galga sujeta a la masa sísmica, y al esta recibir una fuerza produce
        un desplazamiento proporcional a la fuerza aplicada, lo que deforma a
        su vez la galga extensiométrica y finalmente esto se traduce como un
        cambio de resistencia en el sensor. La ventaja de los acelerometros
        piezoresistivos es que pueden medir valores de voltaje DC lo que los
        hace útil en el estudio de impactos, sin embargo, son también usados en el
        analisis de vibración en el rango de mediana frecuencia.


    \item Acelerometros piezoelectricos:

        Son el acelerometro mas usado en aplicaciones industriales ya que
        poseen caracteristicas como:

        \begin{itemize}
            \item Alto rango dinamico.
            \item Bajos niveles de ruido.
            \item Alta linealidad.
            \item Alto ancho de banda.
            \item Poco desgaste, ya que no poseen partes moviles.
        \end{itemize}


        Su construcción es bastante sencilla, se tiene disco de un cristal
        pizoelectrico unido por dos terminales circulares, de forma similar a
        la de un condensador, y justo encima se tiene una masa de prueba.
        Al experimentar una fuerza el cristal se deforma lo que produce una
        diferencia de carga y un voltaje proporcional a la fuerza aplicada.
        Como la aceleración de la masa de prueba es también proporcional a la
        fuerza aplicada, la aceleración del acelerometro sera entonces
        directamente proporcional al voltaje y la carga producida en el cristal.

\end{itemize}


\subsubsection{Procesamiento de señales}

Despues de ser almacenada la informacion, debe ser estudiada, procesada, para lo
cual se utilizan una serie de herramientas, tecnicas o softwares especializados
a cada necesidad. Este estudio se puede categorizar en dos ramas principales:

\begin{itemize}
    \item Dominio del tiempo, termino utilizado para describir el analisis
        de funciones matematicas o señales con respecto al tiempo, la sucesion
        de estados que atraviesa la señal de forma natural, y los estudios mas
        comunes son en  \textbf{tiempo continuo} y en \textbf{tiempo discreto}.

    \item Dominio de la frecuencia, termino utilizado para describir el analisis
        de funciones matematicas, señales o movimientos periodicos respecto a
        su frecuencia, numero de veces que sucede un evento en un periodo, se
        utilizan transformadas para llevar las funciones o señales del dominio
        del tiempo, base, al dominio de la frecuencia, deseado, la mas famosa es
        la \textbf{transformada de fourier}.
\end{itemize}


Graficamente se suelen entender el \textbf{dominio temporal} como la evolucion
de una señal con respecto al tiempo, es decir su evolucion natural, por otro
lado el \textbf{dominio frecuencial} muestra los componentes de la señal segun
la frecuencia en la que oscilan dentro de un rango determinado.

%  grafica de las señales??


El análisis en frecuencia suele ser mas utilizado debido a que
la mayoría de las fallas poseen frecuencias características, y dado que en  el
análisis de frecuencia se descompone en frecuencias la señal, se facilita la
detección de fallas caracteristicas y asi mismo, la amplitud de la frecuencia es
directamente proporcional al nivel de la falla. Por lo tanto se obtiene un
espectro amplio del estado de la pieza.
Cabe resaltar que cuando las frecuencias son bajas o muy cercanas entre si,
se dificulta determinar e identificar alguna falla, suele suceder cuando se
estudia una falla o evento con frecuencia muy baja o muy cercana a la frecuencia
natural de la señal o elemento medido. En estos casos es mejor
usar un análisis en el dominio del tiempo, ya que se facilita la
deteccion de las fallas.



\subsubsection{Herramienta Computacional}


Una herramienta computacional puede ser definida como cualquier software,
sistema de integracion, analisis o almacenamiento el cual ayuda a los cientificos
o usuarios a solucionar un problema especifico en una determinada rama. Entre
estos pueden variar desde sistemas complejos como compiladores, algoritmos
e incluso sistemas operativos hasta herramientas como hojas de calculos, sistemas
de oficina o medios de comunicacion. Funcionan mediante la implementacion de
tecnicas y protocolos para solucionar problemas de forma iterativa o con una
secuencia de pasos concreta.

Siguiendo este orden de ideas, una gran cantidad de estas herramientas son
encontradas en la libreria de informacion mas grande del mundo, el internet.
Todas estas comparten la peculiaridad de que son un \textbf{sistema web} y por
ende pueden
ser accedidas con facilidad desde cualquier punto con un dispositivo capaz de
tener conexion a internet y un navegador web. Esta facilidad se debe a que un
\textbf{servidor} se encarga de hacer el procesamiento de la informacion y envia
el resultado con un formato especifico, tipicamente json, el cual se renderiza
en una pagina web.

\subsubsection{Sistema Web}


\subsubsection*{Página web}

Una pagina web es un tipo de documento que usa en el protocolo HTTP, lo que uno
conoce usualmente como la web. A groso modo una pagina web tiene contenido y
permite redigirir o obtener nuevo contenido a partir de una direccion web. El
protocolo comenzo siendo estatico y dirigido usualmente a la transmicion de
texto pero poco a poco se fue convirtiendo en mas que eso. Es facil decir que
HTTP es el protocolo mas usado en todo el mundo, tanto que es usado como
sinonimo de internet cuando es solo una parte de el.

Debido a la invencion de tecnologias como javascrit y AJAX hoy en dia es
posible tener aplicaciones web, que son programas junto a una interfaz grafica,
que permite comunicarse con servidores que realizan la mayor parte del trabajo.
Debido a la popularidad de la web, asi mismo como su universalidad de la
misma. Es posible el desarrollo de aplicaciones complejas que funcione desde la
comodidad de dispotivos moviles. La web permite por tanto facilidad a la hora
de transmitir informacion asi como poder acceder a cualquier contenido desde
cualquier dispositivo a cualquier momento.

La web suele ser el metodo de acceso de mucha tecnologia y si bien el
desarrollo web usa el mismo estandar de tecnologias. En el lado del servidor
las tecnologias son mucho mas amplias. Cualquier aplicacion que pueda correr
en un ordenador puede ser conectada a una interfaz web. Teniendo una solo
limitante que es la latencia y si bien los protocolos estan avanzado para hacer
esto un problema del pasado seguira siendo cierto que la comunicacion con el
servidor no es instantanea y nunca lo sera.

Pero aun asi, una interfaz web para un usario promedio es una solucion comoda y
accesible y permite mayor libertad al usuario.


\subsubsection*{Servidores}

Un servidor web es un ordenador de proposito especifico que permite la
transmicion de datos a uno o multiples clientes web.


\subsubsection{Modelo estadístico}

Un modelo estadistico es un modelo matematico que posee variables aleatorias,
es decir posee una o mas variables, de las cuales no se tiene completa certeza
de su valor o proviene de algun evento aleatorio.

Un modelo estadistico permite inferir ciertas caracteristicas de un evento,
como que tan probable es tal evento y como se distribuye los valores de la
variable.

Es un primer paso en generar un modelo mas precioso o nos permite obtener
informacion cuando no se tiene suficiente informacion, o la naturaleza del
sistema es extremadamente compleja y dicha tarea es simplemente imposible.
