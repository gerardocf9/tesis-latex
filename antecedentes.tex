katalin Agoston hace un analisis sobre la deteccion y el diagnostico de fallas
en motores electricos mediante analisis de vibracion, explicando cuales son los
origenes mas comunes de las vibraciones en los motores, asi como cual es la
mejor forma de colocar los acelerometros para detectar las mismas.  El estudio
finaliza con un explicacion de un modelo matematico Sobre las vibraciones en
los motores electricos.

[Articulo](https://reader.elsevier.com/reader/sd/pii/S2212017315000791?token=03845C7CFDEE24C12FEBFC86A82AC205588D9FC41E06F6BF53D406F917818026D854DC781AC81A7E35D9BD96F87A746B&originRegion=us-east-1&originCreation=20210422192313)

Ivar Koene, Ville Klar, Raine Viitale en una publicacion de HardwareX hablan
sobre un sensor de vibracion "Open Source" de bajo costo llamado Memsio. En el
articulo explican que en la industria los sensores de vibracion mas usados son
los sensores pizoelectricos, su popularidad viene dada por su mayor tolerancia
al ruido y en algunos casos mayor precision a otras dispositivos, sin embargo
debido a los avances en los dispositivos MEMS, los autores explican que no solo
las diferencias en la calidad de los dos sensores es cada vez menor, sino que
debido su bajo costo y la facil integracion con microcontroladores hacen
posible desarrollar dispositivos de relativo bajo costo.

Mensio, puede ser desarrollado con un costo menor a 80$ y todo la informacion
incluyendo esquematicos y firmware es completamente open source, tiene un
interfaz web y su carcasa puede ser desarrolla con materiales de bajo costo,
pudiendo ser desarrollado con impresion 3D.

[Mensio](https://reader.elsevier.com/reader/sd/pii/S2468067220300171?token=18A062DB10415F47528579BAD2C55C9D1C2F28D7B564AC91FEF71226ABAF173B806F07773498DCA7FC60574426CF9B1A&originRegion=us-east-1&originCreation=20210422194048)