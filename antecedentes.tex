\thispagestyle{empty}

	\section{REVISIÓN BIBLIOGRÁFICA}


\subsection{ANTECEDENTES}

\textcite{Pinto} realizan una simulación de redes de sensores inalámbricos, con el fin de mejorar la detección de fallas de los sensores. Esta simulación fue realizada con Castalia un simulador de redes de sensores inalámbricos y el sensor usado fue un detector de luminosidad. Algunas de las razones que menciona el artículo de por qué fue escogida una simulación es debido a que realizar el experimento con sensores requiere un costo elevado de hardware y un estudio analítico no es efectivo, ya que la complejidad del sistema es muy grande. Este trabajo es de utilidad debido a que sirve como orientación a la elaboración de la simulación del sensor que generara el modelo estadistico.\\


\textcite{Ugwiri} Presentan un resumen de las técnicas más usadas actualmente en la detección de fallas en motores eléctricos mediante análisis de vibración, además de realizar un experimento donde se ponen a prueba algunos de estos conceptos. El propósito de este antecedente es el de apoyar la elaboración del modo de "vista exhaustiva".\\


\textcite{Koene} Elaboran un sensor de vibración inalámbrico de código libre llamado Memsio. Este dispositivo es alimentado por baterías y permite la adquisición de datos a alta velocidad por medio del uso de un acelerómetro microelectromecánico. Los autores mencionan que en la industria los sensores de aceleración más usados son los piezoeléctricos, debido a tener una mayor precisión y tolerancia al ruido, sin embargo los avances de los dispositivos microelectromecánico y su bajo costo hacen cada vez más factibles su uso para el monitoreo. Este antecedente muestra la tendencia de la reducción de precios de los sensores inalámbricos lo cual apoya al propósito del trabajo al hacer factible las redes de sensores.
