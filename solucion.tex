
%\subsection{solución}

Para solucionar la problemática planteada, se propone el desarrollo de un sistema capaz de tomar datos de forma continua y enviarlos  a un servidor el cual permita el almacenar, estudiar y muestrear la información en distintos niveles de profundidad, con respecto al análisis realizado.\\


Dadas las dificultades de medición expuestas y la carencia de datos continuos para el análisis exhaustivo, la solución propuesta es un sistema digital, diagramado en la Figura \ref{diagrama}, el cual sirve como base y prototipo ante  una posible automatización, al ser este un sistema modular permite la utilización del software-herramienta en combinación con otros proyectos (como lo puede ser el usar hardware libre Memsio de \textcite{Koene} como fuente de datos al servidor) con muy pocas modificaciones.\\


Para esto, se dispone de una base de datos (BBDD externa), proporcionada por un tercero (empresa dedicada al estudio de la vibración), la cual dará pie a un análisis estadístico para la obtención de un modelo del compartimento de un motor eléctrico con distintos niveles de daño y la salida correspondiente dada por un acelerómetro digital. De esta forma se alimentar el servidor con datos proporcionados por el modelo y un valor semilla (características del motor, grado de daño, etc) los cuales se almacenaran en una nueva base de datos (BBDD) para, posteriormente, ser procesada y  generar 3 niveles de análisis: \\

%un modelo de la salida de un acelerómetro digital dado el compartimento de un motor eléctrico con distintos niveles de daño.

\begin{itemize}
\item La vista principal, permitirá observar una cantidad específica de motores, simbolizando los existentes en una planta o piso, y su estado general.

\item La vista específica, dará la información actual e histórica referente a un único motor previamente seleccionado.

\item La vista exhaustiva se refiere a un análisis en frecuencia de la vibración de un motor especificado con anterioridad, con la finalidad de permitir al operador o ingeniero encargado determinar las posibles averías y sus causas.
\end{itemize}


Y finalmente, toda esta información y opciones se mostrarían a través de una página Web para facilitar su acceso.\\



\begin{figure}[htb]
\centering
\caption{Diagrama de la simulación digital.}
\label{diagrama}
\includegraphics[width=15cm, height=23cm]{Diagrama_sensorica.png}
\end{figure}
