\thispagestyle{empty}

\section{MARCO METODOLÓGICO}

Como ha sido explicado en el primer capitulo, se pretende hacer una herramienta
computacional para el análisis de la vibración en motores eléctricos alimentada
mediante datos de una simulación digital. Partiendo de este hecho, se puede
comenzar con la definición de los siguientes aspectos.

\subsection{NATURALEZA DE LA INVESTIGACIÓN}

El presente trabajo es clasificado como Proyecto Especial, puesto que según
\textcite{Hernandez} lleva a:

\begin{center}
    \parbox[ht]{13.5 cm}{Trabajos que lleven a creaciones tangibles,
    susceptibles de ser utilizadas como soluciones a problemas demostrados, o
    que respondan a necesidades e intereses de tipo cultural. Se incluyen en
    esta categoría los trabajos de elaboración de libros de texto y de
    materiales de apoyo educativo, el desarrollo de software, prototipos y de
    productos tecnológicos en general, así como también los de creación
    literaria y artística.}
\end{center}
