\thispagestyle{empty}

\section{MARCO METODOLÓGICO}

Como fue explicado en el primer capitulo, se pretende hacer una herramienta
computacional para el análisis de la vibración en motores eléctricos alimentada
mediante datos de una simulación digital. Partiendo de esto, se comenzará
con la definición de los siguientes aspectos.

\subsection{NATURALEZA DE LA INVESTIGACIÓN}

El presente trabajo es clasificado como Proyecto Especial, puesto que según
\textcite{Hernandez} lleva~a:

\begin{center}
    \parbox[ht]{13.5 cm}{Trabajos que lleven a creaciones tangibles,
    susceptibles de ser utilizadas como soluciones a problemas demostrados, o
    que respondan a necesidades e intereses de tipo cultural. Se incluyen en
    esta categoría los trabajos de elaboración de libros de texto y de
    materiales de apoyo educativo, el desarrollo de software, prototipos y de
    productos tecnológicos en general, así como también los de creación
    literaria y artística.}
\end{center}

\subsection*{METODOLOGÍA DE DESARROLLO DEL SOFTWARE}

Para la implementación de los distintos sistemas que componen la herramienta
computacional propuesta, se necesitar seguir una metodología de diseño de
software. Como la herramienta se clasifica como una aplicación WEB, las
metodologías utilizadas en el desarrollo de este tipo aplicaciones son
aplicables a este trabajo.

Las aplicaciones WEB cuentan con muchas similitudes a las de aplicaciones de
computadores personales, sin embargo una de sus principales diferencias es que
estas están en un estado de constante cambio por lo tanto
se pierde la noción de versiones, los cambios toman efectos de forma inmediata
y de forma gradual. Esta diferencia es producto de que su función principal es
la de transmitir información. Como está información cambia de forma constante
los requisitos de la aplicación suelen también ser variables. Todo esto unido
al hecho de que los cambios en la aplicación pueden realizarse con la
menor fricción posible, los cambios se pueden observar al volver a cargar la
página, hace que este tipo de aplicación  exista en un estado de evolución
continua. Según \cite{pressman2002} la evolución constante de las aplicaciones
WEB puede ser comparada con la jardinería, se hace un trabajo inicial el cual
seria equivalente a sembrar un jardín y una vez se tenga el sitio implementado,
se debe realizar el trabajo de mantenerlo que seria equivalente a regar y
abonar las plantas.

Debido a lo presentado anteriormente, para le elaboración de los diferentes
componentes de la herramienta se realizará un desarrollo de forma continua,
en donde la aplicación tendrá la capacidad de evolucionar si algún día
cambian los requisitos de la misma. Para esto se utilizara la ayuda de
\textbf{Git} como sistema de control de versiones, lo cual permitiría integrar
los diferentes cambios de forma mas eficaz y facilitar la cooperación a la
hora del desarrollo de la aplicación. Para la elaboración de los diferentes
componentes se usa también un proceso iterativo, cada vez que se implemente
un componente se comprueba el funcionamiento del mismo, y se corrige de ser
necesario.

Este modelo de implementación posee una estructura muy marcada, la cual se adapta
a las metas, usuarios finales y a la filosofía de navegación que se elija.
Según \cite{pressman2002} esta se puede desglosar como:

\begin{itemize}
    \item Diseño arquitectónico, el cual consiste en la definición de la
        estructura global, las plantillas y parte del patrón de diseño que se
        utilizara para estructurar la red. Específicamente, se trata de cimentar
        las bases para facilitar la creación del contenido y se estructura la
        forma de navegación por el sistema, además de los componentes que la
        integrarán.

    \item Diseño de navegación, en esta fase se eligen las rutas de navegación
        que permiten el acceso al contenido, además, se hace las distinciones
        con respecto a los posibles usuarios que podrán acceder a la red y
        por ende, las funcionalidades y/o permisos a los que tendrán acceso.

    \item Diseño de la interfaz, se refiere al diseño gráfico, estético, y a
        las facilidades que se le dan a los usuarios, dado que esta es la primera
        impresión que se da puede significar la diferencia entre el uso o no de
        la aplicación.

    \item Cabe destacar que en la actualidad, se consideran mas fases de las
        mencionadas por Pressman, las dos mas resaltantes son la
        \textbf{Fase de programación} y la \textbf{Fase de Testeo}, aunque la
        ultima es nombrada como un proceso posterior del desarrollo. Estas dos
        fases son consideradas en la actualidad dados los constantes requerimientos
        de implementaciones adicionales o modificaciones a las ya existentes.
        Para permitir la integración de la mismas sin exponerse al colapso del
        sistema, se suelen utilizar \textbf{test} automatizados los cuales deben
        ser diseñados e implementados.

\end{itemize}
