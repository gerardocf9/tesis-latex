\thispagestyle{empty}

\section{MARCO METODOLÓGICO}

Como fue explicado en el primer capitulo, se pretende hacer una herramienta
computacional para el análisis de la vibración en motores eléctricos alimentada
mediante datos de una simulación digital. Partiendo de esto, se comenzará
con la definición de los siguientes aspectos.

\subsection{NATURALEZA DE LA INVESTIGACIÓN}

El presente trabajo es clasificado como Proyecto Especial, puesto que según
\textcite{Hernandez} lleva~a:

\begin{center}
    \parbox[ht]{13.5 cm}{Trabajos que lleven a creaciones tangibles,
    susceptibles de ser utilizadas como soluciones a problemas demostrados, o
    que respondan a necesidades e intereses de tipo cultural. Se incluyen en
    esta categoría los trabajos de elaboración de libros de texto y de
    materiales de apoyo educativo, el desarrollo de software, prototipos y de
    productos tecnológicos en general, así como también los de creación
    literaria y artística.}
\end{center}

\subsection*{METODOLOGÍA DE DESARROLLO DEL SOFTWARE}

Para la implementación de los distintos sistemas que componen la herramienta
computacional propuesta, se necesitar seguir una metodología de diseño de
software. Como la herramienta se clasifica como una aplicación WEB, las
metodologías utilizadas en el desarrollo de este tipo aplicaciones son
aplicables a este trabajo.

Las aplicaciones WEB cuentan con muchas similitudes a las de aplicaciones de
computadores personales, sin embargo una de sus principales diferencias es que
estas están en un estado de constante cambio por lo tanto
se pierde la noción de versiones, los cambios toman efectos de forma inmediata
y de forma gradual. Esta diferencia es producto de que su función principal es
la de transmitir información. Como está información cambia de forma constante y
los requisitos de la aplicación suelen también ser variables. Todo esto unido
con el hecho de que los cambios en la aplicación pueden realizarse con la
menor fricción posible, los cambios se pueden observar al volver a cargar la
pagina, hace que este tipo de aplicación  exista en un estado de evolución
continua. Según \cite{pressman2002} la evolución constante de las aplicaciones
WEB puede ser comparada con la jardinería, se hace un trabajo inicial el cual
seria equivalente a sembrar un jardín y una vez se tenga el sitio implementado,
se debe realizar el trabajo de mantenerlo que seria equivalente a regar y
abonar las plantas.

Debido a lo presentado anteriormente, para le elaboración de los diferentes
componentes de la herramienta se realizara un desarrollo de forma continua,
en donde la aplicación tendrá la capacidad de evolucionar si algún día
cambian los requisitos de la misma. Para esto se utilizara la ayuda de
\textbf{Git} como sistema de control de versiones, lo cual permitiría integrar
los diferentes cambios de forma mas eficaz y facilitar la cooperación a la
hora del desarrollo de la aplicación. Para la elaboración de los diferentes
componentes se usara también un proceso iterativo, cada vez que se implemente
un componente se probara su funcionamiento y de encontrar una falla,
solucionar la falla y volver a probar hasta que no se consigan fallas que
afecten el funcionamiento del programa.

