\documentclass[12pt]{article}

%\usepackage{src/cap/fontspec}
%\setmainfont{src/cap/Times New Roman}

\usepackage{tocbibind}

\usepackage[a4paper]{geometry}
\geometry{top=2.54cm, bottom=2.54cm, left=3.2cm, right=2.54cm}

\usepackage[spanish]{babel}
\usepackage[utf8]{inputenc}
\usepackage[T1]{fontenc}
\usepackage[none]{hyphenat}

\hyphenation{mi-cro-elec-tro-me-cá-ni-cos}
\hyphenation{ace-le-ró-me-tros}
\hyphenation{ca-pa-ci-ti-vos}

\usepackage{float}
\usepackage{graphicx} % graficos
\graphicspath{ {./images/} }
\usepackage{setspace}%interlineado agrega:
% \doublespacing \onehalfspace \singlespace \spacing{src/cap/x}
\spacing{1.5}
\setlength{\parindent}{0pt}
\pagestyle{headings}

%\usepackage {src/cap/ natbib }
\setlength{\parskip}{12pt}

\usepackage[backend=biber, autocite=inline, labeldateparts=true, maxcitenames=1,style = apa]{biblatex}

\addbibresource{bibliografias.bib}
\addbibresource{referencias.bib}

\usepackage{csquotes}
\usepackage[bookmarks = true, colorlinks=true, linkcolor = black, citecolor = black, menucolor = black, urlcolor = black]{hyperref}


\usepackage{enumitem}%eliminar el espaciado entre items de lista

\usepackage{array}%centrar verticalmente en tablas
%****************************************************
%     secciones con formato especifico
\usepackage{titlesec}
\titleformat{\section}[display]
{\normalfont\bfseries\LARGE}{\filcenter CAPÍTULO \thesection}{3pt}{\filcenter \LARGE}


\newcommand{\comillas}[1]{``#1"}

%  \makeatletter
%     \renewcommand{\l@section}[2]{\@dottedtocline{2}{1.6em}{1.6em}{#1}{}}
%  \makeatother


\begin{document}
\sloppy
\renewcommand{\listfigurename}{Índice de Figuras}
\renewcommand{\listtablename}{Índice de Tablas}
\renewcommand{\tablename}{Tabla}
\pagenumbering{roman}
	\setcounter{page}{0}
\begin{titlepage}

\begin{table}[t]
\centering
\begin{tabular}{ p{3cm} p{8.5cm} p{3cm} }
	\begin{flushleft}\includegraphics[width=2.4cm]{logo_poli.png}\end{flushleft} &



	\begin{center}
	República Bolivariana de Venezuela\\
	Universidad Nacional Experimental Politécnica “Antonio José de Sucre”\\
	Vice Rectorado Barquisimeto \\
	Departamento de Ingeniería Electrónica\\

%***************************************************
%************** aqui va el titulo ******************
%***************************************************

	\vspace*{45mm}
	\begin{LARGE}Herramienta computacional para el análisis de la vibración en motores eléctricos alimentada mediante datos de una simulación digital\\\end{LARGE}

	\end{center}


	& \begin{flushright}\includegraphics[width=1.7cm]{logo_electronica.jpg} \end{flushright}
\end{tabular}


	\vspace*{3mm}

	\parbox[c]{12cm}{

	}



    \vspace*{19mm}



\begin{flushright}
Integrantes:\\


Gerardo Alfonzo Campos Fonseca\\
V. 27085179\\
José Andrés Cortez Terán\\
V. 26540824\\

\vspace*{2mm}
Tutor: Dra. Luisa Mercedes Escalona\\
Cotutor: Dr. Carlos Zambrano\\

\end{flushright}
\vspace*{7mm}

\begin{center}Barquisimeto, Marzo 2022\end{center}
\end{table}
\end{titlepage}


    \include{src/contraportada}
    \newpage
    \newpage
    
\addcontentsline{toc}{section}{DEDICATORIA}
%\setcounter{figure}{0}
\phantom{}
\vfill
\begin{center}
   \it A nuestras familias y amigos por siempre estar presentes.
\end{center}
\vfill
\phantom{}

    
\addcontentsline{toc}{section}{AGRADECIMIENTOS}
%\setcounter{figure}{0}
\section*{AGRADECIMIENTOS}

    Agradecemos primeramente a dios por darnos la vida y la  fuerza en los momentos
    más difíciles; a nuestros padres por criarnos, educarnos y siempre apoyarnos en
    nuestras metas; a nuestras familias por todo el amor, cariño y por ayudarnos y
    guiarnos en nuestra formación personal; a todas aquellas personas que ya no
    están con nosotros pero su recuerdo sigue en nuestras mentes y corazones; a
    nuestros amigos por darnos incontables buenos momentos y el escape de la rutina
    que muchas veces es necesario; a nuestros compañeros por su gran amistad, su
    apoyo incondicional y por ser personas que nos impulsan cada día a ser mejores
    en nuestra carrera; a nuestros profesores, sobre todo a los profesores del
    departamento de ingeniería electrónica por su gran vocación, su impecable ética
    de trabajo y su grandes conocimientos que no solo nos impulsan como
    profesionales sino también como  personas; a todo el personal universitario por
    su increíble labor en estos tiempos difíciles y finalmente pero no menos
    importante a nuestra universidad por brindarnos la oportunidad de demostrar
    nuestros conocimientos, de crecer y de formar parte de esta gran familia que
    siempre será parte de nuestros corazones.


	%\newpage
	\tableofcontents
	\newpage
	\listoffigures
	\newpage
	\listoftables

\begin{refsegment}
\thispagestyle{empty}
\addcontentsline{toc}{section}{RESUMEN}

    \begin{table}[t]
        \centering
        \begin{tabular}{ p{3cm} p{8.5cm} p{3cm} }
            \begin{flushleft}\includegraphics[width=2.4cm]{logo_poli.png}\end{flushleft} &

                \begin{center}
                    República Bolivariana de Venezuela\\
                    Universidad Nacional Experimental Politécnica “Antonio José de Sucre”\\
                    Vice Rectorado Barquisimeto \\
                    Departamento de Ingeniería Electrónica\\

                \end{center}

                & \begin{flushright}\includegraphics[width=1.7cm]{logo_electronica.jpg} \end{flushright}
        \end{tabular}

        \vspace*{-3mm}

\parbox[c]{15cm}{
    \begin{center}
        \textbf{Herramienta computacional para el análisis de la vibración en
        motores eléctricos alimentada mediante datos de una simulación
        digital\\}


        \vspace*{3mm}
        \textbf{Autores:} Gerardo Campos, José Cortez\\

        \textbf{Tutor:} Dra. Luisa Escalona,
        \textbf{Cotutor:} Dr. Carlos Zambrano\\
    \end{center}
}
    \end{table}


La presente investigación ha tenido como propósito el desarrollo de una
herramienta computacional para el análisis de la vibración en motores
eléctricos, haciendo uso de  datos obtenidos mediante una simulación
digital para verificar su funcionamiento y una metodología de desarrollo
Web enfocada a microservicios. Para lograrlo se desarrolló un modelo
estadístico, capaz de entregar las mediciones necesarias para la
implementación, a partir de una base de datos existente; así mismo se
crearon 3 microservicio para cubrir las necesidades de obtención de
información y comunicación con el modelo, almacenar la información
(database as a service DBaaS) y mostrar la información por medio de una
página Web; se desarrollaron estructuras de datos y se diseñó una base de
datos no relacional para el almacenamiento de la información, se automatizó
la interconexión entre el modelo estadístico, el servidor de adquisición de
información y el DBaaS mediante un cliente y un Script y se despliega todo
el servicio en una página Web con 3 vistas, General, específica,
exhaustiva, en los cuales se puede obtener una vista global del estado de
los motores en la base de datos, un recuento de la evolución histórica del
motor (por medio de gráficas y una tabla)  y una gráfica de un análisis en
frecuencia correspondiente al espectro de vibración que posee el motor;
además la página consta de un  manual de usuario para facilitar su
utilización. De esta forma de tiene un sistema completamente modular que
puede escalar con el tiempo y puede ser alimentado por cualquier sistema de
adquisición de datos que respete los formatos y estructuras definidas.

Palabras claves: servidor, microservicio, modelo estadístico, vibración,
motores eléctricos.


%***************************************************
%**********  Capitulo 1  ***************************
%***************************************************

	\newpage
    \begingroup
    \let\clearpage\relax
	\setcounter{page}{1}

\begin{center}
	\section{Planteamiento del problema}
\end{center}


\subsection{Descripción del problema}

	En la actualidad se vive un crecimiento exponencial a nivel industrial dadas las altas demandas de alimentos e insumos de toda clase; Esto es posible gracias al motor eléctrico, este es un artefacto que transforma la energía eléctrica en energía mecánica (movimiento), de manera que puede impulsar el funcionamiento de una máquina y son utilizados ampliamente en: Bombas para Agua, Bombas Industriales, Mezcladoras, Molinos, Correas Transportadoras, Zarandas, Cortadoras, Ventiladores, Grúas, y en todo proceso que involucre movimiento.\\
	Si a esta diversidad de usos se le suma la gran perdida (horas, insumos,dinero, etc) que ocasiona una parada de emergencia en una planta, Se obtiene la gran importancia de que los motores se encuentren completamente operativos y funcionales. Para procurar su buen estado se realizan mantenimientos.\\



	Existen diferentes tipos de mantenimiento, entre ellos están: 
	\begin{itemize}
		\item correctivo: se espera que ocurra una falla para
	reparar o cambiar un equipo. Esto puede degradar la vida útil del equipo y debido a que la falla puede ocurrir en cualquier momento, usualmente se produce un paro en la linea de producción, por lo tanto este tipo de mantenimiento suele y debe ser evitado.

		\item preventivo: para evitar una falla mayor se detiene la maquinaria para hacer un mantenimiento preparado con anticipación, se inspecciona la maquinaria y se remplazan las piezas propensas a dañarse. Este tipo de mantenimiento en algunas circunstancias es más que suficiente pero en el caso de los rodamientos puede ser contraproducente.

		\item predictivo: se predice cuando una falla esta a punto de ocurrir; A través de mediciones y estudio se predice cuando una falla esta a punto de ocurrir y de esta forma se realiza una mejor planificación. Cabe resaltar que este no solo permite predecir fallas sino que puede dar
		información acerca del origen de las mismas.
	\end{itemize}

		En el caso de los motores eléctricos, como dice Dr.S. J. Lacey en The Role of Vibration Monitoring in Predictive Maintenance, el mantenimiento preventivo tiene muchas desventajas dado que existen tanto problemas de índole mecánico como administrativo y monetario, como lo pueden ser, los altos costos de reemplazo, dado que las partes se reemplazan muy pronto, el riesgo de perdida completa dado un error humano, instalación de una pieza defectuosa, además de la posibilidad de generar daño o una incorrecta instalación de la misma, Y por ultimo, el hecho de que las piezas reemplazadas pueden tener muchos años de vida útil~\cite{Lacey}.\\
		por otro lado, el mantenimiento predictivo ofrece mas control sobre estas variables, aunque no evita las posibilidades de error humano, si permite reaccionar a este; Adicionalmente a esto, se debe considerar las altas perdidas y retrasos, ademas de dificultades administrativas, que generan las paradas periódicas de la planta.




	
%	
%\subsection{solución}

Para solucionar la problemática planteada, se propone el desarrollo de un sistema capaz de tomar datos de forma continua y enviarlos  a un servidor el cual permita el almacenar, estudiar y muestrear la información en distintos niveles de profundidad, con respecto al análisis realizado.\\


La solución propuesta es un sistema digital diagramado en la Figura \ref{diagrama}, la cual producirá un análisis estadístico para obtener las medidas típicas de un acelerómetro en motores eléctricos con distintos niveles de daños, esta data permitirá, después de ser almacenada en una base de datos y procesada, generar 3 niveles de análisis:\\


\begin{itemize}
\item La vista principal, permitirá observar una cantidad específica de motores, simbolizando los existentes en una planta o piso, y su estado general.

\item La vista específica, dará la información actual e histórica referente a un único motor previamente seleccionado.

\item La vista exhaustiva se refiere a un análisis en frecuencia de la vibración de un motor especificado con anterioridad, con la finalidad de permitir al operador o ingeniero encargado determinar la causa de las posibles averías.
\end{itemize}


Y finalmente, toda esta información y opciones se mostrarían a través de una página Web para facilitar su acceso.\\



\begin{figure}[htb]
\centering
\caption{Diagrama de la simulación digital.}
\label{diagrama}
\includegraphics[width=15cm]{Diagrama_sensorica.png}
\end{figure}

	\subsection{Objetivos de la investigación}

\subsubsection{Objetivo general}
	\begin{enumerate}
		\item  Desarrollar la simulación digital del monitoreo de la vibración en motores eléctricos mediante un acelerómetro, con la finalidad de hacer diagnósticos predictivos.
	\end{enumerate}
	
\subsubsection{Objetivos específicos}

	\begin{enumerate}
		\item Justificar la escogencia de las herramientas y lenguajes a utilizar en las diferentes etapas que requiere la simulación. (José cortez y Gerardo Campos)

		\item Elaborar un análisis estadístico de motores con distinto grado de daño que establezca la distribución de la salida de acelerómetro digital. (José cortez)

		\item Elaborar una base de datos con la cual se establezcan el análisis  en simulación digital. (José cortez y Gerardo Campos) 
		
		\item Realizar análisis de fallas en frecuencia, a partir de la salida de la distribución del acelerómetro. (Gerardo Campos)

		\item Mostrar la información solicitada de acuerdo al nivel de análisis seleccionado. según sea: Vista Principal, Vista Específica o Vista Exhaustiva. (José cortez y Gerardo Campos)

		\item Comprobar los resultados de la simulación digital. (José cortez y Gerardo Campos)
	\end{enumerate}

	
	\subsection{JUSTIFICACIÓN E IMPORTANCIA}

Para prevenir las fallas mecánicas que ocurren en los motores, por el deterioro
y desgaste de los rodamientos, estos se deben tener bajo constante monitoreo
para poder efectuar un mantenimiento puntual que elimine dichos peligros. De
esta forma el mantenimiento predictivo es la clave para mejorar la vida útil,
funcionamiento y planificación de todo proceso especialmente a niveles
industriales, donde la cantidad de motores eléctricos es bastante elevada, y
por lo cual la automatización del proceso es crucial.

Sin embargo, dado los altos costos que implican realizar una automatización y
en especial a escalas industriales se suele hacer una simulación o una
emulación para estudiar y obtener el modelo más preciso, gracias a la facilidad
para manipular con mayor eficacia los diseños y conocer su comportamiento real
sin necesidad de construirlo, antes de realizar la implementación y todo el
proceso que esta conlleva.

Habiendo expuesto la importancia del mantenimiento como también la de realizar
simulaciones, se justifica el hecho de realizar el desarrollo del trabajo aquí
propuesto el cual permita emular el comportamiento y las salidas de un
acelerómetro, como también otorgue las herramientas necesarias para poder
realizar un mantenimiento predictivo, y de esta forma se pueda estudiar a
profundidad su estado actual como también su evolución histórica. Todo esto
sumado a las facilidades de portabilidad que ofrece un sistema Web, facilitando
la revisión constante sin las dificultades de los protocolos de acceso y
sanidad.

	
\subsection{LIMITACIONES Y ALCANCES}
	En función de los objetivos planteados con anterioridad, se puede definir tanto las limitaciones como el alcance del proyecto.

\subsubsection{Limitaciones}
\begin{itemize}
	\item Recursos económicos impiden la adquisición de dispositivos para pruebas en motores reales.

	\item Disponibilidad de muestras, aunque se cuenta con una BBDD lo suficiente grande para cubrir el comportamiento de la vibración en motores, incluso de distinta potencia, esta es discreta y con intervalos de tiempo considerables entre cada muestra.

	\item Hardware utilizado, dado que para el desarrollo del servidor y la prueba del mismo se cuenta es con computadoras portátiles con bastante antigüedad, limitando de esta forma los recursos disponibles y por tanto la velocidad de desarrollo.

	\item La gran variedad de conocimientos. El conocimiento requerido es bastante amplio y ademas cubre áreas fuera del campo de la electrónica, como lo son probabilidad y mecánica de las cuales se cuenta con muy poco conocimiento previo, motivo por el cual se ralentiza el avance del proyecto.

	\item Las establecidas por el software de tercero, utilizado para incrementar la velocidad en el diseño del proyecto (parte gráfica).
\end{itemize}

\subsubsection{Alcance}
	El principal alcance de este trabajo es el desarrollo de una herramienta computacional para el análisis de la vibración en motores eléctricos; en función de esto:

	\begin{itemize}
		\item La herramienta computacional permite una vista general del estado de todos los motores en un espacio previamente delimitado y seleccionado (planta o piso) que se encuentren en la BBDD.

		\item La herramienta computacional muestra el estado especifico de un motor, seleccionado previamente por el operando, con todas sus características de forma mas detallada. 

		\item La La herramienta computacional genera un análisis en frecuencia de la vibración en un motor especifico para facilitar el estudio y la toma de decisiones del operando.

		\item Dados los puntos anteriores, el sistema consta de una Base de datos la cual permita su correcto funcionamiento y el almacenamiento de los valores históricos de cada motor.

		\item Se implementa un modelo estadístico para generar la información necesaria para el llenado de la BBDD  de la herramienta computacional y de esta forma evaluar su funcionamiento.

		\item Todo el sistema puede ser accedido de forma web para facilitar el acceso.

		\item no se contempla la construcción de hardware de ningún tipo.


	\end{itemize}
    \endgroup


%***************************************************
%**********  Capitulo 2  ***************************
%***************************************************
%
%\subsection{solución}

Para solucionar la problemática planteada, se propone el desarrollo de un sistema capaz de tomar datos de forma continua y enviarlos  a un servidor el cual permita el almacenar, estudiar y muestrear la información en distintos niveles de profundidad, con respecto al análisis realizado.\\


La solución propuesta es un sistema digital diagramado en la Figura \ref{diagrama}, la cual producirá un análisis estadístico para obtener las medidas típicas de un acelerómetro en motores eléctricos con distintos niveles de daños, esta data permitirá, después de ser almacenada en una base de datos y procesada, generar 3 niveles de análisis:\\


\begin{itemize}
\item La vista principal, permitirá observar una cantidad específica de motores, simbolizando los existentes en una planta o piso, y su estado general.

\item La vista específica, dará la información actual e histórica referente a un único motor previamente seleccionado.

\item La vista exhaustiva se refiere a un análisis en frecuencia de la vibración de un motor especificado con anterioridad, con la finalidad de permitir al operador o ingeniero encargado determinar la causa de las posibles averías.
\end{itemize}


Y finalmente, toda esta información y opciones se mostrarían a través de una página Web para facilitar su acceso.\\



\begin{figure}[htb]
\centering
\caption{Diagrama de la simulación digital.}
\label{diagrama}
\includegraphics[width=15cm]{Diagrama_sensorica.png}
\end{figure}


	\newpage
    \begingroup
    \let\clearpage\relax
	\thispagestyle{empty}

\section{REVISIÓN BIBLIOGRÁFICA}

\subsection{ANTECEDENTES}

\textcite{Ramazan} Describen un modelo estadístico sobre el deterioro de
motores eléctricos de inducción. El artículo clasifica el deterioro del motor
en 7 etapas de acuerdo al nivel de vibración de sus rodamientos. Este
antecedente es de gran utilidad para realizar el modelo estadístico que alimenta
la base de datos ya que proporciona bases teóricas y un ejemplo práctico.

\textcite{Pinto} realizan una simulación de redes de sensores inalámbricos, con
el fin de mejorar la detección de fallas de los sensores. Esta simulación fue
realizada con Castalia un simulador de redes de sensores inalámbricos y el
sensor usado fue un detector de luminosidad. Algunas de las razones que
menciona el artículo de por qué fue escogida una simulación es debido a que
realizar el experimento con sensores requiere un costo elevado de hardware y un
estudio analítico no es efectivo, ya que la complejidad del sistema es muy
grande. Este trabajo es de utilidad debido a que sirve como orientación a la
elaboración de la simulación del sensor que generaría el modelo estadístico.

\textcite{Ugwiri} Presentan un resumen de las técnicas más usadas actualmente
en la detección de fallas en motores eléctricos mediante análisis de vibración,
además de realizar un experimento donde se ponen a prueba algunos de estos
conceptos. El propósito de este antecedente es el de apoyar la elaboración del
modo de "vista exhaustiva".

\textcite{Koene} Elaboran un sensor de vibración inalámbrico de código libre
llamado Memsio. Este dispositivo es alimentado por baterías y permite la
adquisición de datos a alta velocidad por medio del uso de un acelerómetro
microelectromecánico. Los autores mencionan que en la industria los sensores de
aceleración más usados son los piezoeléctricos, debido a tener una mayor
precisión y tolerancia al ruido, sin embargo, los avances de los dispositivos
microelectromecánico y su bajo costo hacen cada vez más factibles su uso para
el monitoreo. Este antecedente muestra la tendencia de la reducción de precios
de los sensores inalámbricos lo cual apoya al propósito del trabajo al hacer
factible las redes de sensores.

\textcite{Soto-Ocampo} Elaboraron un sensor de vibración multicanal para
vibraciones de alta frecuencia utilizando una Raspberry pi. El objetivo del
proyecto era conseguir una alternativa de bajo costo, para poder monitorizar
la salud de rodamientos en motores eléctricos, pero sin sacrificar la calidad de
la medición porque, como explican los autores del artículo, la mayoría de las
fallas en los rodamientos ocurre a altas frecuencias y muchas de las
alternativas de bajo costo no alcanzan la frecuencia necesaria. Este antecedente
es de utilidad dada la cantidad de información, tanto teórica como datos de
simulación, que provee.



	\subsection{MARCO TEÓRICO}

Esta parte del capítulo expone el contenido teórico necesario para la
realización de este proyecto. Esto incluye información sobre los motores
eléctricos,  análisis de la vibración, las herramientas computacionales, los
sistemas y el modelado estadístico.

%https://es.wikipedia.org/wiki/Motor_el%C3%A9ctrico


\subsubsection{ Motores eléctricos}

De acuerdo a \textcite{Fraile}, un motor eléctrico es un dispositivo que
transforma energía eléctrica en
energía mecánica mediante la acción de campos magnéticos y están compuestos,
principalmente, por un estator (parte fija) y un rotor (parte móvil).
Existen dos familias principales de motores eléctricos las cuales, a su vez,
se subdividen en \textbf{motores de corriente alterna} como lo son los  motores
de inducción, síncronas, entre otros y los \textbf{motores de corriente continua}
como lo son los motores de escobillas,  sin escobillas, de imán permanente,
entre otros.

El motor más usado, en el sector industrial, es el motor de inducción debido
principalmente a su bajo costo y al poco mantenimiento que requiere para estar
completamente operativo, esto en particular es debido a su sencillo diseño
en comparación al de otros motores de igual potencia, ya que requiere  menor
número de componentes. Cabe resaltar que el segundo tipo de motor más usado es
el motor de corriente
continua, debido a que su control, en términos de torque y potencia, es mucho
mas sencillo para ciertos niveles de potencia.

También existen los motores síncronos cuya construcción es similar a la de un
motor de inducción pero vez requiere de una fuente de alimentación externa
para el rotor, lo cual aumenta su costo, además de esto su velocidad es
constante, por lo tanto este tipo de motor es utilizado en aplicaciones
específicas. Como se mencionó anteriormente, existen otros tipos de motores,
pero son usualmente de menor potencia por
lo tanto su utilidad industrial es mucho más limitada.


\subsubsection*{Rodamientos}

Para que un motor pueda llevar a cabo la transformación de potencia debe rotar.
Esta acción es llevada a cabo por el \textbf{rotor}, el cual está formado por un
eje que soporta un juego de bobinas envueltas sobre un núcleo magnético. Este
se encuentra suspendido por \textbf{rodamientos}, de forma que el movimiento y
las fuerzas producidas en la interacción entre las bobinas y el núcleo con los
campos magnéticos, producidos por el estator, puedan ser utilizados. Además,
los rodamientos, cuando se encuentran en buenas condiciones, permiten minimizar
el roce entre el eje y soporte, maximizando la transferencia de potencia.

Estos también son conocidos como \textbf{rolineras}, las cuales según
\textcite{rodamiento},  son un tipo de cojinete,
un elemento mecánico diseñado para reducir la fricción entre un eje y los
elementos conectados al mismo. Existen muchos tipos de rolineras, sin embargo
su estructura se fundamenta en dos anillos concéntricos, algún elemento rotativo
como pueden ser bolas o rodillos, una jaula que se encarga de mantener los
elementos de rodadura separados además de guiados y un lubricante o un sistema
de lubricación.

Dado que son un elemento mecánico, el cual es continuamente usado, sufre mucho
desgaste y es el elemento más propenso a dañarse; según los estudios realizados
por ~\textcite{Kammermann} el 59\% de las fallas son causadas por los rodamientos
y asimismo su principal falla es el desgaste, de igual forma se presentan
otras fallas estructurales. Por estas razones es fundamental tener bajo continuo
monitoreo este elemento, esto se suele hacer a través de un análisis de vibración.


\subsubsection{Análisis de Vibración}

Como se explica en \textcite{wiki:Vibration}, la vibración es el  movimiento
periódico de un cuerpo o medio
elástico, alrededor de un punto de equilibrio. En general podemos decir que una
vibración es un caso específico de una oscilación, cuando esta es de origen
mecánico.

Asimismo, se conoce como análisis de vibración, al conjunto de técnicas que permite
obtener información de un equipo, a partir de sus vibraciones. Es una de las
técnicas más usadas en el mantenimiento predictivo de equipos mecánicos, debido
a que es un proceso poco invasivo y de bajo costo. El análisis de vibración
permite diagnosticar fallas de forma temprana, así como detectar señales
prematuras de desgaste.

Según Girdhar \textcite{Girdhar},  el análisis de vibración nos permite detectar las
siguientes fallas:

\begin{itemize}[noitemsep]

\item Defectos en los engranajes
\item Defectos en los rodamientos
\item Desalineamientos
\item Desbalances
\item Ejes torcidos
\item Excentricidad
\item Fallas eléctricas
\item Fuerzas hidráulicas o aerodinámicas
\item Mala sujeción en las piezas
\item Problemas en las correas de transmisión
\item Problemas de lubricación
\item Resonancia
\item Rozamientos en el rotor
\end{itemize}


\subsubsection*{Adquisición de datos}

Para realizar cualquier tipo de análisis, primero se deben adquirir datos y,
por regla general, mientras más datos se tengan y más precisas sean las
mediciones, mejores serán los estudios que se pueden realizar.


La adquisición de datos, según \textcite{adquisiciondatos}, es el proceso de
realizar mediciones de fenómenos físicos
y registrarlos, en algún formato específico, para analizarlos posteriormente.
Cabe resaltar que en algunos casos es necesario hacer un acondicionamiento de
la señal, esto
implica modificarla controladamente al reducir o aumentar su amplitud además de
sumarle un nivel offset, voltaje continuo, para que cumpla requisitos mínimos
y pueda ser procesada por los siguientes circuitos.

Al momento de la adquisición de datos, se toman medidas de una señal analógica
la cual se convertirá al pasar por una serie de etapas y dispositivos en una
señal digital y se guardará en un formato deseado y en una unidad de
almacenamiento masivo (ROM, flash, etc.).
El primer paso en la toma de  datos comienza con el sensor, que es un
dispositivo el cual transforma la unidad física de interés, en una señal que
pueda ser procesada con mayor facilidad, luego se adecuará mediante un
circuito especializado a las características y requerimientos del sistema,
posteriormente será procesada por un convertidor Analógico-Digital, el cual se encarga de
muestrear, retener y procesarla  y, de esta forma, obtener una versión
digital de la misma, la cual se  guardará mediante un software desde
una computadora.

En el caso de las vibraciones, los sensores más usados son los
acelerómetros.  Esto se debe principalmente a que tienen mayor ancho de banda
que los sensores de posición y los de velocidad, lo que les permite detectar
vibraciones de mayor frecuencia. Adicionalmente, dado que las vibraciones en los
rodamientos, por naturaleza, son de alta frecuencia y, como se mencionó anteriormente,
son componentes críticos en los motores eléctricos.
Estas características hacen al acelerómetro el sensor más utilizado para las
mediciones de vibración en motores eléctricos. Sin embargo,
en casos más especializados, como podría ser el análisis de vibración
de maquinaria de larga envergadura, son también usados los otros tipos de
sensores ya que sus características pueden facilitar el estudio.


\subsubsection{Acelerómetros}

Como se expone en \textcite{Fraden}, un acelerómetro es un dispositivo capaz de
medir aceleración, no es
necesariamente la misma que la aceleración de coordenadas (cambio de la velocidad de
un elemento en el espacio), sino que es la correlación asociada con el fenómeno
de peso experimentado por una masa de prueba que se
encuentra en el marco de referencia del dispositivo.  Funciona mediante
la utilización de la segunda ley de Newton, \textbf{la fuerza resultante
ejercida sobre un cuerpo es proporcional a su masa por su aceleración}. Los
acelerómetros por lo general cuentan con una masa de prueba (también conocida
como masa sísmica), alguna especie de resorte y un marco de soporte que, a su
vez, puede funcionar como amortiguador. Debido a esto, los acelerómetros se
pueden modelar matemáticamente como un sistema lineal de segundo orden,
por lo que su respuesta en frecuencia posee un pico de resonancia.


\subsubsection*{Tipos de acelerómetros}

\begin{itemize}
    \item  Acelerómetros Capacitivos:

        Los acelerómetros capacitivos son uno de los modelos más sencillos
        capaces de medir
        aceleración, además de ser fácilmente utilizables y reproducibles en masa.
        Funcionan mediante una masa sísmica dado que al  experimentar una fuerza
        se desplaza con una aceleración proporcional a la fuerza aplicada.
        Si a la masa se le agregan unos resortes unidos a una carcasa, los resortes
        ejercerán una fuerza proporcional al desplazamiento de la masa generando un
        desplazamiento fácilmente medible.

        Este fenómeno se produce ya que estos efectos en conjunto al
        amortiguamiento producen un sistema lineal de
        segundo orden, este sistema matemático genera una salida en desplazamiento
        al aplicarse como entrada una fuerza
         Finalmente, al conectarse un sensor de desplazamiento, se observa
        la aceleración a la que esta sometido el instrumento.
        El sensor de desplazamiento más
        popular para este tipo de medidores es el capacitivo.

        En su mayoría, los acelerómetros capacitivos vienen como dispositivos
        microelectromecánicos (\textbf{mems} por sus siglas en inglés) que son
        dispositivos
        fabricados con técnicas similares a las de fabricación de circuitos
        integrados; a su vez, poseen componentes mecánicos microscópicos. Son los
        acelerómetros más populares en aplicaciones no industriales, sin embargo,
        en aplicaciones de bajo consumo, bajo coste o cuando las frecuencias con
        las que se trabajan no son tan altas, pueden ser usados a niveles
        industriales, ya que poseen múltiples ventajas como:

        \begin{itemize}[noitemsep]
            \item No requieren adecuación de la señal.
            \item Pueden comunicarse directamente con un microcontrolador.
            \item Su precio es económico.
            \item Son compactos.
        \end{itemize}


    \item  Acelerómetros piezoresistivos:

        Son, después de los acelerómetros piezoeléctricos, los más usados a nivel
        industrial. Su funcionamiento es similar al de los acelerómetros
        capacitivos; ante una aceleración de entrada se produce un desplazamiento
        de salida más, en este caso, están constituidos por una o varias
        galgas extensiométricas, una masa de prueba y unos resortes de soporte.
        La galga sujeta a la masa sísmica y, al esta recibir una fuerza, produce
        un desplazamiento proporcional a la fuerza aplicada, lo que deforma a
        su vez la galga extensiométrica lo cual se traduce como un
        cambio de resistencia en el sensor. La ventaja de los acelerómetros
        piezoresistivos es que pueden medir valores de voltaje DC lo que los
        hace útil en el estudio de impactos, son también usados en el
        análisis de vibración en el rango de mediana frecuencia.

\newpage
    \item Acelerómetros piezoeléctricos:

        Según \textcite{WeberPiezoelectricAT}, es el acelerómetro más usado en
        aplicaciones industriales ya que
        poseen características como:

        \begin{itemize}[noitemsep]
            \item Alto rango dinámico.
            \item Bajos niveles de ruido.
            \item Alta linealidad.
            \item Alto ancho de banda.
            \item Poco desgaste, ya que no poseen partes móviles.
        \end{itemize}


        Su construcción es bastante sencilla, se tiene disco de un cristal
        piezoeléctrico unido por dos terminales circulares, de forma similar a
        la de un condensador, y justo encima, tienen una masa de prueba.
        Al experimentar una fuerza el cristal se deforma lo que produce una
        diferencia de carga y un voltaje proporcional a la fuerza aplicada.
        Como la aceleración de la masa de prueba es también proporcional a la
        fuerza aplicada, la aceleración del acelerómetro será entonces
        directamente proporcional al voltaje y la carga producida en el cristal.

\end{itemize}


\subsubsection{Procesamiento de señales}

Después de ser almacenada la información, debe ser estudiada, procesada, para lo
cual se utiliza una serie de herramientas, técnicas o software especializados
a cada necesidad. Este estudio se puede categorizar en dos ramas principales:

\begin{itemize}
    \item Dominio del tiempo, según \textcite{wiki:DominioTiempo}, es un término
        utilizado para describir el análisis
        de funciones matemáticas o señales con respecto al tiempo, la sucesión
        de estados que atraviesa la señal de forma natural. Los estudios más
        comunes son en  \textbf{tiempo continuo} y en \textbf{tiempo discreto}.

    \item Dominio de la frecuencia, según \textcite{wiki:DominioFrecuencia}, es un
        término utilizado para describir el análisis
        de funciones matemáticas, señales o movimientos periódicos respecto a
        su frecuencia, número de veces que sucede un evento en un periodo.
        Utilizan transformadas para llevar las funciones o señales del dominio
        del tiempo, base, al dominio de la frecuencia, deseado, la más famosa es
        la \textbf{transformada de Fourier}.
\end{itemize}


Gráficamente se suele entender el \textbf{dominio temporal} como la evolución
de una señal con respecto al tiempo, es decir su evolución natural, por otro
lado, el \textbf{dominio frecuencial} muestra los componentes de la señal según
la frecuencia en la que oscilan dentro de un rango determinado.
En la figura \ref{Dominios} se observan ejemplos gráficos de señales en ambos
dominios.

	\begin{figure}[htb]
		\centering
        \caption{ Diagrama de los dominios temporal y frecuencial}
        \includegraphics[width=\linewidth]{Dominios.png}
        Realizada con el Software Octave a partir de los datos de
                \textcite{HUANG20181745}
        \label{Dominios}

	\end{figure}


El análisis en frecuencia suele ser más utilizado debido a que
la mayoría de las fallas poseen frecuencias características y, dado que en  el
análisis de frecuencia se descompone en frecuencias la señal, se facilita la
detección de fallas características, así mismo, la amplitud de la frecuencia es
directamente proporcional al nivel de la falla. Por lo tanto, se obtiene un
espectro amplio del estado de la pieza.
Cabe resaltar que cuando las frecuencias son bajas o muy cercanas entre si,
se dificulta determinar e identificar alguna falla, suele suceder cuando se
estudia una falla o evento con frecuencia muy baja o muy cercana a la frecuencia
natural de la señal o elemento medido. En estos casos es mejor
usar un análisis en el dominio del tiempo que  facilita la
detección de las fallas.



\subsubsection{Herramienta Computacional}


Según \textcite{Herramienta}, una herramienta computacional puede ser definida  como
cualquier software,
sistema de integración, análisis o almacenamiento que  ayuda a los científicos
o usuarios a solucionar un problema específico en una determinada rama. Pueden
variar desde sistemas complejos como compiladores, algoritmos
e incluso sistemas operativos hasta herramientas como hojas de cálculos, sistemas
de oficina o medios de comunicación. Funcionan mediante la implementación de
técnicas y protocolos para solucionar problemas de forma iterativa o con una
secuencia de pasos concreta.

Siguiendo este orden de ideas, una gran cantidad de estas herramientas son
encontradas en la librería de información más grande del mundo, el Internet.
Todas comparten la peculiaridad de que son un \textbf{sistema} y,  por ende, pueden
ser accedidas con facilidad desde cualquier punto con un dispositivo capaz de
tener conexión a Internet y un navegador. Esta facilidad se debe a que un
\textbf{servidor} se encarga de hacer el procesamiento de la información y envía
el resultado con un formato específico, típicamente  JSON, por \comillas{notación de
objeto de JavaScript} el cual es un formato de texto sencillo para el intercambio de
datos, este se renderiza (proceso para generar una representación gráfica por
medio de programas informáticos) en una página Web.

\subsubsection{Sistema Web}

Los sistemas Web, de acuerdo a \textcite{SistWeb1} y \textcite{wiki:systemWeb}, o
también conocidos como aplicaciones Web son sistemas que
utilizan la tecnología Web y el Internet o Intranet para transmitir la
información y los
servicios a usuarios u otros sistemas/aplicaciones. Estos sistemas utilizan los
principios del hipertexto para renderizar la información en cualquier
navegador o \textbf{página Web} y el poder de los \textbf{servidores} para
almacenar y procesar la información. Por estas características son independientes
de cualquier plataforma o sistema operativo, además de  no requerir ningún
proceso de instalación, facilitando de esta forma el acceso, la gestión y la
rapidez de obtención de información.



\subsubsection*{Página Web}
Una página Web, como se explica en \textcite{WebpageMozila},  es un documento
accesible desde cualquier navegador con acceso
a Internet que puede incluir audio, vídeo, texto y sus diferentes
combinaciones.
Funciona al usar el protocolo HTTP, conocido usualmente como la Web, y una
estructura de hipertexto la cual permite redirigir, enlazar y estructurar el
contenido y lo hace fácilmente accesible desde un navegador Web.

Funciona gracias al protocolo HTTP, \comillas{Hypertext Transfer Protocol},
el cual es la base de cualquier intercambio de datos en la Web y un protocolo
de estructura cliente-servidor, esto implica que una petición de datos es
iniciada por el elemento que recibirá los datos (el cliente), normalmente un
navegador Web, y es cubierta por el elemento que envía los datos (el servidor).
Este protocolo comenzó siendo estático y dirigido usualmente a la transmisión de
texto pero se fue convirtiendo en más que eso y, en la actualidad, permite la
transferencia de documentos de todo tipo, Script, vídeos, entre otros; a tal
punto que es fácilmente categorizado como el protocolo más usado en todo el
mundo, siendo incluso utilizado como sinónimo de Internet cuando es solo una
parte de él.

Debido a la invención de tecnologías, como JavaScript y AJAX, hoy en día es
posible tener aplicaciones Web que son programas, junto a una interfaz gráfica,
que permiten comunicarse con servidores que realizan la mayor parte del trabajo
del desarrollo de aplicaciones complejas que funcionen desde la
comodidad de dispositivos móviles. La Web permite, por tanto, facilidad al
transmitir información así como el acceso a cualquier contenido desde
cualquier dispositivo, en cualquier momento.

La Web suele ser el método de acceso de muchas tecnologías y, si bien en la actualidad
el desarrollo Web usa el mismo estándar de tecnologías, el lado del servidor
contempla una variedad mucho más amplia, dado que cualquier aplicación que
pueda correr en un ordenador puede ser conectada a una interfaz Web. Teniendo
como limitante principal la latencia,  tiempo que tarda la información en viajar,
una interfaz Web es, para un usuario promedio, una solución cómoda y
accesible la cual  permite incluso  mayor comodidad y facilidad de acceso.


\subsubsection*{Servidor}

Un servidor Web, como lo define \textcite{servidor},  es un ordenador de propósito
específico que permite la
transición de datos a uno o múltiples clientes Web. Para esto, el dispositivo
debe estar configurado para escuchar las solicitudes de los clientes en un
entorno red. Esto se logra mediante una aplicación externa o el uso de un
sistema operativo dedicado; almacena los archivos
necesarios para el procesamiento de información y los datos necesarios para
mostrarla, además, se encarga de distribuirla al usuario final.

Los servidores se suelen clasificar según su función y es común que cumplan más
de una función, o se encuentren más de un tipo en una red. Algunos de estos son
servidores de archivos, impresión, aplicaciones, DNS, \textbf{Web}, entre otros.
Actualmente, los servidores Web son los más abundantes en el mercado
y se caracterizan por alojar la información y los datos de los usuarios a través
de Internet o Intranet. Estos responden a las solicitudes de páginas Web u otros
servidores basados en esta tecnología.

\subsubsection{Base de datos (BBDD)}
Una base de datos como dice \textcite{bbdd} es una herramienta para almacenar y
organizar información.
Las bases de datos pueden contener información de cualquier tipo
y se utiliza para evitar problemas como redundancia y para facilitar el manejo de
grandes cantidades de la misma, en fin,  una base de datos
computarizada es un contenedor de objetos la cual permite:

\begin{itemize}
    \item Agregar nuevos elementos.
    \item Editar y mover información en la base de datos.
    \item Eliminar información.
    \item Organizarla y visualizarla de diferentes formas.
    \item Compartirla con otros usuarios, programadores o interesados.
\end{itemize}

Las bases de datos, según \textcite{bbddTipos} se suelen dividir en dos tipos, de
acuerdo al tipo de lenguaje
utilizado para manipularlo, estas son SQL( ``Structured Query Languaje" utilizado
para bases de datos relacionales) y NoSQL (Sus tipos de datos no suelen ser
estructurados, aunque algunas bases de datos permiten su estructuración); su
utilización o selección depende mucho del tipo de procesamiento que se le
dará a la información contenida en ella, algunos otros factores importantes
a considerar son la velocidad de escritura-lectura y paralelismo.

\subsubsection{MongoDB}
MongoDB según \textcite{MongoDB} es un sistema de base de datos no relacional de código
abierto el cual utiliza documentos flexibles en lugar de tablas y columnas para
almacenar y procesar varios tipos de información. Al ser una solución NoSQL,
MongoDB no requiere un sistema de manejo relacional permitiendo esto un modelado
mas elástico al momento de guardar la información, además, permite mas libertad
en las consultas lo cual, además de simplificar el manejo para los desarrolladores,
permite desarrollar un ecosistema fácilmente escalable en aplicaciones y servicios
multi plataformas.

Cabe resaltar que el modelo de estructuración de datos que utiliza MongoDB es
BSON el cual es formato binario de JSON.

\subsubsection{MongoDB Atlas}
Como \textcite{MongoDBAtlas} especifica en su documentación oficial, MongoDB Atlas
es un servicio de base de datos (Database-as-a-Server, DBaaS) el cual permite
establecer, llevar a producción y escalar las bases de datos sin preocuparse por
el hardware físico, actualizaciones de software y los detalles de configuración.

Es decir, MongoDB Atlas es un sistema completamente manejado en la nube el cual
maneja toda la complejidad de llevar a producción el sistema, manejarlo y revisar
el estado de sus componentes, además, permite la utilización de cualquier proveedor
de servicios (AWS,Azure y GCP).

Cabe resaltar que MongoDB Atlas es un servicio pago, sin embargo, permite la
utilización de un ``cluster" gratuito, de almacenamiento limitado, para
prácticas y aprendizaje.

\subsubsection{Interfaz de programación de aplicaciones (API)}
Interfaz de programación de aplicaciones es una forma de simplificar el diseño
de software al permitir el intercambio de información y funcionalidades de forma
rápida y segura. Como explica \textcite{API} una API permite a las compañías y
desarrolladores abrir y expandir las informaciones y funcionalidades que poseen
con grupos externos de desarrolladores, compañeros de negocios e incluso departamentos
internos dentro de la misma compañía, esto permite separar los desarrollos y
trabajar de forma paralela ya que los desarrolladores no necesitan conocer la
implementación, simplemente la utilizan como interfaz para comunicarse con otros
productos y servicios.

Cabe resaltar que sin estas fuera imposible el desarrollo de muchas aplicaciones
populares. Una API funciona al ser un conjunto de normas que definen como se
comunicarán las computadoras o aplicaciones entre ellas, es decir, sirve como
una capa de abstracción entre el servidor y la aplicación.

\subsubsection{Modelo estadístico}

Un modelo estadístico, de acuerdo a \textcite{modeloIBM}, es una representación
matemática que permite, mediante
ecuaciones, codificar información extraída de los datos y, de esta forma,
predecir el comportamiento de un sistema ante situaciones dadas. Funcionan
mediante  variables aleatorias, una o más variables de las cuales no
se tiene completa certeza de su valor o provienen de algún evento aleatorio.

Un modelo estadístico permite inferir ciertas características de un evento,
como qué tan probable es tal evento y cómo se distribuyen los valores de la
variable. Además, se suele usar como primer paso en generar un modelo más
preciso o para la obtención de información cuando no se tiene suficiente,
es difícil su acceso o la naturaleza del
sistema es extremadamente compleja y dicha tarea es simplemente imposible.


\subsubsection{Prueba de bondad de ajuste}
Una prueba de bondad de ajuste es una prueba estadística que se realiza a un
modelo estadístico con el fin de obtener un valor que permita comparar qué tan
bien se ajusta el modelo estadístico seleccionado a los datos experimentales.
En general las pruebas de bondad suelen cuantificar las diferencias que existen
entre la distribución teórica y la distribución experimental. Según \textcite{statisticsMit} la
prueba estándar de bondad de ajuste es la prueba de kolmogorov-Smirnov, siendo
esta la más usada porque representa intuitivamente la noción de distancia entre
dos distribuciones.

\subsubsection{Prueba de Kolmogorov-Smirnov}
La prueba de Kolmogorov-Smirnov es una prueba estadística no paramétrica que
permite medir la distancia entre dos distribuciones, su principal uso es para
calcular pruebas de bondad de ajuste comparando los valores teóricos con los
valores experimentales de un modelo estadístico. Para comparar la distancia
entre distribuciones primero se consigue la función de distribución acumulada
(FDA) para cada una de las distribuciones a comparar, una vez obtenidas las
funciones se calcula el valor absoluto de su diferencia y se consigue el
supremo del conjunto (para un número finito de puntos es el mayor valor) y
finalmente una vez obtenido este valor se puede calcular con este un valor \textbf{P}
donde la hipótesis nula es que las distribuciones provienen de la misma
distribución. En el caso de pruebas de bondad solo se posee una muestra y el
método sufre algunas modificaciones, \textcite{GregoryW} explican que la prueba de
Kolmogorov-Smirnov de una muestra compara la distribución observada con una
distribución empírica modelada con los datos de la muestra.


\subsubsection{Lenguaje de programación}
Los lenguajes de programación pueden ser definidos, según \textcite{ETAC} como
sistemas estructurados
que permiten a las personas o programadores interactuar y dar instrucciones a un
programa o software con la finalidad de lograr objetivos.

En la actualidad existe una gran cantidad de lenguajes de programación, algunos
desarrollados en la antigüedad que todavía desarrollan un papel importante en
los sistemas, como lo son C, C++, Java y otros más modernos como lo pueden ser
Go, Rust y Python.

Como se explica en \textcite{javaTpoint}
Los lenguajes de programación suelen ser clasificados en bajo nivel y alto nivel,
haciendo referencia a la necesidad de un compilador o intérprete para poder ser
ejecutado por la computadora, siendo los de bajo nivel lenguaje de máquina o
ensamblador y los de alto nivel los lenguajes comúnmente conocidos, como
Python, Java, JavaScript, PHP, C++, Objective C, Cobol, Perl, Pascal, LISP,
FORTRAN, Go y Swift. Estos lenguajes de alto nivel se pueden subdividir
de acuerdo a la necesidad de un compilador (C,C++,Go, etc.) o de un intérprete
(Python, JavaScript, Ruby, etc.) en \textcite{LenguajesCompiladosEInterpretados}, se
puede leer más de esto. Asimismo, existe otra pequeña subdivisión en el ``tipado"
del lenguaje, que es la necesidad de especificar el tipo de valor que una
variable o constante puede tomar, siendo estas posibilidades un tipado fuerte,
medio o débil, también llamados dinámico (débil) o estático (fuerte).


\subsubsection{Go}
Go es un lenguaje de programación creado en 2007 en Google, específicamente por
Robert Griesemer, Rob Pike y Ken Thompson, su sintaxis es similar a la del lenguaje
C y tiende a ser dinámico como Python pero con un rendimiento equiparable a los
de C o C++. Como se especifica en su página y documentación oficial
\textcite{GolangDocumentacion}: Go es un proyecto de código abierto (open source)
y gratuito
para hacer a los programadores más productivos, es un lenguaje expresivo, conciso,
limpio y eficiente con mecanismos de concurrencia (paralelismo) que permiten fácilmente
obtener el máximo rendimiento posible de las máquinas multinúcleo y redes de máquinas
actuales, mientras permite una programación flexible y modular, además de un compilado
rápido con un recolector de basura (el lenguaje se encarga de liberar memoria
no utilizada).
Es un lenguaje compilado rápido y con tipado estático (las variables y sus tipos tienen que
ser definidos con anterioridad) que se siente como un lenguaje interpretado y
con tipado dinámico (variables modificables sin tipo definido, más sencillo de
escribir el código pero menos estructurado).

Adicional a esto, este lenguaje tiene un ecosistema de comunidades, librerías y
herramientas en constante crecimiento que facilitan el aprendizaje y el desarrollo
de código.

\subsubsection{Fyne}

Fyne es una de las librerías (conjunto de herramientas o paquete) más utilizado
en Go para el desarrollo de interfaces gráficas de usuario (GUI), como se
indica en su página oficial \textcite{fyne}
el conjunto de herramientas de Fyne es una herramienta fácil de aprender, gratuita
y de código abierto (open source) que permite la creación de aplicaciones gráficas
para escritorios, teléfonos y más. Combina el poder y la simplicidad del lenguaje
Go con una cuidadosamente creada librería de Widgets (miniaplicaciones o herramientas)
que hacen ahora más fácil que nunca la construcción de aplicaciones y su despliegue
a producción en todas las plataformas (sistemas operativos, Ios, Linux, Windows, etc.)
y tiendas (Windows Store, Google Play, etc.).

\subsubsection{HTML}

HTML no es considerado un lenguaje de programación propiamente dicho por su
incapacidad de realizar acciones de lógica o aritmética, más específicamente y
de acuerdo a \textcite{HTML},
es un lenguaje de enmaquetado (una forma de escritura que utiliza elementos
sintácticos para dar forma y estructura a la información escrita posteriormente
a un proceso de  compilación o interpretación por la máquina, otros ejemplos son
Latex y Markdown), específicamente un lenguaje de enmaquetado de hipertexto y es
el bloque más básico para la Web ya que define y estructura el contenido de la
misma.

``Hipertexto"\  se refiere a los enlaces que se utilizan para conectar la página
Web con otras partes, ya sean de la misma página o paginas externas a la misma.
Estos enlaces son fundamentales en la Web ya que permiten la actualización y el
entrelazamiento de las páginas creadas por otras personas, esto permite
una participación activa el la\  ``World Wide Web".

HTML usa ``etiquetas"\  para dar información adicional  el texto, las imágenes,
el contenido y permitir
una correcta muestra del contenido, estos elementos sintaxicos son por ejemplo:
 <head>, <title>, <body>, <header>, <footer>, <article>, <section>, <p>, <div>,
 <span>, <img>, <aside>, <audio>, <canvas>, <datalist>, <details>, <embed>,
 <nav>, <output>, <progress>, <video>, <ul>, <ol>, <li>, entre otros.

 De esta forma, el mismo texto entre etiquetas distintas, va a tener características
 visuales distintas, por ejemplo, <h1>Algo<h1> es un título, negritas y tamaño de
 fuente muy superior a <p>Algo<p> que sería un párrafo.


\subsubsection{CSS}

CSS al igual que HTML no es considerado un lenguaje de programación, por las mismas
razones, en cambio, como se describe en \textcite{CSS} es una hoja de estilos en cascada,
un lenguaje utilizado para describir y dar estilo a documentos escritos en HTML o
XML, esto lo consigue al describir cómo los elementos deberían ser mostrados en la
pantalla, papel, habla u otro tipo de multimedia.

Dado estas características CSS es una parte fundamental en el desarrollo Web, ya
sea en su versión para o con la inclusión de Frameworks o librerías como lo son
Bootstrap o TailWind CSS.

Funciona mediante la especificación y modificación de las características en
entornos, etiquetas o elementos únicos, mediante identificadores, de un entorno
HTML o XML y por esto permite la modificación de los atributos.
Usualmente, en términos Web, se modifican tamaños de fuente, peso de la misma,
color de fondo o de fuente, márgenes, espaciado interno (padding), centrado,
entre otras cosas, y mediante la fijación de estas características a atributos
particulares se puede desarrollar un entorno gráfico completo, animaciones y
transiciones.

\subsubsection{JavaScript}

Como se describe en \textcite{JavaScript}, este lenguaje de programación es liviano e
interpretado, débilmente tipado y es comúnmente conocido como el lenguaje estándar
de los Script en las páginas Web; sin embargo, también puede ser utilizado en
otros ambientes como a nivel de servidor y en aplicaciones multiplataforma.
JavaScript es un lenguaje basado en prototipos, multiparadigma, dinámico y de
ejecución en un solo hilo (no aprovecha la paralelización de los múltiples
núcleos) y soporta todo tipo de estilo de programación.

Sus estándares han ido evolucionando con el tiempo y son conocidos como
`` ECMAScript Language Specification"; comúnmente JavaScript corre del lado del
cliente y es utilizado para diseñar como las páginas Web se ven o se comportan
ante algún evento especifico.

Al ser un lenguaje tan popular y común, posee una gran comunidad y una cantidad
muy significativa de Frameworks  y librerías, facilitando su aprendizaje
y maximizando la cantidad de cosas que permite hacer. Entre las más conocidas
están: Node.js para el servidor, React, Angular y Vue para el cliente, además
existen librerías como JQuerry que facilitan el lenguaje.

\subsubsection{React}

React es una librería de JavaScript de código abierto diseñada por Facebook y
lanzada por primera vez en 2013, como dice su documentación oficial \textcite{React}
está diseñada para la construcción de interfaces de usuario, tiene una filosofía
de página única (singlepage) pero puede ser utilizada para desarrollar aplicaciones
de múltiples páginas. Esta librería tiene la característica de ser declarativa
y basada en componentes.

Como se mencionó anteriormente, React hace sencilla la creación de interfaces
de usuario al permitir la combinación de JavaScript con XML-HTML en ``jsx"\  dando
la posibilidad de devolver y renderizar HTML desde funciones con componentes
lógicos. Además, es completamente modular por lo que se pueden desarrollar
individualmente los elementos de la interfaz y juntarlos con total facilidad,
estos elementos son llamados componentes y utilizan todas las propiedades de
JavaScript y la manipulación del DOM (Document Object Model) para mostrar
y actualizar la información cuando es necesario.


\subsubsection{Python}

En su página oficial, específicamente en sus preguntas frecuentes
\textcite{pythonDocs} se especifica que Python es un lenguaje de programación
interpretado, interactivo y orientado a objetos, este soporta muchos paradigmas
además del orientado a objetos, como lo son el procedimental y el funcional.
Python incorpora funcionalidades como módulos, excepciones, un tipado dinámico
y un muy alto nivel de dinamismo en los tipos de datos y clases, asimismo,
combina un poder computacional bastante alto con una muy clara sintaxis,
librerías, sistemas y compatibilidad con todos los sistemas operativos, además
de una facilidad para extenderse a los lenguajes C o C++.

Cabe resaltar que Python es completamente gratuito y tiene una gran capacidad,
dada su gran cantidad de librerías y Frameworks, para cumplir muchas tareas,
como lo son estadística, sistemas Web, ciencia de datos entre otros.

\subsubsection{Django}

Django es un Framework Web de alto nivel de Python el cual fomenta un
desarrollo rápido, limpio y pragmático. Como se especifica en su documentación
oficial,
\textcite{DjangoDoc} Fue creado para desarrolladores con experiencia que necesitan
un desarrollo rápido y completo en un proyecto Web, con la finalidad de
permitir al programador desarrollar la aplicación propiamente dicha, es
completamente gratuito y de código libre.

Entre las características de Django son su gran velocidad para el diseño, gran
cantidad de extras y tareas adicionales fácilmente configurables
(autenticación, administración, etc.), seguridad (previene errores comunes
como SQL injection y  cross-site scripting-requests), versatilidad y facilidad
de escalabilidad.

Cabe destacar que Django permite la inclusión de librerías de Python para
cualquier ámbito y existen algunas específicamente diseñadas para facilitar el
desarrollo con este Framework, por ejemplo, Django rest Framework, para la
creación de API desde el servidor de Django  y Django-cors-headers que permite
modificar y controlar el acceso de solicitudes a API desde clientes.

\subsubsection{Numpy}

Numpy es un paquete-extensión de Python diseñado para la computación
científica. Como se explica en su documentación oficial \textcite{Numpy} es una
herramienta de computación numérica que ofrece una gran cantidad de  funciones
matemáticas, generación de números aleatorios, rutinas algebraicas de álgebra
lineal, transformadas de Fourier, el tratado de arreglos N-dimensionales,
vectorización e indexación de los mismos, todo esto a un gran rendimiento y una
facilidad notable de su uso . Además de todo esto, esta es una herramienta
gratuita de código abierto.

\subsubsection{Pandas}

Pandas es otra herramienta dedicada al cálculo numérico, es una herramienta
rápida, poderosa, flexible y de fácil utilización, además de código abierto,
utilizada para el análisis y la manipulación de datos, es creada en Python.
Entre sus características destacan su velocidad y eficiencia con objetos de
tipo DataFrame para la manipulación de la información, sus herramientas para la
lectura y escritura de información en distintos formatos, entre los cuales
destacan los formatos CSV, Excel, SQL DataBase y el HDF5, facilidad para la
mutabilidad, alineamiento, filtración y eliminación, inserción y separación de
datos, y una muy alta optimización con puntos de código críticos escritos en
Cython o C. Además de esto, otras características adicionales son expuestas en
su documentación oficial \textcite{PandasDocs}

\subsubsection{SciPy}
SciPy es una librería-colección de algoritmos para la computación científica en
Python, como se explica en su sitio oficial \textcite{Scipy}, está desarrollado de
forma abierta en GitHub a través del consenso y el trabajo de la comunidad
científica de Python y de la comunidad de SciPy. Esta herramienta provee estructuras
de datos y
algoritmos para optimización, integración, interpolación, problemas algebraicos,
ecuaciones, ecuaciones diferenciales, estadística, entre otros. Está escrita
en lenguajes de medio-bajo nivel, como lo son Fortran, C y C++, haciendo sus
implementaciones bastante optimizadas y permitiendo el uso de código compilado
con la flexibilidad y facilidad de uso de Python.

\subsubsection{FastApi}
\textcite{FastApi} define FastApi como un Framework moderno y rápido, con alto
rendimiento, diseñado para
la construcción de API Web en Python, tiene una versión mínima de 3.6. Sus
características principales son:

\begin{itemize}
    \item Velocidad, equiparable con Node.js y Go haciéndolo
uno de los Frameworks de Python mas rápidos actualmente.
    \item Velocidad de escritura, permitiendo alcanzar un aumento en el desarrollo
        de hasta un 300\%.
    \item Facilidad, está desarrollado para ser intuitivo, fácil, permitiendo
        reutilización y a su vez eliminando hasta un 40\% de los errores (``bugs")
        inducidos por humanos.
    \item Utiliza estándares de código abierto como lo son ``OpenApi"" y ``JSON Schema"".
\end{itemize}

\subsubsection{Octave}
Octave es un software originalmente escrito por John W.Eaton y muchos otros,
esto es debido a que es un lenguaje gratuito y constantemente incorpora
funciones o correcciones hechas por la comunidad. Como se especifica en su
documentación oficial \textcite{octave}, es un lenguaje de programación de alto
nivel, orientado primordialmente a la realización de cálculos numéricos.
Permite el uso de una interfaz  o la terminal para resolver ecuaciones
lineales, no lineales, problemas numéricos además de conversiones y
transformadas matemáticas. Es completamente compatible con el lenguaje Matlab y
además puede ser utilizado como un lenguaje orientado a procesos.

Cabe resaltar que Octave posee una gran gama de librerías o módulos escritos en
C++, C, Fortran entre otros lenguajes, y es completamente gratuito y
redistribuible.


\subsubsection{Git}
Es un software de control de versiones diseñado por Linus Torvalds pensado para
la confiabilidad y compatibilidad del mantenimiento de versiones de aplicaciones
especialmente útil cuando estas tienen un gran número de versiones y archivos.
Como explica \textcite{Git}, Git es un sistema de control de versiones distribuido,
gratuito y de código abierto, diseñado para manejar eficientemente tanto pequeños
como grandes proyectos de forma rápida y eficiente. Se diferencia de otros sistemas
por características como ramificaciones locales baratas, en términos computacionales
espacio y velocidad, múltiples flujos de trabajo en áreas de ensamblaje convenientes
(Permite trabajar tanto en clientes gráficos como en la terminal).

\subsubsection{GitHub}
GitHub es una plataforma para el montaje de  código y el control de versiones
basada en Git. Se utiliza para la creación de código fuentes ya que permite y
facilita la interconexión entre los programadores, como se explica en \textcite{github}
 GitHub  permite incrementar la velocidad del desarrollador, además de asegurar
cada paso, automatizar los espacios y flujos de trabajo, facilitando de esta
forma los patrones de desarrollo e integración continua y permitiendo la creación
de equipos de trabajo sin el impedimento de la ubicación geográfica.

\subsubsection{Hosting}
Según \textcite{Hosting} un Hosting es el servicio que permite que un sitio
Web o dominio permanezca en internet, facilitando el guardar la información de un sitio
y acceder a este, además en este espacio se puede acceder a API o servidores.

Existen múltiples tipos de Hosting y múltiples servicios o empresas que se
dedican a prestar este servicio, algunas con características completamente pagas
y otras que permiten la utilización del sitio, con recursos mas restringidos de
formas gratuitas y la ampliación de estos e inclusión de funcionalidades extras
mediante la contratación de un plan con determinados recursos y políticas.

Entre las empresas mas conocidas se encuentra DigitalOcean  la cual, como se
explica en su sitio oficial \textcite{DigitalOcean} se dedica a alquilar servicios
de cómputo en la nube los cuales son robustos y escalables, además de contener
una muy buena documentación. DigitalOcean permite el agregar API además de
servidores mas complejos, los últimos mediante una implementación de una máquina
virtual con ``ubuntu server"\ como sistema, y facilitar de esta forma el acceso
a la información entre servicios, microservicios o servidores.

    \endgroup
    %
\subsection{Glosario}

\begin{itemize}
    \item motor eléctrico
    \item estator
    \item rotor
    \item torque
    \item potencia
    \item rodamiento
    \item Transferencia de potencia
    \item vibración
    \item sensor
    \item adquisición de datos
    \item ancho de banda
    \item acelerómetro
    \item aceleración
    \item posición
    \item velocidad
    \item frecuencia
    \item acelerómetro capacitivo
    \item acelerómetro piezoeléctrico
    \item acelerómetro pizorresistivo
    \item Sistema de segundo orden
    \item dispositivo microelectromecánicos
    \item masa sísmica
    \item amortiguamiento
    \item Circuitos Integrados
    \item galgas extensiométricas
    \item rango dinámico
    \item voltajes DC
    \item voltajes AC
    \item ruido
    \item linealidad
    \item ancho de banda
    \item cristal piezoeléctrico
    \item amplitud
    \item nivel offset
    \item señal analógica
    \item señal digital
    \item convertidor Analógico-Digital
    \item Dominio del tiempo
    \item Dominio de la frecuencia
    \item transformada de fourier
    \item componentes de una señal|
    \item Frecuencia natural
    \item herramienta computacional
    \item servidor
    \item pagina web
    \item Json
    \item Hipertexto
    \item internet
    \item intranet
    \item Web
    \item Pagina web
    \item Servidor
    \item javascript
    \item AJAX
    \item latencia
    \item Variable aleatoria
    \item Modelo estadístico
    \item Git
    \item Github
\end{itemize}


    %Faltan mas teorias, de lo usado y tal vez otro antescedente

    %Cuadros o graficas muy breve, que contenga la informacion
    %y dudas y permita recordar

%***************************************************
%**********  Capitulo 3  ***************************
%***************************************************
	\newpage
    \begingroup
    \let\clearpage\relax

    % naturaleza de la investigacion, Tipo de investigacion
	\thispagestyle{empty}

\section{MARCO METODOLÓGICO}

Como fue explicado en el primer capítulo, se pretende hacer una herramienta
computacional para el análisis de la vibración en motores eléctricos alimentada
mediante datos de una simulación digital. Partiendo de esto, se comenzará
con la definición de los siguientes aspectos.

\subsection{NATURALEZA DE LA INVESTIGACIÓN}

El presente trabajo es clasificado como Proyecto Especial, puesto que según
\textcite{Hernandez} lleva~a:

\begin{center}
    \parbox[ht]{13.5 cm}{Trabajos que lleven a creaciones tangibles,
    susceptibles de ser utilizadas como soluciones a problemas demostrados, o
    que respondan a necesidades e intereses de tipo cultural. Se incluyen en
    esta categoría los trabajos de elaboración de libros de texto y de
    materiales de apoyo educativo, el desarrollo de software, prototipos y de
    productos tecnológicos en general, así como también los de creación
    literaria y artística.}
\end{center}


\subsection{TIPO DE INVESTIGACIÓN}

De acuerdo con la clasificación el tipo de investigación de este trabajo se
cataloga como investigación aplicada, puesto que "persigue fines inmediatos y
concretos a través de la búsqueda de un nuevo conocimiento técnico con aplicación
inmediata a un problema determinado"\ \textcite{Velez}.


\subsection{PROCEDIMIENTO DE RECOLECCIÓN DE INFORMACIÓN}

\subsection{FASES DE LA INVESTIGACIÓN}

    %Procedimiento de recoleccion de datos e informacion
    %Fases de la investigacion


    %Procedimiento de recoleccion de datos e informacion
    %Fases de la investigacion

    
\subsection{METODOLOGÍA}
\subsubsection{Metodología de desarrollo de software}

Para la implementación de los distintos sistemas que componen la herramienta
computacional propuesta, se necesitar seguir una metodología de diseño de
software. Como la herramienta se clasifica como una aplicación WEB, las
metodologías utilizadas en el desarrollo de este tipo aplicaciones son
aplicables a este trabajo.

Las aplicaciones WEB cuentan con muchas similitudes a las de aplicaciones de
computadores personales, sin embargo una de sus principales diferencias es que
estas están en un estado de constante cambio por lo tanto
se pierde la noción de versiones, los cambios toman efectos de forma inmediata
y de forma gradual. Esta diferencia es producto de que su función principal es
la de transmitir información. Como está información cambia de forma constante
los requisitos de la aplicación suelen también ser variables. Todo esto unido
al hecho de que los cambios en la aplicación pueden realizarse con la
menor fricción posible, los cambios se pueden observar al volver a cargar la
página, hace que este tipo de aplicación  exista en un estado de evolución
continua. Según \cite{pressman2002} la evolución constante de las aplicaciones
WEB puede ser comparada con la jardinería, se hace un trabajo inicial el cual
seria equivalente a sembrar un jardín y una vez se tenga el sitio implementado,
se debe realizar el trabajo de mantenerlo que seria equivalente a regar y
abonar las plantas.

Debido a lo presentado anteriormente, para le elaboración de los diferentes
componentes de la herramienta se realizará un desarrollo de forma continua,
en donde la aplicación tendrá la capacidad de evolucionar si algún día
cambian los requisitos de la misma. Para esto se utilizara la ayuda de
\textbf{Git} como sistema de control de versiones, lo cual permitiría integrar
los diferentes cambios de forma mas eficaz y facilitar la cooperación a la
hora del desarrollo de la aplicación. Para la elaboración de los diferentes
componentes se usa también un proceso iterativo, cada vez que se implemente
un componente se comprueba el funcionamiento del mismo y se corrige, de ser
necesario.

Este modelo de implementación posee una estructura muy marcada, la cual se adapta
a las metas, usuarios finales y a la filosofía de navegación que se elija.
Según \cite{pressman2002} esta se puede desglosar como:

\begin{itemize}
    \item Diseño arquitectónico, el cual consiste en la definición de la
        estructura global, las plantillas y parte del patrón de diseño que se
        utilizará para estructurar la red. Específicamente, se trata de cimentar
        las bases para facilitar la creación del contenido y se estructura la
        forma de navegación por el sistema, además de los componentes que la
        integrarán.

    \item Diseño de navegación, en esta fase se eligen las rutas de navegación
        que permiten el acceso al contenido, además, se hacen las distinciones
        con respecto a los posibles usuarios que podrán acceder a la red y,
        por ende, las funcionalidades y/o permisos a los que tendrán acceso.

    \item Diseño de la interfaz, se refiere al diseño gráfico, estético y a
        las facilidades que se le dan a los usuarios, dado que esta es la primera
        impresión que se da, puede significar la diferencia entre el uso o no de
        la aplicación.

    \item Cabe destacar que, en la actualidad, se consideran mas fases que las
        mencionadas por Pressman, las dos mas resaltantes son la
        \textbf{Fase de programación} y la \textbf{Fase de Testeo}, aunque la
        última es nombrada como un proceso posterior al desarrollo. Estas dos
        fases son consideradas en la actualidad dados los constantes requerimientos
        de implementaciones adicionales o modificaciones a las ya existentes.
        Para permitir la integración de la mismas sin exponerse al colapso del
        sistema, se suelen utilizar \textbf{test} automatizados los cuales deben
        ser diseñados e implementados.

\end{itemize}

	
\subsection{RECURSOS}
\begin{itemize}	
	\item Computadoras portátiles.
	\item Documentación.
	\item Base de datos de la vibración en motores eléctricos.
\end{itemize}	
	%
\subsection{CRONOGRAMA DE ACTIVIDADES}

	Las actividades a realizar son las siguientes: 


\begin{enumerate}
	%************** primer objetivo *****************************
	\item Elegir y justificar el uso del lenguaje para el análisis y modelado estadístico.
	\item Elegir y justificar el uso del lenguaje para el análisis en frecuencia.
	\item Elegir y justificar el uso del lenguaje para el manejo y control de la información.
	\item Elegir y justificar el uso del lenguaje para la base de datos.
	\item Elegir y justificar el uso del lenguaje para el muestreo de la información.
	%************** segundo objetivo *****************************
	\item Estudiar la BBDD para obtener un modelo del comportamiento de la vibración en motores eléctricos.
	\item Programar el modelo y su interacción con el servidor.
	%************** tercer objetivo *****************************
	\item Definir y programar el modelo de la base de datos.
	\item Utilizar el programa del modelo estadístico anteriormente generado para llenar la BBDD
	%************** cuarto objetivo *****************************
	\item Crear el programa para dada la entrada de un array de datos(vibración discretizada), obtener la salida del estudio en frecuencia.
	\item Hacer un script con el punto anterior para permitir el llamado desde el servidor. 
	%************** quinto objetivo *****************************
	\item Programar los serializadores para llevar la información de la BBDD al nivel visual solicitado.
	\item Crear la pagina para la vista general.
	\item Crear la pagina para la vista especifica.
	\item Crear la pagina para la vista exhaustiva.
	%************** sexto objetivo *****************************
	\item Elaboración de la pagina web con los objetivos 12 al 15.
	%************** septimo objetivo *****************************
	\item Comprobar los resultados.
\end{enumerate}


	y el cronograma queda de la forma:\\

	\begin{figure}[htb]
		\centering
		\caption{Cronograma de actividades.}
		\label{cronograma}
		\includegraphics[width=17cm, height=20cm]{diagrama_grant.jpg}
	\end{figure}


    \endgroup


%***************************************************
%**********  Capitulo 4  ***************************
%***************************************************
	\newpage
    \begingroup
    \let\clearpage\relax

    \thispagestyle{empty}

\section{ANÁLISIS Y DISCUSIÓN DE RESULTADOS}

\subsection{PROCEDIMIENTO DE RECOLECCIÓN DE INFORMACIÓN}

\subsection{ELECCIÓN DE LAS HERRAMIENTAS Y LENGUAJES}
    Dada la gran cantidad de actividades y los requerimientos que tenian las mismas,
    se hizo un estudio de los lenguajes y herramientas mas utilizados en
    la actualidad para tareas similares y junto a esto, se establecieron
    ciertos criterios y requerimientos minimos para cada tarea a realizar, de
    esta forma, se tomaron las siguientes decisiones:

    \subsubsection{Servidor}
    La decision del servidor se puede a su vez dividir de acuerdo a los
    microservicios realizados, es decir, el servidor dedicado a sensorica y
    el servidor Web. En ambos casos se manejaron los mismos lenguajes, asimismo, debido a
    las caracteristicas variantes entre estos y la posibilidad, dada la arquitectura
    distribuida utilizada, de utilizar multiples herramientas se opto por utilizar
    el que mejor cumplia los requerimientos establecidos; como se observa en la
    tabla \ref{tab:LenguajesServidor}, se comparan los lenguajes Golang, Python
    con framework Django y NodeJs, esto es debido a que los servidores siguen un
    sistema Web y es mas facil el desarrollo en estos lenguajes especializados que
    en sistemas como Apache o Nginx mas robustos pero generales.

    \begin{table}[ht]
        \begin{center}
            Distintas opciones manejadas al momento de desarrollar los servidores.\\

            \vspace{0.3cm}
            \begin{tabular}{|c|c|c|c|}
                \hline
                Caracteristica              & Golang & Django & NodeJs\\\hline
                Velocidad                   & 10    & 7     &   8   \\\hline
                Facilidad de implementacion & 8     & 9     &  7\\\hline
                Robustez                    & 6     & 9     & 7 \\\hline
                Escalabilidad               & 10    & 10    & 9 \\\hline
                Paralelismo                 & 10    & -     & 2 \\\hline
                Facilidades a conexion http2& 9     &7      & 7 \\
                \hline
            \end{tabular}
        \end{center}
        \caption[Comparativa de posibles lenguajes nivel servidor]{Comparacion entre
        los posibles lenguajes utilizables para servidores}
        \label{tab:LenguajesServidor}
    \end{table}

    Para el microservicio de sensorica, las caracteristicas mas importantes furon
    el paralelismo y la facilidad de implementar una conexion http2, por la
    necesidad de una conexion full duplex, y por estos motivos Golang fue un claro
    ganador, dado que es un lenguaje diseñado para microservicios y paralelismo y
    el uso de librerias facilitan increiblemente el establecer una conexion http2.

    Por otro lado, en terminos del servidor Web, las necesidades giraban entorno
    a la facilidad de implementacion, robustez y escalabilidad que otorgaban los
    lenguajes siendo en este caso Python + Django el ganador indiscutible.


    Cabe resaltar que paralelismo hace referencia al mejor uso posible de los
    recursos de hardware del servidor por la gran cantidad de conexiones
    continuas y tareas adicionales que deben de ser realizadas y estar
    establecidas, no a la cantidad de peticiones que recibe el servidor.

    \subsubsection{Cliente}

    En terminos de clientes tambien se debe hacer diferencia entre los 2 sistemas
    creados, cliente de sensorica y cliente Web. El cliente de sensorica se
    implemento en Golang por las mismas razones dadas para el servidor de sensorica.

    Por otro lado, en la actualidad los clientes Webs estan compuestos casi en
    su totalidad de una combinacion de HTML-CSS-JavaScript, existen casos en los
    que no se usa JavaScript o clientes escritos en otros lenguajes e interpretados
    en la web; de esta forma, la decision esta en que framework de javascript se
    va a utilizar, existen una gran variedad que cubre desde JavaScript ``vanilla",
    sin framework y JQuerry, que otorgan interactividad pero se complica rapidamente
    cuando la aplicacion crece, hasta los 3 gigantes en la actualidad, ReactJs,
    Angular y VueJs, que facilitan y agilizan increiblemente el proceso de desarrollo.
    Como se explica en la tabla \ref{tab:LenguajesCliente}, La razon mas importante
    en la toma de la decision fueron conocimientos previos con ReactJs.


    \begin{table}[ht]
        \begin{center}
            Distintas opciones manejadas al momento de desarrollar los clientes Webs.\\

            \vspace{0.3cm}
            \begin{tabular}{|c|c|c|c|}
                \hline
                Caracteristica              & JavaScript & React & Otros JsFrameworks\\\hline
                Facilidad de implementacion & 1         & 7     &  7\\\hline
                Conocimientos previos       & 6         & 5     &  - \\\hline
                \hline
            \end{tabular}
        \end{center}
        \caption[Comparativa de posibles lenguajes nivel cliente Web]{Comparacion entre
        los posibles lenguajes utilizables para clientes Webs}
        \label{tab:LenguajesCliente}
    \end{table}



    \subsubsection{Modelo estadistico}

    \subsubsection{Analisis en frecuencia}
%1 por objetivo
    % LENGUAJES Y HERRAMIENTAS UTILIZADOS
    \subsection{IMPLEMENTACIÓN DEL MODELO ESTADÍSTICO}

El diseño del modelo comenzó con la limpieza de los datos y la selección de las
variables de interés de la base de datos proporcionada. Las variables de
interés fueron la velocidad horizontal, la velocidad vertical, la aceleración,
la potencia del equipo, el tipo de equipo y el código del mismo. Una vez
obtenida la data se le realizó un análisis exploratorio, el cual
consistió principalmente en la elaboración de gráficas y sacar estadísticos de
las variables de interés de lo cual se obtuvieron las observaciones que se incluyen
a continuación.

Las variables de velocidad horizontal, velocidad vertical y aceleración son
siempre positivas, todas muestran en su distribución una asimetría con
tendencia hacia la izquierda, lo cual las hace similar a distribuciones como la
exponencial, la weibull o la log-normal. Cuando se calcula la correlación entre
las variables velocidad horizontal y vertical obtenemos una
correlación de Pearson del $56.71\%$, el cuadrado de este valor o $R^2$ (también
conocido como coeficiente de determinación) es del $32.16\%$ y se puede
interpretar como que el valor de la velocidad horizontal predice $32\%$ del valor
de la velocidad vertical, estos valores son muy elevados para considerar las
variables como independientes. En el caso de la aceleración las correlaciones
no fueron significativas.

Debido a la correlación considerable entre las dos velocidades no era posible
modelar las variables como independientes pero tampoco era tan elevada para
realizar una regresión lineal, por lo tanto se recurrió a buscar un cambio de
variables en el cual desapareciera la dependencia entre las dos variables
aleatorias. Como la velocidad es un vector y se tienen sus componentes,
un cambio de variables lógico es el de coordenadas polares ya que como es
una transformación biyectiva no se pierde información aplicandola y, a
su vez, es fácil de interpretar. La correlación entre las nuevas variables
magnitud y ángulo fue del $-25.18\%$ y un valor de $R^2$ del $6.34\%$ lo cual para
efectos prácticos se puede considerar como dos variables aleatorias
independientes. Similarmente con las velocidades con componentes cartesianos la
correlación con la aceleración es no significante.

Una vez obtenidas las tres variables de interés, para modelar se buscó la
distribución que mejor se aproxima a los datos, la escogencia de la
distribución de probabilidad por lo general no es un proceso cuantitativo sino
que el investigador conociendo en profundidad el problema escoge entre un grupo
de distribuciones conocidas, cuál se adapta mejor al problema planteado, pero
debido a que hoy en día existen herramientas computacionales para la
estadística este proceso se puede hacer un poco más cuantitativo. Para la
selección de los modelos de cada variable se iteró entre 84 distribuciones
pertenecientes a Scipy y se aplicó a cada una la prueba de Kolmogorov–Smirnov,
se filtraron aquellas distribuciones con valores P mayores a $5\%$ y, finalmente, se
ordenaron de mayor a menor, debido a que la hipótesis nula de la
prueba de Kolmogorov-Smirnov es que si las distribuciones poseen la misma
distribución un valor elevado de P indica que la distribución, se aproxima al
valor teórico.

El modelo estadístico seleccionado fue el siguiente, se tienen 3
variables independientes: magnitud de la velocidad, ángulo de la velocidad y
aceleración. Para la magnitud de la velocidad se seleccionó una distribución
Burr de tipo III debido a que es una distribución netamente positiva y la
prueba de Kolmogorov dio un valor P del $78,01\%$ siendo este el segundo valor más
alto y el primero que cumple la condición de ser positiva siempre, en el caso
del ángulo múltiples distribuciones satisfacen pero como la distribución normal
tenía un valor P del $93.79\%$ y es una distribución conocida, se seleccionó esta;
finalmente, en el caso de la aceleración la distribución de Burr tipo III poseía
un valor P de $99.28\%$ si bien había otras distribuciones con valores ligeramente
mayor se optó por escoger esta porque ya la magnitud de la velocidad seguía esta
distribución y simplifica la implementación.

En el caso de la vista exhaustiva se utilizó una mezcla entre un modelo de la
señal no estadístico y el modelo estadístico previamente seleccionado de tal
forma que se pueda obtener una gran cantidad de datos para poder realizar el
análisis de frecuencia pero a la vez conservar la coherencia con el resto de
mediciones; para esto se ajustó la amplitud de la señal de acuerdo al valor
esperado según el modelo de la aceleración y para expresar el ruido de la
medición a los parámetros del modelo se les suma una señal de ruido de tipo
gaussiana.

Una vez obtenido el modelo, se implementó una API Web usando FastAPI con dos
endpoints, uno utilizado para la vista general y específica y otro para la
vista exhaustiva. Ambos endpoints solicitan el nivel de daño el cual es un
número entre 0 y 10 que indica el grado de daño del motor y solicitan el id del
motor, el cual es un número entero  que identifica el motor. Esta API se encuentra
en un Host independiente en Digital Ocean y la accede el microservicio de Go
para obtener los valores del modelo.

    % MODELO ESTADISTICO DE LA VIBRACION
    
\subsection{IMPLEMENTACIÓN DEL ANALISIS EN FRECUENCIA}

    % ANALISIS EN FRECUENCIA
    

\subsection{IMPLEMENTACIÓN DE LA BASE DE DATOS}

    Como se ha mencionado anteriormente, se ha escogido una base de datos no
    relacional, específicamente MongoDB, dentro de esta se han desarrollado
    una base de datos llamada ``tesis"\   y dos colecciones llamadas ``MotorData"\  y
    ``MotoresInDB"\ ; la primera colección se encarga de almacenar toda la información
    enviada por los sensores (en este caso el cliente simulado con los datos
    proporcionados por el modelo estadístico) y la segunda colección contiene una
    lista con los identificadores únicos de cada motor que tenga al menos un documento
    en la colección ``MotorData"\ , es decir, hay información registrada de su actividad.

    Cada colección tiene una estructura fija definida para facilitar el manejo y
    la consistencia de los documentos, esta es similar un json y sus campos
    están explicados en las tablas \ref{tab:MotorDatabson} para ``MotorData"\  y
    \ref{tab:MotorInDBbson} para ``MotorInDB"\ .

    \begin{table}[ht]
        \begin{center}
            Tabla de la estructura seguida para la colección ``MotorData"\ .\\

            \vspace{0.3cm}
            \begin{tabular}{|c|c|p{9cm}|}
                \hline
                Elemento        & tipo de dato & Descripción \\\hline\hline
                %
                $\_$id      & []bytes  & Elemento utilizado por MongoDB para
                identificar y facilitar la búsqueda de los documentos\\\hline
                %
                IdMotor         & string   & Identificador único del motor.\\\hline
                %
                Características & string   & Descripción e información del motor.\\\hline
                %
                IdSensor        & []uint64 & lista de los identificadores-sensores
                que tiene conectado este motor.\\\hline
                %
                Data            & []DataSensor & lista en forma de sub colección
                que contiene los resultados del sensor.\\\hline
                %
                Time            & time  & Estampa de tiempo, fecha y hora de la muestra.\\\hline
                \hline
                \multicolumn{3}{|c|}{Sub colección  ``DataSensor"\ }\\\hline\hline
                %
                IdSensorData & uint64 & Identificador único del sensor que tomo la muestra.\\\hline
                Aceleración  & float64 & Muestra de aceleración medida en g .\\\hline
                VelocidadX & float64 & Muestra de velocidad en el eje X.\\\hline
                VelocidadY & float64 & Muestra de velocidad en el eje Y.\\\hline
                VelocidadZ & float64 & Muestra de velocidad en el eje Z.\\
                \hline
            \end{tabular}
        \end{center}
        \caption[Estructura de MotorData]{Estructura de la colección MotorData}
        \label{tab:MotorDatabson}
    \end{table}

\vspace{1cm}


    \begin{table}[ht]
        \begin{center}
            Tabla de la estructura seguida para la colección ``MotorInDB"\ \\
            \vspace{0.3cm}
            \begin{tabular}{|c|c|p{11cm}|}
                \hline
                Elemento & tipo     & Descripción \\\hline\hline
                %
                \_id      & []bytes  & Elemento utilizado por MongoDB para
                identificar y facilitar la búsqueda de los documentos\\\hline
                %
                IdMotor  & []string & Lista que contiene todos los identificadores
                únicos de los motores. Facilita búsqueda e implementación de los
                clientes Webs\\\hline
            \end{tabular}
        \end{center}
        \caption[Estructura de MotorInDB]{Estructura de la colección MotorInDB}
        \label{tab:MotorInDBbson}
    \end{table}

    Cabe relatar que ``uint64"\  hace referencia a un número natural de 64 bits
    y es usado en los identificadores de sensores ya que permite el uso de 64
    bits (16 nibbles (gripos de cuatro)) los cuales son codificados de
    la forma expuesta en la tabla \ref{tab:CodIdSensor}, para
    poder transmitir mas información y facilitar la escalabilidad del sistema en
    un futuro.
    Asimismo, se utiliza
    ``float64"\ para representar  a un número racional representado como punto
    flotante de 64 bits, Esta es una unidad común porque permite maximizar la
    precisión en la medida.

    \begin{table}[ht]
        \begin{center}
            Tabla de la estructura seguida para ``IdSensor"\  \\

            \vspace{0.3cm}
            \begin{tabular}{|c|c|c|c|}
                \hline
                Tipo de sensor & Reservado& Ubicación   & Serial \\\hline
                $ B_{15} $ & $ B_{14}B_{13} $ &$ B_{12} $  &  $ B_{11}\cdots B_{0} $\\
                \hline
            \end{tabular}

            \vspace{0.3cm}
            \begin{tabular}{|c|p{13cm}|}
                \hline
                Campo       & Descripción
                \\\hline\hline
                tipo        & tipo de sensor usado, Acelerómetro, Temperatura, etc.
                Con $0000b$ siendo Acelerómetro y 15 posibilidades adicionales.
                \\\hline
                Reservado   & No se utilizan, son 0x00 siempre y se reservan para
                posibles expansiones y/o necesidades.
                \\\hline
                Ubicación   & Posición con respecto al motor y acoples. Con:
                $0000b$ Lado con carga, $ 0001b$ Lado libre, $ 0010b\cdots1111b $
                disponibles para acoples y chumaceras.
                \\\hline
                Serial      & número de fabricación del sensor, desde 0 hasta $2^{48}$.
                \\\hline
            \end{tabular}
        \end{center}
        \caption[Estructura IdSensor]{Estructura seguida en el identificador de
        los sensores para permitir y facilitar la escalabilidad del sistema}
        \label{tab:CodIdSensor}
    \end{table}


\subsection{LLENADO DE LA BASE DE DATOS}
    Para el llenado con la informacion se desarrollo un script en Golang,
    llamado ``LlenarBBDD"\ , este
    se encarga de hacer las peticiones al modelo y a la base de datos de forma
    automatica, realiza 120 peticiones, equivalentes a 120 dias, con un aumento
    progresivo del nivel de daño cada 40 dias-peticiones. Al ser un script funciona
    mediante la terminal y utiliza las banderas expuestas en la tabla
    \ref{tab:BanderasLLenadoBBDD} para tomar la informacion.

\begin{table}[ht]
        \begin{center}
            Banderas utilizadas en el llamado del script LlenarBBDD \\

            \vspace{0.3cm}
            \begin{tabular}{|c|p{7cm}|p{5cm}|}
                \hline
                Bandera & Descripcion & Requerimiento \\\hline
                -i & Para especificar el Id del Motor. & Requerida.\\\hline
                -d & Para indicar el nivel de daño  & Opcional, por defecto 1.\\\hline
                -c & Indicar las caracteristicas e informacion del motor & Opcional, por defecto ``Estado del motor".\\\hline
                -s1& Especificar el Id del sensor 1 &Requerida.\\\hline
                -s2& Especificar el Id del sensor 2 &Requerida.\\\hline
                -s3& Especificar el Id del sensor 3 & Opcional, por defecto se omite.\\\hline
                -s4& Especificar el Id del sensor 4  &Opcional, por defecto se omite.\\\hline
                -r & Reiniciar la base de datos si su vaulor es ``true"& opcional, por defecto false.
                \\\hline
            \end{tabular}
        \end{center}
        \caption[Banderas Script para el llenado de la BBDD]{
        Banderas utilizables al momento de llamar el script para llenar la informacion
        de la base de datos}
        \label{tab:BanderasLLenadoBBDD}
    \end{table}


\subsection{IMPLEMENTACIÓN DE LOS SERVIDORES}

    Los servidores son necesarios para recolectar la información de los sensores,
    establecer una comunicación full dúplex la cual permita obtener información
    a tiempo real del comportamiento del sensor, además de permitir toda la
    interacción web que da la visibilidad. Dado que estas tareas pueden ser
    separadas y manejadas en forma de APIs se crearon 2 servidores, permitiendo
    de esta forma la división de la carga de trabajo y por ende disminuyendo los
    requerimientos mínimos del equipo en donde se monta cada servidor individual.
    Así mismo, esto permite una mayor escalabilidad y paralelismo, dado que en el
    caso de ser necesaria una ampliación en la capacidad de cómputo se puede colocar
    otro equipo en vez de aumentar las capacidades del equipo ya existente. Este
    hecho permite disminuir los costos sustancialmente.

    La implementación de estos 2 servidores da origen a un \textbf{servidor dedicado
    a sensorica}, desarrollado como un micro-servicio en el lenguaje de programación
    Go (también conocido como Golang) por las razones previamente expuestas, y
    a un \textbf{servidor dedicado al tratamiento web} desarrollado con una
    combinación de Python y el framework Django para el Backend y html-css-javascript
    con el framework de React en un paradigma multipaginas de renderizado desde
    el servidor para el frontend-cliente Web.

    \subsubsection{Servidor dedicado a Sensorica}

    Este servidor fue realizado en Go por los motivos expuestos anteriormente y
    cumple la función de micro-servicio, se encarga de la recolección y comunicación
    con la red de sensorica, la cual es implementada por el cliente de la simulación
    y el modelo estadístico, así mismo envía la información relacionada con la
    vista exhaustiva (para esta se requieren mediciones a tiempo real y por esto
    este servidor tiene una conexión full duplex con los sensores).

    Como se observa en la tabla \ref{tab:ServerSensorica}

    \begin{table}[ht]
        \begin{center}
            Tabla de las funcionalidades del servidor dedicado a sensorica   \\

            \vspace{0.3cm}
            \begin{tabular}{|p{5cm}|p{10cm}|}
                \hline
                Dirección       & Tarea realizada
                \\\hline\hline
                DireccionIP/sensormessage &
                Se encarga del comportamiento, acceso
                e intercambio de información sensor-servidor. Es una comunicación
                bidireccional con http2 (https) y se intercambian por el canal
                establecido tanto la información de medición diaria, (después es
                subida a la base de datos) como la
                información de medición exhaustiva (es enviada al cliente que
                la solicito).

                \\\hline
                DireccionIP/exhaustive \textbf{?idMotor=identificador}   &
                Se encarga de solicitar la información para la vista exhaustiva,
                el identificador único del motor que se solicitara la data
                es enviado por el url (?idMotor=identificador) .
                \\\hline
            \end{tabular}

            \vspace{0.3cm}
            Cabe resaltar que ``DireccionIP"\ hace referencia a la dirección en la
            que sera montado (Host) el sitio, en caso de un ambiente local, por
            ejemplo, es ``localhost:8080" (este es el usado para las pruebas,
            cuando se cambia a producción se modifica por el del host contratado)
        \end{center}
        \caption[Funciones Servidor Sensorica]{ Relación de punto de acceso y
        funcionalidad implementada en el servidor de sensorica}
        \label{tab:ServerSensorica}
    \end{table}


    El primer ``Endpoint"\ (DireccionIP/sensormessage) realiza las siguientes tareas:

    \begin{itemize}
        \item Establecer una conexión http2 con los solicitantes, para esto se intercambian,
            además de los paquetes e información necesarios para establecer el protocolo,
            unos mensajes que permiten identificar el motor y la cadena de sensores
            correspondientes a esta información.
            %
        \item Posteriormente se revisa si el motor
            tiene una conexión activa (no debe de existir por unicidad de la información)
            y si ya se ha recibido información de este motor previamente, de no ser así,
            se agrega a la lista de motores de la que se posee información.
            %
        \item En este punto el servidor se dedica a escuchar la llegada o solicitud
            de información. Esta puede ser del sensor con el que se estableció
            conexión (mediante un canal interno) o  de la solicitud de información
            para una vista exhaustiva. Para cada caso se hace lo siguiente:
            \begin{itemize}
                \item[*] Si es información, se verifica que venga en formado valido
                    y se sube a la base de datos.
                \item[*] Si es una solicitud de información, se verifica que
                    la conexión con el motor sea la indicada y se solicita la
                    información, se espera la respuesta y se envía por el mismo
                    canal interno. Esta solicitud es hecha por el segundo Endpoint.
                \item[*] No se dio ninguno de los casos, entonces el cliente se
                    desconectara. Se procede a eliminar la conexión y la lista
                    mediante un procedimiento de cierre de conexión.
            \end{itemize}
            %
        \item Finalmente, existe un procedimiento de cierre de conexión ya sea
            por solicitud del cliente o por un error ocurrido.
    \end{itemize}


    El segundo ``Endpoint"\ (DireccionIP/exhaustive\textbf{?idMotor=identificador})
    realiza las siguientes tareas:

    \begin{itemize}
        \item Decodifica el URL enviado para obtener el parámetro (idMotor) y
            comprueba su existencia. En caso de error se envía un mensaje
            de petición incorrecta.
        %
        \item Se comprueba que el motor solicitado exista en las conexiones
            actuales. Es importante resaltar que se refiere a la conexión
            bidireccional, ya que de caso contrario no se puede obtener información
            a tiempo real. Por esto, si la conexión es inexistente, se envía un
            mensaje de petición incorrecta, no se puede conectar al motor.
        %
        \item Se solicita por un canal interno al otro Endpoint la información
            deseada y se espera su respuesta.
        %
        \item Se envía la respuesta con estado de creado y la información.
    \end{itemize}

    \subsubsection{Servidor web}
    Este servidor fue realizado en Python con el Framework Django
    por los motivos expuestos anteriormente y cumple la función de servidor Web,
    es el servidor principal ya que se encarga de enviar y solicitar informacion
    para crear la visualizacion del cliente, ademas de estudiar y crear las graficas
    y enviar informacion al cliente web para analisis.
    Sus conecciones son por el protoclo http, y https para la pedicion de informacion
    exhaustiva con el microservicio de sensorica. y realiza las tareas expuestas
    en la tabla \ref{tab:serWeb}

    \begin{table}[ht]
        \begin{center}
            Tabla de las funcionalidades del servidor Web   \\

            \vspace{0.3cm}
            \begin{tabular}{|p{5cm}|p{10cm}|}
                \hline
                Dirección       & Tarea realizada
                \\\hline\hline
                DireccionIP/ &
                Enviar el HTML necesario, mediante un Template de Django,
                para mostrar la vista general en el navegador.
                \\\hline
                DireccionIP/especifica \textbf{?IdMotor=identificador}&
                Enviar el HTML necesario, mediante un Template de Django,
                para mostrar la vista especifica en el navegador.
                \\\hline
                DireccionIP/exhaustiva \textbf{?IdMotor=identificador}  &
                Enviar el HTML necesario, mediante un Template de Django,
                para mostrar la vista exhaustiva en el navegador.
                \\\hline
                DireccionIP/ static/... &
                Es una funcionalidad del servidor la cual permite el envio de
                archivos estaticos por solicitudes externas, ya sea por un
                requerimiento del HTML o del JavaScript.
                \\\hline
                DireccionIP/ get\_data\_motores &
                API que entrega  una lista, formato JSON,
                con la ultima medicion almacenada de cada sensor en la base de
                datos.
                \\\hline
                DireccionIP/ get\_list\_motores &
                API que entrega una lista, formato JSON,
                con todos los motores de los que se
                dispone informacion en la base de datos.
                \\\hline
                DireccionIP/get\_especifica \textbf{?IdMotor=identificador}&
                API que se encarga de devolver un JSON con toda la informacion
                en la base de datos referente al motor con el Id pasado en el
                Url, ademas de las direcciones, en el mismo servidor que seran
                entregadas mediante el endpoint ``DireccionIP/static/... ", que
                ocupan las graficas historicas referente a ese motor.
                \\\hline
                DireccionIP/get\_exhaustiva \textbf{?IdMotor=identificador}&
                API que se encarga de devolver un JSON con toda la informacion
                en la base de datos referente al motor con el Id pasado en el
                Url, ademas de las direcciones, en el mismo servidor que seran
                entregadas mediante el endpoint ``DireccionIP/static/... ", que
                ocupan las graficas historicas y del estudio de Fourier
                referente a ese motor.
                \\\hline
            \end{tabular}

            \vspace{0.3cm}
            Cabe resaltar que ``DireccionIP"\ hace referencia a la dirección en la
            que sera montado (Host) el sitio, en caso de un ambiente local, por
            ejemplo, es ``localhost:8000" (este es el usado para las pruebas,
            cuando se cambia a producción se modifica por el del host contratado)
        \end{center}
        \caption[Funciones Servidor Web]{ Relación de punto de acceso y
        funcionalidad implementada en el servidor Web }
        \label{tab:serWeb}
    \end{table}

    Cada endpoint realiza las siguientes tareas



    % SERVIDORES, CONEXION CON SENSORES, BASES DE DATOS,
    
\subsection{IMPLEMENTACIÓN DE LOS CLIENTES}

Un cliente puede ser definido como un equipo o software que se conecta a un
servidor para obtener un beneficio, sea por poder de computo, para obtener
una información dada o comunicarse con un programa que se ejecuta en el lado
del servidor. Siguiendo esta definición, se crearon 2 clientes, uno para
facilitar la inserción de datos en la simulación del sensor, y el cliente Web,
el cual permite manejar el análisis y la monitorización de los sensores.

\subsubsection{Cliente de Sensorica}

Este fue elaborado en Go para facilitar la interconexión con el servidor y
de igual forma aprovechar el paralelismo y la multiplataforma que el
lenguaje ofrece. Se puede subdividir en 3 acciones fundamentales:

\begin{itemize}
    \item GUI: Es la interfaz gráfica de usuario, utiliza el Framework Fyne por
        facilidades
        de diseño y es una ventana que permite ingresar datos equivalentes
        a las características del motor y conectar con el servidor (además de
        especificar en qué dirección está el motor), y posee una ventana de
        \textbf{log}  en la cual se comunica al usuario acciones como la conexión exitosa y
        el envío de información al servidor. esta puede ser vista
        en la figura \ref{img:fyne} y como se observa, permite especificar:
        \begin{enumerate}
            \item Dirección Ip: Lugar a conectar.
            \item Id Motor: Identificador único del motor.
            \item Potencia: Información adicional, opcional.
            \item Nivel de daño: numero entre 1 y 10 que determina los posibles
                valores del modelo.
            \item Información: Información adicional, pensado para comentarios,
                opcional.
            \item Sensores: lista de 1 a 5 posibles sensores, que representan
                los identificadores únicos que tienen los sensores asociados
                al motor.
        \end{enumerate}

        Cabe resaltar que a ser una GUI es el proceso principal, las demás
        tareas son realizadas de forma paralela.

    \item Conexión al servidor: Es un protocolo que ocurre cada vez que se presiona
        el botón de \textbf{conectar}, se encarga de intercambiar información con el servidor
        para poder conectarse (bajo el protocolo Https) y envía los datos de
        qué motor se va a conectar (simular) y qué sensores tiene asociados, espera
        una aprobación de conexión (para que no exista un motor con el mismo id enviando
        información), obtiene las características especificadas en la GUI y
        comienza un proceso de envió-recibo de información.
        Este es un proceso
        que se ejecuta paralelamente a la GUI y se inicia cada vez que se
        presiona el botón de \textbf{conectar}, Si había un proceso previo y se vuelve a
        presionar \textbf{conectar} finaliza el anterior y comienza uno nuevo.

    \item Comunicación con el servidor: Es un proceso de envió bidireccional
        de información, esta rutina se inicia cuando la conexión al servidor
        es completada exitosamente y se subdivide en 3 procesos que, a su
        vez, corren paralelamente (se toma el proceso de conexión al servidor
        y se crean 2 hijos para un total de 4 procesos paralelizados si se
        incluye la GUI). Estos procesos son:

        \begin{enumerate}
            \item \textbf{timer}, Es un proceso que se encarga de cronometrar
                cada cuánto
                se va a mandar un mensaje de información con los datos
                correspondientes a una medición normal al servidor.

            \item \textbf{listen}, Es un proceso que se encarga de verificar si
                hay una
                solicitud de información, ya sea del subproceso \textbf{imer} (como un
                mensaje normal) o del servidor para solicitar información de la
                vista exhaustiva (la cantidad de información enviada es sustancialmente
                diferente) o para la terminación del proceso e informa a \textbf{handler}
                qué se debe hacer.

            \item \textbf{handler}, Se encarga de realizar la tarea pedida por \textbf{listen},
                al enviar la información solicitada al servidor, enviando 2 mensajes;
                uno con el tipo de información que se envía y otro con la información.
                Esto fue establecido como una especie de protocolo para dar mayor
                seguridad y a la vez facilitar el intercambio de información con
                el servidor.
        \end{enumerate}

        Cabe resaltar que estos procesos son asíncronos y no sufren prelaciones
        entre ellos.
\end{itemize}

    \begin{figure}[htb]
		\centering
        \caption{GUI del cliente de sensorica}
        \includegraphics[width=\linewidth]{clients/ClienteSensoricaSinConectar.png}
        Interfaz de usuario para la conexion del cliente de sensorica. \label{img:fyne}
	\end{figure}

\subsubsection{Cliente Web}

Este cliente es el mas conocido, ya que es el que permite que se muestre la
información en el navegador Web. Está constituido por las vistas general,
específica y exhaustiva, cada una de ellas representa una página Web separada y
todas fueron construidas utilizando el Framework de JavaScript \textbf{React}
y con HTML específicamente un Template de Django y CSS para dar estructura
básica y estilo respectivamente.

Se optó por utilizar un
estilo multi páginas con renderizado de lado servidor (específicamente del
servidor Web) por la necesidad de los cálculos avanzados y gráficas que
se deben realizar, además de los llamados importantes e interconexiones con la
API del servidor de Go y de las BBDD. Esto difiere con el paradigma tradicional de
React (monopágina de renderizado en servidor) pero permite optimizar recursos y
facilita expansiones a futuro.

Su estructura viene dada por las 3 páginas o sub aplicaciones que permiten:

\begin{itemize}
    \item General: conocer el estado general de un grupo de motores, indicando
        en código de colores (verde, amarillo, rojo) el nivel de daño que posee
        un motor. Este nivel es determinado por las muestras mas actuales de
        la información del motor y unos valores parámetros proporcionados en
        conjunto con los datos de los cuales se elaboró el modelo estadístico.

        Cabe resaltar que estos valores son una extrapolación empírica de la
        vibración en una planta específica, es configurable y puede variar dadas
        las características propias de cada instalación. Esto es debido a que
        las bases utilizadas en la instalación, los soportes, entre otros factores
        \textbf{causados por ignorar las normas de instalación} afectan las medidas.

    \item Específica:  permite el estudio del estado de un motor específico,
        es enviado el parámetro que lo identifica en el Url, al servidor.
        Permite observar
        una gráfica histórica y una tabla exportable a excel de sus características y
        evolución en el tiempo, siempre y cuando se tenga medición de ese periodo
        en la base de datos, asimismo permite solicitar la vista exhaustiva del
        mismo motor si es requerido un nivel mayor de análisis.

    \item Exhaustiva: esta vista incluye todo lo anterior de la vista específica,
        con la diferencia de que permite regresar al menú general en vez de hacer
        una vista exhaustiva; además, realiza un análisis en frecuencia
        del estado a tiempo real del motor. Este análisis se puede realizar si
        se tiene acceso en tiempo real con el motor, es decir,
        el servidor de \textbf{sensorica} debe tener una conexión con un sensor que se
        encargue de monitorear el respectivo motor permitiéndole así solicitar
        la información necesaria para hacer el análisis.

        Cabe resaltar que esta acción es tratada como una vista aparte ya que
        tiene un peso computacional relativamente alto asociado y, por ende,
        además de consumir recursos tiene una duración de carga de algunos
        segundos que deteriora la experiencia de usuario, por esto
        obtiene solamente por una solicitud explicita.
\end{itemize}





    
\subsection{COMPROBACIÓN DE LOS RESULTADOS}

Como se ha mencionado anteriormente, el trabajo consta de una gran cantidad de
puntos que han sido trabajados de forma modular con la finalidad
de maximizar la escalabilidad y de facilitar al máximo posible el realizar pruebas
a los módulos y a los acoples realizados. La comprobación debe de
 tomar en consideración 2 factores, los resultados técnicos, es decir
funcionalidades de la aplicación las cuales, como se mencionó anteriormente, no se
automatizaron por factores tiempo y cantidad de pruebas y, los resultados finales,
es decir la capacidad de la herramienta de facilitar el trabajo, mediante
las páginas generales, específicas y exhaustivas.

\subsubsection{Comprobación de resultados a nivel técnico}

Las fallas internas de los servidores, o de procesos de comunicación
con la base de datos, conllevan a la finalización de la ejecución del proceso
(aplicación) correspondiente; esto se hace para poder reiniciar el sistema y
agregar el error o falla catastrófica a un archivo (log) que sirve para seguir el
comportamiento del servidor.

En el caso del servidor de sensorica, estos errores son manejados de forma manual
y se utiliza la característica de Go \textbf{``panic"} para detener la ejecución, mas
específicamente un paquete de la librería estándar \textbf{``log"} con la función
\textbf{``log.Faltalf"}.
Por otro lado, el Framework de Django se encarga de esto, si el error es manejable;
algún problema en una petición, o algo no catastrófico, arroja una excepción y un
error 500, fallo interno en el servidor, y finaliza la petición; en el caso de no
poder ser recuperado finaliza el proceso.

La finalización del servidor es una práctica relativamente común ya que permite
el reinicio del proceso, esto se hace mediante una configuración interna al
momento de hacer \textbf{deploy} en el servidor, como se observa en la figura \ref{img:ProcesosLinux}
es la configuración recomendada por DigitalOcean para un reinicio
continuo, en caso de falla, a intervalos de 5 segundos.

	\begin{figure}[htb]
		\centering
        \caption{Configuración de proceso en Linux para reinicio automático de procesos}
        \includegraphics[width=\linewidth]{comprobacion_resultados/tecnicos/reinicio_servidor.png}
        \textcite{ConfiguracionProcesos}.
        \label{img:ProcesosLinux}
	\end{figure}

Por esta razón, las pruebas mas considerables son interconexiones y configuraciones
de seguridad y problemas de datos. es decir:

\begin{itemize}
    \item Prueba de no unicidad de la información, al haber múltiples
        clientes de sensorica enviando información sobre el mismo motor.

        Para esta posibilidad, se optó por una configuración que rechaza el
        segundo intento de conexión, es decir, se rechaza cualquier intento de
        conexión de motor con una conexión establecida previamente (que todavía
        exista), y su implementación se observa en la figura \ref{img:NoUnicidad}.
%
    \item Prueba de falla en la conexión de la base de datos.

        Cada servidor toma esta falla de una forma específica, en el caso del
        servidor de sensorica, esta es considerada una falla catastrófica, por
        lo que se finaliza la ejecución, como se observa en la figura
        \ref{img:NoBBDDSensorica}.

        Por otro lado, el servidor Web arroja una excepción, la cual es manejada
        internamente por Django y resuelta de forma que la petición es respondida
        con un error 500, falla interna del servidor, este error en la respuesta
        dada es manejado por el cliente y se observa en la figura
        \ref{img:NoBBDDWeb} .
%
    \item Prueba de solicitud de vista exhaustiva a un motor sin conexión
        establecida.

        Este caso corresponde al servidor de sensorica; al no haber una conexión
        con el sensor establecida no se pueden tomar las mediciones para una vista
        exhaustiva, estas deben ser en tiempo real, por lo tanto se responde
        con una error en la respuesta como se observa en la imagen \ref{img:ErrorExhaustiva}.
%
    \item Prueba de petición no completada en el cliente Web,
        información invalida (motor no existente) o error del servidor Web.

        En este caso se muestra al cliente un error por pantalla, como ventana
        emergente, y se redirige a la vista general como se observa en la figura \ref{imga:SolicitudNoCompletada}  .
\end{itemize}


	\begin{figure}[htb]
		\centering
        \caption{Prueba de error, no unicidad de información}
        \includegraphics[width=\linewidth]{comprobacion_resultados/tecnicos/NoUnicidadSensorica.png}
        logs: Izquierda cliente conectado, Centro cliente Rechazado, Derecha.
        Servidor.
        \label{img:NoUnicidad}
	\end{figure}

    \begin{figure}[htb]
		\centering
        \caption{Prueba de error, no BBDD Sensorica}
        \includegraphics[width=\linewidth]{comprobacion_resultados/tecnicos/conexionInvalidaNoBBDDSensorica.png}
        El sistema no inicia por no poder conectarse a la BBDD.
        \label{img:NoBBDDSensorica}
	\end{figure}

    \begin{figure}[htb]
		\centering
        \caption{Prueba de error, no BBDD Web}
        \includegraphics[width=\linewidth]{comprobacion_resultados/tecnicos/conexionInvalidaNoBBDDWeb.png}
        El sistema continua ejecución y resuelve el error con un error 500.
        \label{img:NoBBDDWeb}
	\end{figure}

    \begin{figure}[htb]
		\centering
        \caption{Prueba de error, petición no completada en el cliente Web}
        \includegraphics[width=\linewidth]{comprobacion_resultados/tecnicos/conexionInvalidaNoMotorSensorica.png}
        Solicitud  desde el navegador, para facilitar vista, al servidor
        sin motores conectados.
        \label{img:ErrorExhaustiva}
	\end{figure}

    \begin{figure}[htb]
		\centering
        \caption{Prueba de error, petición exhaustiva inválida}
        \includegraphics[width=\linewidth]{comprobacion_resultados/tecnicos/redireccionClienteWeb.png}
        Solicitud no completada en el cliente Web. \label{imga:SolicitudNoCompletada}
	\end{figure}

\subsubsection{Comprobación de resultados finales}
\begin{itemize}
    \item Corroboración del análisis en nivel de daño básico (vista general)
        que se ajuste a los parámetros requeridos por el cliente.
    \item Corroboración de las gráficas históricas.
    \item Corroboración de la gráfica de la transformada de Fourier.
\end{itemize}



    % UI, CLIENTE WEoB
    \begin{figure}[H]
        \centering
        \caption{Comprobación de lo niveles en los motores}
        \begin{tabular}{m{6cm}m{6cm}}
            \includegraphics[width=6cm]{comprobacion_resultados/finales/n1.png}&
            \includegraphics[width=6cm]{comprobacion_resultados/finales/n1mongo.png}\\
            \includegraphics[width=6cm]{comprobacion_resultados/finales/n2.png}&
            \includegraphics[width=6cm]{comprobacion_resultados/finales/n2mongo.png}\\
            \includegraphics[width=6cm]{comprobacion_resultados/finales/n3.png}&
            \includegraphics[width=6cm]{comprobacion_resultados/finales/n3mongo.png}
        \end{tabular}
        \label{img:NivelesMotores}
    \end{figure}

    \begin{figure}[H]
        \centering
        \caption{Comprobación de las fronteras en los motores}
        \begin{tabular}{m{6cm}m{6cm}}
            \includegraphics[width=6cm]{comprobacion_resultados/finales/f1.png}&
            \includegraphics[width=6cm]{comprobacion_resultados/finales/f1mongo.png}\\
            \includegraphics[width=6cm]{comprobacion_resultados/finales/f2.png}&
            \includegraphics[width=6cm]{comprobacion_resultados/finales/f2mongo.png}
        \end{tabular}
        \label{img:FronterasEnLosMotores}
    \end{figure}

    \begin{figure}[H]
        \centering
        \caption{Comprobación graficas historicas}
        \begin{tabular}{m{8cm}m{8cm}}
            \includegraphics[width=8cm]{comprobacion_resultados/finales/as_graficas1.png}&
            \includegraphics[width=8cm]{comprobacion_resultados/finales/vxs_graficas1.png}\\
            \includegraphics[width=8cm]{comprobacion_resultados/finales/vys_graficas1.png}&
            \includegraphics[width=8cm]{comprobacion_resultados/finales/vzs_graficas1.png}\\
            \includegraphics[width=8cm]{comprobacion_resultados/finales/graficas1Mongo1.png}&
            \includegraphics[width=8cm]{comprobacion_resultados/finales/graficas1Mongo2.png}\\
            \multicolumn{2}{c}{\includegraphics[width=12cm]{comprobacion_resultados/finales/tablaGraficas1.png}}
        \end{tabular}
        \label{img:GraficasHistoricas}
    \end{figure}

 %   \begin{figure}[H]
 %       \centering
 %       \caption{Comparación FFT python y octave}
 %       \begin{tabular}{m{16cm}}
 %           \includegraphics[width=16cm]{comprobacion_resultados/finales/comFFT1.png}\\
 %           \includegraphics[width=16cm]{comprobacion_resultados/finales/comFFT1Terminales.png}
 %       \end{tabular}
 %       \label{img:FFTMidley}
 %   \end{figure}

    \begin{figure}[H]
        \centering
        \caption{FFT Octave vs Python datos del modelo}
            \includegraphics[width=16cm]{comprobacion_resultados/finales/comFFT2.png}
        \label{img:FFTSensores}
    \end{figure}
    \endgroup


%***************************************************
%**********  Capitulo 5  ***************************
%***************************************************
    \thispagestyle{empty}

\addcontentsline{toc}{section}{CONCLUSIONES Y RECOMENDACIONES}
\section*{CONCLUSIONES Y RECOMENDACIONES}
\subsection{Conclusiones}
Tras la exposición de los resultados plasmados en el Capítulo IV, se pueden
plantear las siguientes conclusiones:

\begin{itemize}
    \item El desarrollo  de una herramienta computacional para el análisis de
        la vibración en motores eléctricos se hizo posible mediante la
        implementación de dos servidores, uno de estilo
        microservicio y un cliente Web, puede ser alimentada mediante la
        simulación de un  modelo estadístico o
        mediante mediciones a tiempo real de cualquier sistema
        que respete las estructuras de datos empleado.
        %
    \item Se implementaron múltiples servicios, cada uno escrito con un
        lenguaje de programación que satisface los requerimientos de velocidad,
        paralelismo y robustez necesarios para  los requerimientos de
        uso, además de una fácil adecuación y modificación a las necesidades,
        de múltiples clientes.
        %
    \item Existe una gran cantidad de lenguajes de programación en la actualidad
        diseñados para satisfacer una gran cantidad de tareas en múltiples campos,
        siendo Go y Python herramientas de gran poder y fácil implementación
        al realizar servidores y API, además de la creación de Script
        para la automatización de procesos; de igual forma Python es increíblemente
        polifacético y cuenta con una gran cantidad de librerías y Frameworks
        especialmente en el área de la ciencia de datos y estadística.
    %
    \item En términos de programación Web a nivel cliente,
        HTML, CSS, JavaScript, son casi una obligación, sin embargo,
        JavaScript cuenta con una  variedad de Frameworks para agilizar el
        trabajo, siendo React uno
        de los mas utilizados en la industria.
        %
    \item Se implementaron dos modelos estadísticos, uno para entregar el
        equivalente a mediciones diarias y otro para mediciones continuas a tiempo
        real necesarias para hacer un análisis en frecuencia,
        utilizando distribuciones de probabilidad para representar 3 variables de interés,
        velocidad vertical, horizontal y aceleración, expresadas de forma independiente
        como magnitud y ángulo de la velocidad y aceleración las dos primeras
        siendo representadas en coordenadas polares.

    \item Las distribuciones que satisfacen de mejor manera las variables
        modeladas fueron la distribución Burr
        de tipo III, para la magnitud de velocidad y aceleración, y la distribución
        normal para el ángulo, de esta forma se obtienen las mediciones diarias;
        con un modelo de señal no estadístico ajustado a la amplitud, dada por
        el modelo de la aceleración, mas una gaussiana, para simular ruido, se obtienen
        las mediciones continuas.
        %
    \item Se implementó una API Web, para facilitar la accesibilidad al modelo, con
        el Frameworks FastAPI y dos endpoints que entregan la información especificada
        en el Url y en los parámetros de configuración, nivel de daño (un número del 0 al 10)
        y el identificador único del motor que tiene asociado.
        %
    \item El análisis en frecuencia se realiza mediante una
        transformada rápida de Fourier (FFT) lo que permite llevar la señal
        discretizada (mediciones a tiempo real) al dominio de la frecuencia
        y posteriormente graficarla; todo esto se logra mediante las librerías
        \textbf{numpy} y \textbf{matplotlib} de Python.
        %
    \item  Se utilizó una base de datos no relacional, \textbf{MongoDB}, en forma
        de microservicio con la empresa \textbf{MongoDB Atlas};
        esta base de datos llamada \textbf{tesis}
        contiene 2 colecciones, una se encarga de almacenar la información de las
        mediciones diarias y la otra de almacenar los identificadores únicos que
        representan a los motores de los que se posee información.
        %
    \item La automatización de procesos es fundamental para agilizar las tareas y
        la forma mas sencilla y eficiente de hacerlo fue mediante un Script; para
        llenar la base de datos se utilizó uno, escrito en Go, el cual acepta
        configuración al momento de la ejecución, vía parámetros, siendo obligatorios
        la especificación de los identificadores únicos correspondientes al motor
        y los primeros 2 sensores.
        %
    \item Se crearon 2 microservicios como servidores, el primero llamado
        \textbf{servidor de sensorica}
        se encarga de la comunicación con los sensores, el segundo es el \textbf{servidor Web},
        y se ocupa de satisfacer las peticiones Web al enviar el HTML-CSS-JavaScript
        necesario para mostrar la página, los archivos estáticos e
        información requeridos por los mismos para mostrar el estado de los motores
        dependiendo del nivel de análisis solicitado.
        %
    \item Facilitar la utilización del modelo estadístico y la conexión con
        el microservicio de \textbf{sensorica} se implementó con un \textbf{cliente de
        sensorica} escrito en Go el cual permitió especificar las características
        inherentes al motor (Identificador del motor y de los sensores, características, etc)
        y al daño existente en el mismo (en una escala del 1 al 10), además,
        especificar la dirección IP, o puerto local, al cual se conectará.
        %
    \item Se crearon 3 vistas para el cliente Web, estas son:
        \begin{enumerate}
            \item \textbf{General}, con el cual se conoce el estado de todos
                los motores registrados en la base de datos, mediante paginación
                en una ventana se muestran 12 motores por vez, y controlar
                cuáles son mostrados,
                y especificar el identificador
                único del motor del cual se quiere obtener mas información.
                %
            \item \textbf{Específica},
                muestra la evolución histórica del motor mediante gráficas
                y una tabla exportable a Excel; adicionalmente permite solicitar
                la vista exhaustiva.
                %
            \item \textbf{Exhaustiva},
                solicita mediciones a tiempo real de la aceleración del motor
                y las descompone en frecuencia, permitiendo observar una gráfica
                de las frecuencias y magnitudes  en las que vibra
                el motor; adicionalmente muestra toda la información de la
                vista específica.
        \end{enumerate}
        %
    \item Se probó manualmente el sistema en los puntos claves
        e interconexiones, con lo cual  se comprobó:
        \begin{enumerate}
            \item La unicidad de la información.
            \item Fallas de conexión con el microservicio de base de datos.
            \item Solicitud de información a tiempo real a un motor sin conexión
                establecida.
            \item Peticiones invalidas en el servidor Web
        \end{enumerate}
        %
    \item Se comprobó el cumplimiento de
        los requerimientos mínimos que satisfacen las condiciones planteadas,
        mediante la
        demostración de la clasificación en los distintos niveles de daño,
         se corroboró la coherencia y legibilidad
        de las gráficas históricas y de la gráfica del análisis en frecuencia.
        %
\end{itemize}



\subsection{Recomendaciones}
\begin{itemize}
    \item Utilizar un sistema de adquisición de datos, sensores o placas diseñadas
        para la obtención de las mediciones diarias, por ejemplo los datos a tiempo
        real para alimentar el sistema.

    \item  Implementar un sistema de autenticación y seguridad para que solo
        los usuarios con los
        privilegios correctos puedan acceder a la información, esto
        probablemente requiera el uso de otra base de datos preferiblemente
        relacional, y de esta forma poder hacer uso empresarial
        de la aplicación.

    \item Un sistema de notificaciones configurable por el usuario,
        puede ser una gran ayuda al operador de la aplicación en el diagnóstico
        de fallas de forma temprana, para observar si algún motor recibe una
        medida fuera de los valores normales.

    \item Debido a la complejidad de las conexiones HTTP2 una alternativa es el
        uso de Websockets los cuales permiten una conexión bidireccional sin la
        complejidad del nuevo protocolo evaluando  los requerimientos
        de la aplicación. Otra alternativa, en un futuro
        cercano, posiblemente lo sea el HTTP3 la cual está siendo discutida
        por el comité del estándar HTTP para corregir fallas de la última
        versión.

    \item Creación y uso de  pruebas automatizadas para minimizar el tiempo de
        depuración y evitar fallas en producción, cabe destacar que esto
        implica un coste
        en el tiempo de desarrollo dedicado al diseño de las pruebas.
        La elección de usarlas dependerá de la experiencia previa
        del programador con este tipo de metodología de desarrollo.


\end{itemize}

    %Conclusiones y Recomendaciones (que concluyo del trabajo)
    %que informacion se obtuvo de cada especifico,
    %que cosas de informacion adicional se nos dio mientras
    %desarrollamos

    % 1min aprox dedicado a cada lamina
    % No necesariamente debe de seguirse el orden del trabajo

\end{refsegment}

	\newpage
	\printbibliography[segment=1,title={\centering{REFERENCIAS BIBLIOGRÁFICAS}}]
    \nocite{*}
    \printbibliography[notkeyword={b1},title={\centering{BIBLIOGRAFÍA}}]
%***************************************************
%**********  Anexos      ***************************
%***************************************************
    
\addcontentsline{toc}{section}{ANEXOS}

\phantom{}
\vfill
\begin{center}
{\Huge ANEXOS}
\end{center}
\vfill
\phantom{}

    \include{src/anexos/manualUsoInstalacion}

\end{document}
