\setcounter{page}{1}
\section{Problematica}

En la actualidad se vive un crecimiento exponencial a nivel industrial dadas las altas demandas de alimentos e insumos de toda clase; Esto es posible gracias al motor eléctrico, este es un artefacto que transforma la energía eléctrica en energía mecánica (movimiento), de manera que puede impulsar el funcionamiento de una máquina y son utilizados ampliamente en Bombas para Agua, Bombas Industriales, Mezcladoras, Molinos, Correas Transportadoras, Zarandas, Cortadoras, Ventiladores, Grúas, y en todo proceso que involuvre movimiento.\\
Si a esta diversidad de usos se le suma la gran perdida (horas, insumos,dinero, etc) que ocasiona una parada de emergencia en una planta, Se obtiene la gran inportancia de que los motores se encuentren completamente operativos y funcionales. Para procurar su buen estado se realizan mantenimientos.\\



Existen diferentes tipos de mantenimiento en la literatura, entre ellos podemos
nombrar mantenimiento correctivo, mantenimiento preventivo y mantenimiento
predictivo.  

El mantenimiento correctivo es cuando se espera que ocurra una falla para
reparar o cambiar un equipo, la cual puede ser mayor o menor. Este tipo de
mantimiento es bastante deficiciente, primero la falla puede ser una falla
mayor lo cual puede degradar la vida util del equipo y debido a que la falla
puede ocurrir en cualquier momento, es bastante probable que esto pueda
producir un paro en la linea de producción, por lo tanto este tipo de
mantenimiento suele y debe ser evitado.

El mantenimiento preventivo es cuando para evitar una falla mayor se detiene la
maquinaria para hacer un mantenimiento preparado con anticipacion, se le hace
inspeccion a la maquinaria y se remplazan las piezas que pueden ser propenzas a
dañarse. Este tipo de mantenimiento en algunas circunstancias es más que
suficiente pero en el caso de los rodamientos puede ser contraproducente
(necesito una cita) Es por esto la necesidad de un mejor método.

El mantenimiento predictivo ocurre cuando en vez de planificar una parada de
mantenimiento se predice cuando una falla esta a punto de ocurrir, es decir
realizando mediciones se puede predecir cuando una falla esta a punto de
ocurrir esto permite tener una mejor planificacion y como se menciono
anteriormente en el caso de las rolineras de los motores electricos el
mantenimiento preventivo puede ser contraproducente.

